%%%%%%%%%%%%%%%%%%%%%%%%%%%%%%%%%%%%%%%%%%%%%%%%%%%%%%%%%%%%%%%%%%%%%%%%%%%%%%%
%                           RAPPORT PFE - DocQA-MS                            %
%         Système de Question-Réponse sur Documents Médicaux                  %
%%%%%%%%%%%%%%%%%%%%%%%%%%%%%%%%%%%%%%%%%%%%%%%%%%%%%%%%%%%%%%%%%%%%%%%%%%%%%%%

\documentclass[12pt,a4paper]{report}

%==============================================================================
%                              PACKAGES
%==============================================================================

\usepackage[utf8]{inputenc}
\usepackage[T1]{fontenc}
\usepackage[french]{babel}
\usepackage{listings}

\lstset{
  backgroundcolor=\color{gray!5},
  basicstyle=\ttfamily\small,
  breaklines=true,
  frame=single,
  rulecolor=\color{gray!30}
}
\usepackage[
  top=2cm,
  bottom=2cm,
  left=2.5cm,
  right=2cm,
  headheight=14pt,
  headsep=0.5cm,
  footskip=1cm
]{geometry}
\usepackage{listings}

\lstset{
  backgroundcolor=\color{gray!5},
  basicstyle=\ttfamily\small,
  breaklines=true,
  frame=single,
  rulecolor=\color{gray!30},
  numbers=none,
  tabsize=2
}
\usepackage{lmodern}
\usepackage[table,xcdraw]{xcolor}

% Palette de couleurs bleues (thème médical/technologique)
\definecolor{bleuPrincipal}{RGB}{41, 128, 185}
\definecolor{bleuFonce}{RGB}{23, 74, 117}
\definecolor{bleuClair}{RGB}{133, 193, 233}
\definecolor{bleuTresClair}{RGB}{235, 245, 251}
\definecolor{bleuMarine}{RGB}{26, 82, 118}
\definecolor{bleuTurquoise}{RGB}{72, 201, 176}
\definecolor{bleuNuit}{RGB}{33, 47, 61}
\definecolor{grisFonce}{RGB}{64, 64, 64}
\definecolor{grisTexte}{RGB}{51, 51, 51}
\definecolor{grisClair}{RGB}{245, 245, 245}
\definecolor{blanc}{RGB}{255, 255, 255}
\definecolor{vertSucces}{RGB}{39, 174, 96}
\definecolor{orangeWarning}{RGB}{243, 156, 18}
\definecolor{rougeAlert}{RGB}{231, 76, 60}

\usepackage{graphicx}
\usepackage{float}
\usepackage{array}
\usepackage{booktabs}
\usepackage{longtable}
\usepackage{multirow}
\usepackage{tabularx}
\usepackage{colortbl}

\usepackage{tikz}
\usetikzlibrary{shapes.geometric, arrows, positioning, calc, shadows, decorations.pathmorphing, backgrounds, fit, patterns, fadings}

\usepackage[most]{tcolorbox}
\tcbuselibrary{skins, breakable}

\usepackage{fancyhdr}
\usepackage{titlesec}
\usepackage{titletoc}

\usepackage[
  colorlinks=true,
  linkcolor=bleuFonce,
  urlcolor=bleuPrincipal,
  citecolor=bleuMarine,
  bookmarks=true
]{hyperref}

\usepackage{setspace}
\usepackage{parskip}
\usepackage{enumitem}
\usepackage{microtype}
\usepackage[solid]{fontawesome5}

%==============================================================================
%                        CONFIGURATION DES STYLES
%==============================================================================

\titleformat{\chapter}[display]
{\normalfont\LARGE\bfseries\color{bleuFonce}}
{\colorbox{bleuPrincipal}{\parbox{2.5cm}{\centering\textcolor{white}{\Large Chapitre}\\\textcolor{white}{\fontsize{40}{48}\selectfont\thechapter}}}}
{1cm}
{\titlerule[2pt]\vspace{0.3cm}}

\titlespacing*{\chapter}{0pt}{-20pt}{20pt}

\titleformat{\section}
{\normalfont\Large\bfseries\color{bleuFonce}}
{\colorbox{bleuPrincipal}{\parbox{0.8cm}{\centering\textcolor{white}{\thesection}}}}
{0.4em}
{}

\titlespacing*{\section}{0pt}{15pt}{8pt}

\titleformat{\subsection}
{\normalfont\large\bfseries\color{bleuMarine}}
{\thesubsection}
{0.4em}
{}

\titlespacing*{\subsection}{0pt}{10pt}{5pt}

%==============================================================================
%                     EN-TÊTES ET PIEDS DE PAGE
%==============================================================================

\pagestyle{fancy}
\fancyhf{}
\fancyhead[L]{\textcolor{bleuPrincipal}{\textbf{DocQA-MS}}}
\fancyhead[R]{\textcolor{bleuFonce}{\leftmark}}
\fancyfoot[C]{\tikz{\fill[bleuPrincipal] (0,0) circle (0.3cm); \node[text=white, font=\small\bfseries] at (0,0) {\thepage};}}
\renewcommand{\headrulewidth}{1pt}
\renewcommand{\headrule}{\hbox to\headwidth{\color{bleuPrincipal}\leaders\hrule height \headrulewidth\hfill}}

\fancypagestyle{plain}{
  \fancyhf{}
  \fancyfoot[C]{\tikz{\fill[bleuPrincipal] (0,0) circle (0.3cm); \node[text=white, font=\small\bfseries] at (0,0) {\thepage};}}
  \renewcommand{\headrulewidth}{0pt}
}

%==============================================================================
%                        BOÎTES PERSONNALISÉES
%==============================================================================

\newtcolorbox{resumebox}[1][]{
  enhanced,
  colback=bleuTresClair,
  colframe=bleuPrincipal,
  coltitle=white,
  fonttitle=\bfseries,
  title=#1,
  arc=2mm,
  boxrule=1pt,
  left=8pt, right=8pt, top=6pt, bottom=6pt,
  breakable
}

\newtcolorbox{keywordsbox}{
  enhanced,
  colback=grisClair,
  colframe=bleuMarine,
  boxrule=1pt,
  arc=2mm,
  left=6pt, right=6pt, top=4pt, bottom=4pt,
  fontupper=\itshape
}

\newtcolorbox{infobox}[1][Information]{
  enhanced,
  colback=bleuTresClair,
  colframe=bleuPrincipal,
  coltitle=white,
  fonttitle=\bfseries\small,
  title=#1,
  arc=2mm,
  boxrule=1pt,
  left=6pt, right=6pt, top=4pt, bottom=4pt,
  attach boxed title to top left={yshift=-2mm, xshift=4mm},
  boxed title style={arc=2mm, boxrule=0pt, colback=bleuPrincipal}
}

%==============================================================================
%                     TABLE DES MATIÈRES STYLISÉE
%==============================================================================

\contentsmargin{0cm}

\titlecontents{chapter}[0pc]
{\addvspace{20pt}\color{bleuFonce}\large\bfseries}
{\colorbox{bleuPrincipal}{\parbox{1.2cm}{\centering\textcolor{white}{\thecontentslabel}}}\hspace{0.3cm}}
{}
{\hfill\bfseries\thecontentspage}

\titlecontents{section}[1.5pc]
{\addvspace{2pt}}
{\textcolor{bleuPrincipal}{\thecontentslabel}\hspace{0.3cm}}
{}
{\titlerule*[0.5pc]{.}\thecontentspage}

\titlecontents{subsection}[3pc]
{\small}
{\textcolor{bleuMarine}{\thecontentslabel}\hspace{0.3cm}}
{}
{\titlerule*[0.5pc]{.}\small\thecontentspage}

%==============================================================================
%                       DÉBUT DU DOCUMENT
%==============================================================================

\begin{document}

\pagenumbering{roman}

%%%%%%%%%%%%%%%%%%%%%%%%%%%%%%%%%%%%%%%%%%%%%%%%%%%%%%%%%%%%%%%%%%%%%%%%%%%%%%%
%                     PAGE DE GARDE INNOVANTE
%%%%%%%%%%%%%%%%%%%%%%%%%%%%%%%%%%%%%%%%%%%%%%%%%%%%%%%%%%%%%%%%%%%%%%%%%%%%%%%

\begin{titlepage}
  \begin{tikzpicture}[remember picture, overlay]

    %==========================================================================
    % FOND AVEC DÉGRADÉ ET MOTIFS
    %==========================================================================

    % Fond blanc de base
    \fill[white] (current page.south west) rectangle (current page.north east);

    % Grande forme géométrique bleue (côté droit)
    \fill[bleuPrincipal]
    ([xshift=-6cm]current page.north east) --
    (current page.north east) --
    (current page.south east) --
    ([xshift=-3cm]current page.south east) --
    ([xshift=-8cm, yshift=8cm]current page.south east) -- cycle;

    % Forme superposée plus claire
    \fill[bleuTurquoise, opacity=0.6]
    ([xshift=-4cm]current page.north east) --
    (current page.north east) --
    ([yshift=5cm]current page.south east) --
    ([xshift=-6cm, yshift=10cm]current page.south east) -- cycle;

    % Cercles décoratifs (réseau de microservices stylisé)
    \foreach \x/\y/\r in {
      -2/12/0.8, -1/10/0.5, 0/8/0.6, -1.5/6/0.4, -0.5/4/0.7,
      0.5/11/0.3, 1/9/0.5, 0.5/7/0.4, 1.5/5/0.6, 0/2/0.5
    } {
      \fill[white, opacity=0.15] ([xshift=\x cm, yshift=\y cm]current page.east) circle (\r cm);
    }

    % Lignes de connexion (style microservices network)
    \draw[white, opacity=0.2, line width=1pt]
    ([xshift=-2cm, yshift=12cm]current page.east) --
    ([xshift=-1cm, yshift=10cm]current page.east) --
    ([xshift=0cm, yshift=8cm]current page.east) --
    ([xshift=-1.5cm, yshift=6cm]current page.east) --
    ([xshift=-0.5cm, yshift=4cm]current page.east);

    \draw[white, opacity=0.2, line width=1pt]
    ([xshift=0.5cm, yshift=11cm]current page.east) --
    ([xshift=1cm, yshift=9cm]current page.east) --
    ([xshift=0.5cm, yshift=7cm]current page.east);

    %==========================================================================
    % BANDE DÉCORATIVE INFÉRIEURE
    %==========================================================================

    \fill[bleuFonce]
    (current page.south west) --
    ([yshift=1.2cm]current page.south west) --
    ([xshift=-5cm, yshift=1.2cm]current page.south east) --
    ([xshift=-5cm, yshift=0cm]current page.south east) --
    (current page.south east) -- cycle;

    % Petits hexagones décoratifs (représentant les microservices)
    \foreach \i in {0,1.5,3,4.5,6,7.5,9,10.5} {
      \node[regular polygon, regular polygon sides=6, minimum size=0.5cm,
      fill=bleuClair, opacity=0.3]
      at ([xshift=\i cm, yshift=0.6cm]current page.south west) {};
    }

    %==========================================================================
    % ÉLÉMENT DÉCORATIF GAUCHE (LIGNES FINES)
    %==========================================================================

    \foreach \y in {3,5,7,9,11,13,15,17,19,21} {
      \draw[bleuClair, opacity=0.4, line width=0.5pt]
      ([yshift=\y cm]current page.south west) --
      ([xshift=2cm, yshift=\y cm]current page.south west);
    }

    %==========================================================================
    % LOGO DOCQA-MS
    %==========================================================================

    \node at ([xshift=3.5cm, yshift=-3cm]current page.north west) {
      \includegraphics[width=3cm]{images/logo-docqa.png}
    };

  \end{tikzpicture}

  %==============================================================================
  % CONTENU DE LA PAGE
  %==============================================================================

  \vspace*{0.5cm}

  % En-tête université
  \hspace{4cm}
  \begin{minipage}{10cm}
    {\large\color{grisFonce}\textbf{École Marocaine des Sciences de l'Ingénieur}}\\[0.1cm]
    {\color{grisTexte}Département d'Informatique}\\[0.05cm]
    {\small\color{bleuMarine}Filière : Ingénierie Informatique et Réseaux}
  \end{minipage}

  \vspace{1.5cm}

  % Type de mémoire
  \hspace{1cm}
  \begin{tikzpicture}
    \node[fill=bleuTresClair, rounded corners=3pt, inner sep=8pt] {
      \color{bleuFonce}\scriptsize{Projet Académique}
    };
  \end{tikzpicture}

  % Titre principal
  \hspace{0.5cm}
  \begin{minipage}{11cm}
    % Ligne décorative
    \begin{tikzpicture}
      \fill[bleuPrincipal] (0,0) rectangle (8cm, 3pt);
      \fill[bleuTurquoise] (0,-0.15) rectangle (5cm, 2pt);
    \end{tikzpicture}

    \vspace{0.4cm}

    % Nom de l'application
    {\fontsize{50}{60}\selectfont\color{bleuFonce}\textbf{Doc}\textcolor{bleuPrincipal}{\textbf{QA-MS}}}

    \vspace{0.3cm}

    % Sous-titre
    {\LARGE\color{grisFonce}\textit{Système de Question-Réponse}}\\[0.2cm]
    {\LARGE\color{grisFonce}\textit{sur Documents Médicaux}}\\[0.2cm]
    {\Large\color{bleuMarine}\textit{Architecture Microservices}}

    \vspace{0.5cm}

    % Tags des fonctionnalités
    \begin{tikzpicture}
      \node[fill=bleuPrincipal, text=white, rounded corners=10pt,
      inner xsep=8pt, inner ysep=4pt, font=\small] (tag1) {RAG LLM};
      \node[fill=bleuTurquoise, text=white, rounded corners=10pt,
      inner xsep=8pt, inner ysep=4pt, font=\small, right=0.2cm of tag1] (tag2) {Anonymisation};
      \node[fill=bleuMarine, text=white, rounded corners=10pt,
      inner xsep=8pt, inner ysep=4pt, font=\small, right=0.2cm of tag2] (tag3) {Indexation};
      \node[fill=bleuFonce, text=white, rounded corners=10pt,
      inner xsep=8pt, inner ysep=4pt, font=\small, right=0.2cm of tag3] (tag4) {Synthèse};
    \end{tikzpicture}
  \end{minipage}

  \vspace{1.2cm}

  % Section Réalisé par / Encadré par
  \hspace{0.5cm}
  \begin{minipage}[t]{6.5cm}
    \begin{tikzpicture}
      % Titre
      \node[fill=bleuPrincipal, text=white, font=\bfseries,
      inner xsep=15pt, inner ysep=6pt, rounded corners=2pt]
      at (0,0) {Réalisé par};
    \end{tikzpicture}

    \vspace{0.4cm}

    \hspace{0.3cm}
    \begin{tabular}{@{}l@{}}
      \textcolor{bleuFonce}{\faUser} \hspace{0.2cm} \textbf{Achraf EL HOUFI}\\[0.25cm]
      \textcolor{bleuFonce}{\faUser} \hspace{0.2cm} \textbf{Saad KARZOUZ}\\[0.25cm]
      \textcolor{bleuFonce}{\faUser} \hspace{0.2cm} \textbf{Yassir LAMBRASS}\\[0.25cm]
      \textcolor{bleuFonce}{\faUser} \hspace{0.2cm} \textbf{Anas EL MALYARI}
    \end{tabular}
  \end{minipage}
  \hfill
  \begin{minipage}[t]{6.5cm}

    \vspace{0.4cm}

    \hspace{0.3cm}

  \end{minipage}
  \hspace{1cm}

  \vfill

  % Année universitaire
  \begin{tikzpicture}
    \node[inner sep=0] at (0,0) {
      \hspace{1cm}
      \begin{tikzpicture}
        \draw[bleuPrincipal, line width=2pt] (0,0) -- (3,0);
        \node[fill=white, text=bleuFonce, font=\Large\bfseries, inner xsep=15pt]
        at (6.5,0) {2025 — 2026};
        \draw[bleuPrincipal, line width=2pt] (10,0) -- (13,0);
      \end{tikzpicture}
    };
  \end{tikzpicture}

  \vspace{1cm}

\end{titlepage}

\newpage
\chapter*{Remerciements}
\addcontentsline{toc}{chapter}{Remerciements}

\vspace{-0.5cm}

\begin{center}
  {\fontsize{30}{36}\selectfont\color{bleuPrincipal}\textit{"Merci"}}\\[0.2cm]
  \begin{tikzpicture}
    \draw[bleuClair, line width=1.5pt] (0,0) -- (4,0);
  \end{tikzpicture}
\end{center}

\vspace{0.3cm}

Au terme de ce projet de fin d'études, nous tenons à exprimer notre profonde gratitude envers toutes les personnes qui ont contribué, de près ou de loin, à la réalisation de ce travail. Ce mémoire représente l'aboutissement de plusieurs mois d'efforts, de recherches et de développement, et il n'aurait pas pu voir le jour sans le soutien précieux de nombreuses personnes.

\begin{infobox}[À notre encadrant]
  Nous adressons nos remerciements les plus sincères à \textbf{nos Professeurs}, nos encadrants académique, pour leur disponibilité constante, leur conseils avisés et leur accompagnement bienveillant tout au long de ce projet. leur rigueur scientifique, leur expertise technique en intelligence artificielle et en architecture distribuée, ainsi que leur patience ont été des atouts inestimables qui nous ont permis de mener à bien ce travail. leur orientations pertinentes nous ont guidés dans les moments de doute et nous ont aidés à surmonter les obstacles techniques rencontrés, notamment dans l'intégration des modèles de langage et la conception de l'architecture microservices.
\end{infobox}

\vspace{0.3cm}

Nous exprimons également notre reconnaissance à l'ensemble du \textbf{corps professoral} du département d'informatique qui nous a transmis, au fil des années, les connaissances et compétences nécessaires pour aborder ce projet avec confiance. Leur dévouement à l'enseignement et leur passion pour l'informatique nous ont inspirés et motivés.

Nous ne saurions oublier nos \textbf{familles} qui nous ont soutenus inconditionnellement durant toute notre formation. Leur amour, leurs encouragements et leurs sacrifices ont été le pilier de notre réussite. À nos parents, nos frères et sœurs, nous disons merci du fond du cœur.

Enfin, nous remercions nos \textbf{amis et collègues} pour leur soutien moral, leurs conseils et les moments de partage qui ont rendu cette aventure plus agréable. La cohésion de notre équipe a été un facteur clé dans la réussite de ce projet.

\begin{center}
  \vspace{0.3cm}
  \textit{\color{bleuMarine}À toutes et à tous, nous exprimons notre profonde gratitude.}
\end{center}

%%%%%%%%%%%%%%%%%%%%%%%%%%%%%%%%%%%%%%%%%%%%%%%%%%%%%%%%%%%%%%%%%%%%%%%%%%%%%%%
%                           RÉSUMÉ
%%%%%%%%%%%%%%%%%%%%%%%%%%%%%%%%%%%%%%%%%%%%%%%%%%%%%%%%%%%%%%%%%%%%%%%%%%%%%%%

\newpage
\chapter*{Résumé}
\addcontentsline{toc}{chapter}{Résumé}

\vspace{-0.5cm}

\begin{resumebox}[Résumé en Français]

  L'accès rapide et sécurisé à l'information médicale constitue aujourd'hui un enjeu majeur pour les professionnels de santé. Dans un contexte où le volume de données médicales croît exponentiellement, les cliniciens font face à un défi de taille : extraire rapidement les informations pertinentes tout en garantissant la confidentialité des données patients. Face à cette problématique, les avancées en intelligence artificielle, notamment les modèles de langage de grande taille (LLM) et les systèmes RAG (Retrieval-Augmented Generation), offrent des perspectives prometteuses.

  \vspace{0.3cm}

  C'est dans ce contexte que s'inscrit le projet \textbf{DocQA-MS}, un système de Question-Réponse sur Documents Médicaux basé sur une architecture microservices. Cette solution innovante permet aux professionnels de santé d'interroger naturellement des corpus de documents cliniques tout en garantissant l'anonymisation automatique des données sensibles conformément au RGPD. L'application propose une ingestion intelligente de documents, une anonymisation automatique via la reconnaissance d'entités nommées médicales (NER), une indexation sémantique vectorielle, un moteur de question-réponse propulsé par LLM (Llama 3.1 via Ollama), une synthèse comparative multi-documents, ainsi qu'un système complet d'audit et de traçabilité.

  \vspace{0.3cm}

  Le développement de DocQA-MS s'appuie sur une architecture technique moderne et distribuée. Le backend comprend sept microservices développés en Python (FastAPI) et Java (Spring Boot), orchestrés via Docker Compose. L'infrastructure intègre PostgreSQL pour la persistance, RabbitMQ pour la communication asynchrone, et Ollama pour l'exécution locale des modèles LLM. L'interface utilisateur a été développée en React. La méthodologie Scrum a été adoptée pour assurer une gestion agile du projet.

  \vspace{0.3cm}

  Les résultats obtenus démontrent la faisabilité et l'efficacité d'une telle approche. DocQA-MS offre une solution complète et conforme aux exigences de sécurité du domaine médical, permettant aux cliniciens d'accéder instantanément aux informations pertinentes sans compromettre la vie privée des patients.

\end{resumebox}

\vspace{0.4cm}

\begin{keywordsbox}
  \textbf{Mots-clés :} Documents Médicaux, Intelligence Artificielle, RAG, LLM, Microservices, Anonymisation, NER, RGPD, Docker, Spring Boot, FastAPI, React, PostgreSQL, RabbitMQ.
\end{keywordsbox}

%%%%%%%%%%%%%%%%%%%%%%%%%%%%%%%%%%%%%%%%%%%%%%%%%%%%%%%%%%%%%%%%%%%%%%%%%%%%%%%
%                           ABSTRACT
%%%%%%%%%%%%%%%%%%%%%%%%%%%%%%%%%%%%%%%%%%%%%%%%%%%%%%%%%%%%%%%%%%%%%%%%%%%%%%%

\newpage
\chapter*{Abstract}
\addcontentsline{toc}{chapter}{Abstract}

\vspace{-0.5cm}

\begin{resumebox}[English Abstract]

  Fast and secure access to medical information is today a major challenge for healthcare professionals. In a context where the volume of medical data is growing exponentially, clinicians face a significant challenge: quickly extracting relevant information while ensuring patient data confidentiality. Faced with this problem, advances in artificial intelligence, particularly Large Language Models (LLM) and RAG (Retrieval-Augmented Generation) systems, offer promising perspectives.

  \vspace{0.3cm}

  It is within this context that the \textbf{DocQA-MS} project was developed, a Question-Answering System on Medical Documents based on a microservices architecture. This innovative solution allows healthcare professionals to naturally query clinical document corpora while ensuring automatic anonymization of sensitive data in compliance with GDPR. The application offers intelligent document ingestion, automatic anonymization via Medical Named Entity Recognition (NER), vector semantic indexing, a question-answering engine powered by LLM (Llama 3.1 via Ollama), multi-document comparative synthesis, as well as a complete audit and traceability system.

  \vspace{0.3cm}

  The development of DocQA-MS is based on a modern and distributed technical architecture. The backend comprises seven microservices developed in Python (FastAPI) and Java (Spring Boot), orchestrated via Docker Compose. The infrastructure integrates PostgreSQL for persistence, RabbitMQ for asynchronous communication, and Ollama for local LLM model execution. The user interface was developed in React. The Scrum methodology was adopted to ensure agile project management.

  \vspace{0.3cm}

  The results obtained demonstrate the feasibility and effectiveness of such an approach. DocQA-MS offers a complete solution compliant with the security requirements of the medical domain, allowing clinicians to instantly access relevant information without compromising patient privacy.

\end{resumebox}

\vspace{0.4cm}

\begin{keywordsbox}
  \textbf{Keywords:} Medical Documents, Artificial Intelligence, RAG, LLM, Microservices, Anonymization, NER, GDPR, Docker, Spring Boot, FastAPI, React, PostgreSQL, RabbitMQ.
\end{keywordsbox}

%%%%%%%%%%%%%%%%%%%%%%%%%%%%%%%%%%%%%%%%%%%%%%%%%%%%%%%%%%%%%%%%%%%%%%%%%%%%%%%
%                        TABLE DES MATIÈRES
%%%%%%%%%%%%%%%%%%%%%%%%%%%%%%%%%%%%%%%%%%%%%%%%%%%%%%%%%%%%%%%%%%%%%%%%%%%%%%%

\tableofcontents

%%%%%%%%%%%%%%%%%%%%%%%%%%%%%%%%%%%%%%%%%%%%%%%%%%%%%%%%%%%%%%%%%%%%%%%%%%%%%%%
%                        LISTE DES FIGURES
%%%%%%%%%%%%%%%%%%%%%%%%%%%%%%%%%%%%%%%%%%%%%%%%%%%%%%%%%%%%%%%%%%%%%%%%%%%%%%%

\listoffigures

%%%%%%%%%%%%%%%%%%%%%%%%%%%%%%%%%%%%%%%%%%%%%%%%%%%%%%%%%%%%%%%%%%%%%%%%%%%%%%%
%                        LISTE DES TABLEAUX
%%%%%%%%%%%%%%%%%%%%%%%%%%%%%%%%%%%%%%%%%%%%%%%%%%%%%%%%%%%%%%%%%%%%%%%%%%%%%%%

\listoftables

%%%%%%%%%%%%%%%%%%%%%%%%%%%%%%%%%%%%%%%%%%%%%%%%%%%%%%%%%%%%%%%%%%%%%%%%%%%%%%%
%                     LISTE DES ABRÉVIATIONS
%%%%%%%%%%%%%%%%%%%%%%%%%%%%%%%%%%%%%%%%%%%%%%%%%%%%%%%%%%%%%%%%%%%%%%%%%%%%%%%

\newpage
\chapter*{Liste des Abréviations}
\addcontentsline{toc}{chapter}{Liste des Abréviations}

\vspace{-0.5cm}

\begin{center}
  \begin{tikzpicture}
    \node[fill=bleuPrincipal, text=white, rounded corners=3pt, inner sep=8pt, font=\large\bfseries] {Abréviations et Acronymes};
  \end{tikzpicture}
\end{center}

\vspace{0.4cm}

\renewcommand{\arraystretch}{1.3}
\begin{center}
  \begin{tabular}{>{\columncolor{bleuTresClair}\bfseries}p{2.5cm} p{11cm}}
    \toprule
    \rowcolor{bleuPrincipal}
    \textcolor{white}{\textbf{Abrév.}} & \textcolor{white}{\textbf{Signification}} \\
    \midrule
    API & \textit{Application Programming Interface} – Interface de programmation applicative. \\
    \rowcolor{white}
    CRUD & \textit{Create, Read, Update, Delete} – Opérations de base pour la gestion des données. \\
    DeID & \textit{De-Identification} – Processus d'anonymisation des données patients. \\
    \rowcolor{white}
    IA & \textit{Intelligence Artificielle} – Simulation de l'intelligence humaine par les machines. \\
    JWT & \textit{JSON Web Token} – Standard pour les jetons d'accès sécurisés. \\
    \rowcolor{white}
    LLM & \textit{Large Language Model} – Modèle de langage de grande taille. \\
    NER & \textit{Named Entity Recognition} – Reconnaissance d'entités nommées. \\
    \rowcolor{white}
    RAG & \textit{Retrieval-Augmented Generation} – Génération augmentée par recherche. \\
    REST & \textit{Representational State Transfer} – Architecture pour les API web. \\
    \rowcolor{white}
    RGPD & \textit{Règlement Général sur la Protection des Données} – Réglementation européenne. \\
    UML & \textit{Unified Modeling Language} – Langage de modélisation graphique. \\
    \rowcolor{white}
    CI/CD & \textit{Continuous Integration/Continuous Deployment} – Intégration et déploiement continus. \\
    HTTP & \textit{HyperText Transfer Protocol} – Protocole de transfert web. \\
    \rowcolor{white}
    SQL & \textit{Structured Query Language} – Langage de requête pour BDD. \\
    AMQP & \textit{Advanced Message Queuing Protocol} – Protocole de messagerie (RabbitMQ). \\
    \rowcolor{white}
    PHI & \textit{Protected Health Information} – Informations de santé protégées. \\
    \bottomrule
  \end{tabular}
\end{center}

%%%%%%%%%%%%%%%%%%%%%%%%%%%%%%%%%%%%%%%%%%%%%%%%%%%%%%%%%%%%%%%%%%%%%%%%%%%%%%%
%                        INTRODUCTION GÉNÉRALE
%%%%%%%%%%%%%%%%%%%%%%%%%%%%%%%%%%%%%%%%%%%%%%%%%%%%%%%%%%%%%%%%%%%%%%%%%%%%%%%

%%%%%%%%%%%%%%%%%%%%%%%%%%%%%%%%%%%%%%%%%%%%%%%%%%%%%%%%%%%%%%%%%%%%%%%%%%%%%%%
%                        INTRODUCTION GÉNÉRALE
%%%%%%%%%%%%%%%%%%%%%%%%%%%%%%%%%%%%%%%%%%%%%%%%%%%%%%%%%%%%%%%%%%%%%%%%%%%%%%%

\chapter*{Introduction Générale}
\addcontentsline{toc}{chapter}{Introduction Générale}
\markboth{Introduction Générale}{Introduction Générale}

%==============================================================================
% Décoration de page pour l'introduction
%==============================================================================

\begin{tikzpicture}[remember picture, overlay]
  % Bande verticale gauche
  \fill[bleuPrincipal]
  ([xshift=0.3cm]current page.north west) rectangle
  ([xshift=0.6cm, yshift=-5cm]current page.north west);
  % Cercle décoratif
  \fill[bleuClair, opacity=0.2]
  ([xshift=3cm, yshift=-3cm]current page.north west) circle (2cm);
\end{tikzpicture}

\vspace{-0.5cm}

%==============================================================================
% Citation d'ouverture
%==============================================================================

\begin{center}
  \begin{tikzpicture}
    \node[fill=bleuTresClair, rounded corners=5pt, inner sep=15pt,
    text width=12cm, align=center] {
      {\Large\color{bleuPrincipal}"}\hspace{0.1cm}
      {\itshape\color{grisFonce}Les données sont le nouveau pétrole. Mais comme le pétrole,
      les données sont précieuses, et si elles ne sont pas raffinées, elles ne peuvent pas vraiment être utilisées.}
      \hspace{0.1cm}{\Large\color{bleuPrincipal}"}
      \\[0.3cm]
      {\small\color{bleuMarine}— Clive Humby, Mathématicien britannique}
    };
  \end{tikzpicture}
\end{center}

\vspace{0.5cm}

%==============================================================================
% SECTION 1 : CONTEXTE GÉNÉRAL
%==============================================================================

\section*{\textcolor{bleuFonce}{\faGlobeAfrica\hspace{0.3cm}Contexte Général}}
\addcontentsline{toc}{section}{Contexte Général}

La transformation numérique du secteur de la santé représente aujourd'hui l'un des enjeux majeurs du XXI\textsuperscript{e} siècle. Dans un monde où le volume de données médicales croît de manière exponentielle, les professionnels de santé sont confrontés à un défi de taille : exploiter efficacement cette masse d'informations pour améliorer la prise en charge des patients. Les dossiers médicaux électroniques, les comptes-rendus d'examens, les protocoles de soins et les publications scientifiques constituent un patrimoine informationnel considérable, mais souvent difficile d'accès et sous-exploité.

\vspace{0.3cm}

\begin{tcolorbox}[
    enhanced,
    colback=white,
    colframe=bleuPrincipal,
    boxrule=1.5pt,
    arc=3mm,
    left=10pt, right=10pt, top=10pt, bottom=10pt,
    shadow={2mm}{-2mm}{0mm}{black!20},
    title={\textcolor{white}{\faDatabase\hspace{0.2cm}L'Explosion des Données Médicales}},
    fonttitle=\bfseries,
    coltitle=white,
    attach boxed title to top left={yshift=-3mm, xshift=5mm},
    boxed title style={colback=bleuPrincipal, arc=2mm}
  ]

  Les chiffres sont éloquents et illustrent l'ampleur du défi que représente la gestion des données médicales à l'échelle mondiale :

  \vspace{0.3cm}

  \begin{minipage}[t]{0.48\textwidth}
    \begin{tikzpicture}
      \fill[bleuPrincipal] (0,0) circle (0.8cm);
      \node[text=white, font=\large\bfseries] at (0,0) {30\%};
    \end{tikzpicture}
    \hspace{0.3cm}
    \begin{minipage}[t]{5cm}
      \textbf{30\% des données mondiales} sont générées par le secteur de la santé, faisant de lui le plus grand producteur de données.
    \end{minipage}
  \end{minipage}
  \hfill
  \begin{minipage}[t]{0.48\textwidth}
    \begin{tikzpicture}
      \fill[bleuTurquoise] (0,0) circle (0.8cm);
      \node[text=white, font=\large\bfseries] at (0,0) {2.3};
    \end{tikzpicture}
    \hspace{0.3cm}
    \begin{minipage}[t]{5cm}
      \textbf{2.3 exaoctets} de données médicales sont générés chaque année, soit plus que tous les livres jamais écrits.
    \end{minipage}
  \end{minipage}

  \vspace{0.4cm}

  \begin{minipage}[t]{0.48\textwidth}
    \begin{tikzpicture}
      \fill[bleuMarine] (0,0) circle (0.8cm);
      \node[text=white, font=\large\bfseries] at (0,0) {80\%};
    \end{tikzpicture}
    \hspace{0.3cm}
    \begin{minipage}[t]{5cm}
      \textbf{80\% des données médicales} sont non structurées (textes, images, audio), rendant leur exploitation complexe.
    \end{minipage}
  \end{minipage}
  \hfill
  \begin{minipage}[t]{0.48\textwidth}
    \begin{tikzpicture}
      \fill[bleuFonce] (0,0) circle (0.8cm);
      \node[text=white, font=\large\bfseries] at (0,0) {97\%};
    \end{tikzpicture}
    \hspace{0.3cm}
    \begin{minipage}[t]{5cm}
      \textbf{97\% des données collectées} restent inexploitées, représentant un potentiel considérable pour l'amélioration des soins.
    \end{minipage}
  \end{minipage}

\end{tcolorbox}

\vspace{0.4cm}

Ces chiffres révèlent un paradoxe fondamental : alors que les établissements de santé accumulent des quantités massives d'informations, les cliniciens peinent à accéder rapidement aux données pertinentes lors de la prise en charge des patients. Un médecin hospitalier consacre en moyenne \textbf{deux heures par jour} à la recherche d'informations dans les dossiers patients, temps qui pourrait être dédié aux soins directs.

\vspace{0.3cm}

\begin{tcolorbox}[
    enhanced,
    colback=bleuTresClair,
    colframe=bleuClair,
    boxrule=1pt,
    arc=2mm,
    left=8pt, right=8pt, top=8pt, bottom=8pt
  ]
  \textbf{\textcolor{bleuFonce}{\faBrain\hspace{0.2cm}L'Avènement de l'Intelligence Artificielle en Santé}}

  \vspace{0.2cm}

  L'émergence des technologies d'intelligence artificielle, et notamment des \textbf{grands modèles de langage (LLM)}, ouvre des perspectives révolutionnaires pour le traitement automatisé des documents médicaux. Ces modèles, capables de comprendre et de générer du langage naturel avec une précision remarquable, permettent d'envisager des systèmes de question-réponse intelligents sur des corpus documentaires volumineux.

  \vspace{0.2cm}

  La technologie \textbf{RAG (Retrieval-Augmented Generation)} combine la puissance des LLM avec des systèmes de recherche vectorielle, permettant de générer des réponses contextualisées et sourcées à partir de documents spécifiques. Cette approche est particulièrement adaptée au domaine médical, où la précision et la traçabilité des informations sont critiques.

  \vspace{0.2cm}

  Cependant, l'utilisation de l'IA en santé soulève des questions cruciales de \textbf{confidentialité} et de \textbf{protection des données}. Le Règlement Général sur la Protection des Données (RGPD) impose des contraintes strictes sur le traitement des données de santé, classées comme sensibles. Toute solution technique doit donc intégrer des mécanismes robustes d'anonymisation pour garantir la vie privée des patients.
\end{tcolorbox}

\vspace{0.3cm}

L'architecture microservices s'impose aujourd'hui comme le paradigme de choix pour le développement d'applications complexes et évolutives. En décomposant le système en services indépendants et faiblement couplés, cette approche permet une meilleure scalabilité, une maintenance simplifiée et une résilience accrue face aux pannes. Pour un système de traitement de documents médicaux manipulant des données sensibles, cette architecture offre également l'avantage d'isoler les composants critiques liés à la sécurité.

%==============================================================================
% SECTION 2 : PROBLÉMATIQUE
%==============================================================================

\newpage

\section*{\textcolor{bleuFonce}{\faQuestionCircle\hspace{0.3cm}Problématique}}
\addcontentsline{toc}{section}{Problématique}

Face au contexte décrit précédemment, les professionnels de santé font face à plusieurs obstacles majeurs qui entravent l'exploitation efficace des documents médicaux. Ces barrières constituent le cœur de la problématique à laquelle notre projet tente d'apporter une réponse technologique innovante.

\vspace{0.4cm}

\begin{tikzpicture}
  % Titre central
  \node[fill=bleuPrincipal, text=white, rounded corners=5pt,
  inner sep=12pt, font=\large\bfseries] (center)
  {Obstacles à l'Exploitation des Documents Médicaux};
\end{tikzpicture}

\vspace{0.4cm}

%--- Obstacle 1 ---
\begin{tcolorbox}[
    enhanced,
    colback=white,
    colframe=bleuPrincipal,
    boxrule=0pt,
    borderline west={4pt}{0pt}{bleuPrincipal},
    arc=0mm,
    left=10pt, right=10pt, top=8pt, bottom=8pt,
    shadow={1mm}{-1mm}{0mm}{black!15}
  ]
  \begin{minipage}[c]{0.08\textwidth}
    \begin{tikzpicture}
      \node[fill=bleuPrincipal, text=white, circle, minimum size=1cm,
      font=\large\bfseries] {1};
    \end{tikzpicture}
  \end{minipage}
  \hfill
  \begin{minipage}[c]{0.88\textwidth}
    \textbf{\large\textcolor{bleuFonce}{Volume et Hétérogénéité des Documents}}

    \vspace{0.2cm}

    Les établissements de santé accumulent des milliers de documents aux formats variés : comptes-rendus d'hospitalisation, résultats d'examens biologiques, notes de consultations, protocoles thérapeutiques, courriers médicaux. Cette masse documentaire hétérogène rend la recherche d'information extrêmement chronophage. Un médecin peut devoir consulter des dizaines de documents pour reconstituer l'historique complet d'un patient, avec un risque non négligeable de manquer une information cruciale. Les systèmes de recherche traditionnels, basés sur des mots-clés, s'avèrent insuffisants face à la richesse sémantique du langage médical.
  \end{minipage}
\end{tcolorbox}

\vspace{0.3cm}

%--- Obstacle 2 ---
\begin{tcolorbox}[
    enhanced,
    colback=white,
    colframe=bleuTurquoise,
    boxrule=0pt,
    borderline west={4pt}{0pt}{bleuTurquoise},
    arc=0mm,
    left=10pt, right=10pt, top=8pt, bottom=8pt,
    shadow={1mm}{-1mm}{0mm}{black!15}
  ]
  \begin{minipage}[c]{0.08\textwidth}
    \begin{tikzpicture}
      \node[fill=bleuTurquoise, text=white, circle, minimum size=1cm,
      font=\large\bfseries] {2};
    \end{tikzpicture}
  \end{minipage}
  \hfill
  \begin{minipage}[c]{0.88\textwidth}
    \textbf{\large\textcolor{bleuFonce}{Confidentialité et Conformité Réglementaire}}

    \vspace{0.2cm}

    Les données de santé sont parmi les plus sensibles qui existent. Le RGPD les classe comme données à caractère personnel sensible, imposant des mesures de protection renforcées. Toute exploitation automatisée de documents médicaux doit garantir l'anonymisation des informations identifiantes (noms, dates de naissance, numéros de sécurité sociale, adresses). Les solutions existantes proposent rarement une anonymisation intégrée et certifiée, obligeant les établissements à des traitements manuels coûteux ou à renoncer à l'exploitation de leurs données. Ce dilemme entre utilité des données et protection de la vie privée freine considérablement l'innovation en santé numérique.
  \end{minipage}
\end{tcolorbox}

\vspace{0.3cm}

%--- Obstacle 3 ---
\begin{tcolorbox}[
    enhanced,
    colback=white,
    colframe=bleuMarine,
    boxrule=0pt,
    borderline west={4pt}{0pt}{bleuMarine},
    arc=0mm,
    left=10pt, right=10pt, top=8pt, bottom=8pt,
    shadow={1mm}{-1mm}{0mm}{black!15}
  ]
  \begin{minipage}[c]{0.08\textwidth}
    \begin{tikzpicture}
      \node[fill=bleuMarine, text=white, circle, minimum size=1cm,
      font=\large\bfseries] {3};
    \end{tikzpicture}
  \end{minipage}
  \hfill
  \begin{minipage}[c]{0.88\textwidth}
    \textbf{\large\textcolor{bleuFonce}{Absence de Systèmes Q\&A Intelligents Adaptés}}

    \vspace{0.2cm}

    Les moteurs de recherche classiques déployés dans les systèmes d'information hospitaliers ne permettent pas d'interroger les documents en langage naturel. Un clinicien ne peut pas demander : "Quels sont les antécédents cardiaques de ce patient ?" ou "Y a-t-il eu des interactions médicamenteuses signalées ?". Les réponses à ces questions nécessitent une lecture manuelle et une synthèse par le praticien. Les solutions de chatbot génériques (ChatGPT, Claude) ne peuvent pas être utilisées directement car elles nécessitent l'envoi de données confidentielles vers des serveurs externes, ce qui est incompatible avec les exigences de sécurité du domaine médical.
  \end{minipage}
\end{tcolorbox}

\vspace{0.3cm}

%--- Obstacle 4 ---
\begin{tcolorbox}[
    enhanced,
    colback=white,
    colframe=bleuFonce,
    boxrule=0pt,
    borderline west={4pt}{0pt}{bleuFonce},
    arc=0mm,
    left=10pt, right=10pt, top=8pt, bottom=8pt,
    shadow={1mm}{-1mm}{0mm}{black!15}
  ]
  \begin{minipage}[c]{0.08\textwidth}
    \begin{tikzpicture}
      \node[fill=bleuFonce, text=white, circle, minimum size=1cm,
      font=\large\bfseries] {4};
    \end{tikzpicture}
  \end{minipage}
  \hfill
  \begin{minipage}[c]{0.88\textwidth}
    \textbf{\large\textcolor{bleuFonce}{Traçabilité et Audit des Accès}}

    \vspace{0.2cm}

    Dans le domaine médical, la traçabilité des accès aux données est une obligation légale et une nécessité clinique. Qui a consulté quel document ? Quelles requêtes ont été effectuées ? Ces informations sont cruciales pour la sécurité des données et l'investigation en cas d'incident. Les systèmes actuels offrent rarement une granularité suffisante dans leurs journaux d'audit, et l'intégration de cette fonctionnalité dans une architecture distribuée représente un défi technique significatif. Sans traçabilité complète, les établissements s'exposent à des risques juridiques et à des sanctions en cas de violation de données.
  \end{minipage}
\end{tcolorbox}

\vspace{0.5cm}

\begin{center}
  \begin{tikzpicture}
    \node[fill=bleuTresClair, draw=bleuPrincipal, line width=1.5pt,
    rounded corners=8pt, inner sep=15pt, text width=14cm, align=center] {
      {\large\color{bleuFonce}\textbf{Question Centrale}}\\[0.3cm]
      {\itshape\color{grisTexte}Comment concevoir un système de question-réponse intelligent sur documents médicaux, basé sur une architecture microservices, qui permette aux professionnels de santé d'interroger naturellement leurs corpus documentaires tout en garantissant l'anonymisation automatique des données sensibles, la traçabilité complète des accès, et l'exécution locale des modèles d'IA pour préserver la confidentialité ?}
    };
  \end{tikzpicture}
\end{center}

%==============================================================================
% SECTION 3 : OBJECTIFS DU PROJET
%==============================================================================

\newpage

\section*{\textcolor{bleuFonce}{\faBullseye\hspace{0.3cm}Objectifs du Projet}}
\addcontentsline{toc}{section}{Objectifs du Projet}

Pour répondre à la problématique identifiée, le projet DocQA-MS a été conçu avec des objectifs clairement définis, articulés autour d'une vision ambitieuse : offrir aux professionnels de santé un outil intelligent et sécurisé pour exploiter leurs documents médicaux.

\vspace{0.4cm}

%--- Objectif Principal ---
\begin{tcolorbox}[
    enhanced,
    colback=bleuPrincipal,
    colframe=bleuFonce,
    boxrule=0pt,
    arc=4mm,
    left=15pt, right=15pt, top=12pt, bottom=12pt,
    shadow={3mm}{-3mm}{0mm}{black!30}
  ]
  \begin{center}
    {\Large\color{white}\faStar\hspace{0.3cm}\textbf{Objectif Principal}\hspace{0.3cm}\faStar}
  \end{center}

  \vspace{0.2cm}

  {\color{white}
    Développer \textbf{DocQA-MS}, un système de Question-Réponse sur Documents Médicaux basé sur une architecture microservices. Ce système intègre un moteur de recherche sémantique propulsé par la technologie RAG (Retrieval-Augmented Generation) et un modèle de langage local (Llama 3.1 via Ollama), permettant aux cliniciens d'interroger naturellement leurs corpus documentaires en langage naturel. L'objectif est de fournir une solution complète incluant l'ingestion de documents, l'anonymisation automatique conforme au RGPD, l'indexation sémantique, la génération de réponses contextualisées, la synthèse comparative multi-documents, et un système robuste d'audit et de traçabilité.
  }
\end{tcolorbox}

\vspace{0.5cm}

%--- Objectifs Spécifiques ---
\begin{center}
  \begin{tikzpicture}
    \node[fill=bleuFonce, text=white, rounded corners=3pt,
    inner sep=10pt, font=\large\bfseries]
    {\faListUl\hspace{0.3cm}Objectifs Spécifiques};
  \end{tikzpicture}
\end{center}

\vspace{0.4cm}

\begin{minipage}[t]{0.48\textwidth}
  \begin{tcolorbox}[
      enhanced,
      colback=white,
      colframe=bleuPrincipal,
      boxrule=1pt,
      arc=3mm,
      left=8pt, right=8pt, top=8pt, bottom=8pt,
      title={\textcolor{white}{\faFileAlt\hspace{0.2cm}Ingestion de Documents}},
      fonttitle=\bfseries\small,
      coltitle=white,
      attach boxed title to top center={yshift=-2mm},
      boxed title style={colback=bleuPrincipal, arc=2mm}
    ]
    \begin{itemize}[leftmargin=*, itemsep=3pt]
      \item Support multi-formats (PDF, TXT, DOCX)
      \item Extraction de texte intelligente
      \item Chunking sémantique optimisé
      \item Gestion des métadonnées
    \end{itemize}
  \end{tcolorbox}
\end{minipage}
\hfill
\begin{minipage}[t]{0.48\textwidth}
  \begin{tcolorbox}[
      enhanced,
      colback=white,
      colframe=bleuTurquoise,
      boxrule=1pt,
      arc=3mm,
      left=8pt, right=8pt, top=8pt, bottom=8pt,
      title={\textcolor{white}{\faUserSecret\hspace{0.2cm}Anonymisation (DeID)}},
      fonttitle=\bfseries\small,
      coltitle=white,
      attach boxed title to top center={yshift=-2mm},
      boxed title style={colback=bleuTurquoise, arc=2mm}
    ]
    \begin{itemize}[leftmargin=*, itemsep=3pt]
      \item Détection NER d'entités médicales
      \item Masquage des données personnelles
      \item Conformité RGPD automatisée
      \item Traçabilité des anonymisations
    \end{itemize}
  \end{tcolorbox}
\end{minipage}

\vspace{0.4cm}

\begin{minipage}[t]{0.48\textwidth}
  \begin{tcolorbox}[
      enhanced,
      colback=white,
      colframe=bleuMarine,
      boxrule=1pt,
      arc=3mm,
      left=8pt, right=8pt, top=8pt, bottom=8pt,
      title={\textcolor{white}{\faSearch\hspace{0.2cm}Indexation Sémantique}},
      fonttitle=\bfseries\small,
      coltitle=white,
      attach boxed title to top center={yshift=-2mm},
      boxed title style={colback=bleuMarine, arc=2mm}
    ]
    \begin{itemize}[leftmargin=*, itemsep=3pt]
      \item Génération d'embeddings vectoriels
      \item Indexation dans base vectorielle
      \item Recherche par similarité sémantique
      \item Support multi-documents
    \end{itemize}
  \end{tcolorbox}
\end{minipage}
\hfill
\begin{minipage}[t]{0.48\textwidth}
  \begin{tcolorbox}[
      enhanced,
      colback=white,
      colframe=bleuFonce,
      boxrule=1pt,
      arc=3mm,
      left=8pt, right=8pt, top=8pt, bottom=8pt,
      title={\textcolor{white}{\faRobot\hspace{0.2cm}Q\&A par LLM (RAG)}},
      fonttitle=\bfseries\small,
      coltitle=white,
      attach boxed title to top center={yshift=-2mm},
      boxed title style={colback=bleuFonce, arc=2mm}
    ]
    \begin{itemize}[leftmargin=*, itemsep=3pt]
      \item Modèle Llama 3.1 local (Ollama)
      \item Génération de réponses contextualisées
      \item Sources citées et vérifiables
      \item Historique des conversations
    \end{itemize}
  \end{tcolorbox}
\end{minipage}

\vspace{0.4cm}

\begin{tcolorbox}[
    enhanced,
    colback=bleuTresClair,
    colframe=bleuClair,
    boxrule=1pt,
    arc=3mm,
    left=10pt, right=10pt, top=8pt, bottom=8pt
  ]
  \begin{center}
    {\bfseries\color{bleuFonce}\faLaptopCode\hspace{0.3cm}Objectifs Techniques Transversaux}
  \end{center}

  \vspace{0.2cm}

  \begin{minipage}[t]{0.48\textwidth}
    \begin{itemize}[leftmargin=*, itemsep=2pt, label=\textcolor{bleuPrincipal}{\faCheck}]
      \item Architecture microservices (7 services)
      \item API Gateway centralisée (Python/FastAPI)
      \item Services Java (Spring Boot) et Python
      \item Communication asynchrone (RabbitMQ)
    \end{itemize}
  \end{minipage}
  \hfill
  \begin{minipage}[t]{0.48\textwidth}
    \begin{itemize}[leftmargin=*, itemsep=2pt, label=\textcolor{bleuPrincipal}{\faCheck}]
      \item Conteneurisation Docker Compose
      \item Pipeline CI/CD GitHub Actions
      \item Interface React moderne
      \item Système d'audit complet
    \end{itemize}
  \end{minipage}
\end{tcolorbox}

%==============================================================================
% SECTION 4 : PÉRIMÈTRE DU PROJET
%==============================================================================

\newpage

\section*{\textcolor{bleuFonce}{\faProjectDiagram\hspace{0.3cm}Périmètre du Projet}}
\addcontentsline{toc}{section}{Périmètre du Projet}

La délimitation précise du périmètre d'un projet est essentielle pour garantir sa réussite dans les délais et avec les ressources impartis. Cette section définit clairement ce que le projet DocQA-MS couvre, ainsi que les éléments volontairement exclus de son champ d'application.

\vspace{0.4cm}

\begin{minipage}[t]{0.48\textwidth}
  \begin{tcolorbox}[
      enhanced,
      colback=white,
      colframe=bleuPrincipal,
      boxrule=2pt,
      arc=4mm,
      left=10pt, right=10pt, top=10pt, bottom=10pt,
      title={\textcolor{white}{\large\faCheckCircle\hspace{0.2cm}Inclus dans le Projet}},
      fonttitle=\bfseries,
      coltitle=white,
      attach boxed title to top center={yshift=-3mm},
      boxed title style={colback=bleuPrincipal, arc=3mm}
    ]

    \vspace{0.2cm}

    \textbf{\textcolor{bleuFonce}{Microservices Développés}}
    \begin{itemize}[leftmargin=*, itemsep=2pt, label=\textcolor{bleuPrincipal}{\faCheck}]
      \item API Gateway (FastAPI)
      \item Doc Ingestor (FastAPI)
      \item DeID Service (Spring Boot)
      \item Indexeur Sémantique (Spring Boot)
      \item LLM Q\&A Module (FastAPI)
      \item Synthèse Comparative (Spring Boot)
      \item Audit Logger (Spring Boot)
    \end{itemize}

    \vspace{0.3cm}

    \textbf{\textcolor{bleuFonce}{Fonctionnalités}}
    \begin{itemize}[leftmargin=*, itemsep=2pt, label=\textcolor{bleuPrincipal}{\faCheck}]
      \item Ingestion multi-formats
      \item Anonymisation automatique NER
      \item Recherche sémantique vectorielle
      \item Q\&A en langage naturel
      \item Synthèse comparative
      \item Audit et traçabilité
      \item Interface utilisateur React
    \end{itemize}

  \end{tcolorbox}
\end{minipage}
\hfill
\begin{minipage}[t]{0.48\textwidth}
  \begin{tcolorbox}[
      enhanced,
      colback=white,
      colframe=rougeAlert,
      boxrule=2pt,
      arc=4mm,
      left=10pt, right=10pt, top=10pt, bottom=10pt,
      title={\textcolor{white}{\large\faTimesCircle\hspace{0.2cm}Exclus du Projet}},
      fonttitle=\bfseries,
      coltitle=white,
      attach boxed title to top center={yshift=-3mm},
      boxed title style={colback=rougeAlert, arc=3mm}
    ]

    \vspace{0.2cm}

    \textbf{\textcolor{rougeAlert}{Limitations Fonctionnelles}}
    \begin{itemize}[leftmargin=*, itemsep=2pt, label=\textcolor{rougeAlert}{\faTimes}]
      \item Intégration avec DPI hospitaliers
      \item Support d'images médicales (DICOM)
      \item Reconnaissance vocale
      \item Multi-langues (français uniquement)
      \item Génération de rapports PDF
      \item Alertes en temps réel
      \item Gestion des utilisateurs avancée
    \end{itemize}

    \vspace{0.3cm}

    \textbf{\textcolor{rougeAlert}{Limitations Techniques}}
    \begin{itemize}[leftmargin=*, itemsep=2pt, label=\textcolor{rougeAlert}{\faTimes}]
      \item Déploiement cloud production
      \item Haute disponibilité (HA)
      \item Kubernetes orchestration
      \item Monitoring avancé (Prometheus)
      \item Sauvegarde automatisée
      \item Authentification SSO
      \item Tests de charge massifs
    \end{itemize}

  \end{tcolorbox}
\end{minipage}

\vspace{0.5cm}

\begin{tcolorbox}[
    enhanced,
    colback=bleuTresClair,
    colframe=bleuMarine,
    boxrule=1pt,
    arc=3mm,
    left=10pt, right=10pt, top=8pt, bottom=8pt
  ]
  \textbf{\textcolor{bleuFonce}{\faExclamationTriangle\hspace{0.2cm}Avertissement Important}}

  \vspace{0.2cm}

  DocQA-MS est conçu comme un \textbf{prototype fonctionnel} à vocation académique et ne constitue pas un dispositif médical certifié. Les réponses générées par le système ne doivent en aucun cas se substituer à l'expertise d'un professionnel de santé qualifié. L'anonymisation proposée, bien qu'inspirée des bonnes pratiques du domaine, n'a pas fait l'objet d'une certification formelle de conformité RGPD. En environnement de production réel, une validation juridique et technique approfondie serait nécessaire avant tout déploiement.
\end{tcolorbox}

%==============================================================================
% SECTION 5 : ORGANISATION DU RAPPORT
%==============================================================================

\section*{\textcolor{bleuFonce}{\faBookOpen\hspace{0.3cm}Organisation du Rapport}}
\addcontentsline{toc}{section}{Organisation du Rapport}

Le présent mémoire est structuré en quatre chapitres, chacun abordant un aspect spécifique du projet DocQA-MS. Cette organisation permet une progression logique depuis la présentation du contexte jusqu'au bilan final, en passant par les phases d'analyse, de conception et de réalisation.

\vspace{0.4cm}

\begin{tikzpicture}[
    node distance=0.4cm,
    chapter/.style={
      rectangle,
      rounded corners=3pt,
      minimum width=14.5cm,
      minimum height=1.8cm,
      text width=14cm,
      align=left,
      fill=white,
      draw=bleuPrincipal,
      line width=1pt
    },
    num/.style={
      circle,
      fill=bleuPrincipal,
      text=white,
      font=\bfseries\large,
      minimum size=0.9cm
    }
  ]

  % Chapitre 1
  \node[chapter] (ch1) {
    \hspace{1.2cm}\textbf{\textcolor{bleuFonce}{\large Chapitre 1 : Cadre Général du Projet}}\\[0.1cm]
    \hspace{1.2cm}{\small\color{grisTexte}Présentation de l'organisme d'accueil, étude des solutions existantes, description de l'architecture microservices proposée, méthodologie Scrum et technologies utilisées.}
  };
  \node[num, left=0.2cm of ch1.west, anchor=east] {1};

  % Chapitre 2
  \node[chapter, below=of ch1] (ch2) {
    \hspace{1.2cm}\textbf{\textcolor{bleuFonce}{\large Chapitre 2 : Analyse et Spécification des Besoins}}\\[0.1cm]
    \hspace{1.2cm}{\small\color{grisTexte}Identification des acteurs, besoins fonctionnels et non fonctionnels par microservice, diagrammes de cas d'utilisation et backlog produit Scrum.}
  };
  \node[num, left=0.2cm of ch2.west, anchor=east] {2};

  % Chapitre 3
  \node[chapter, below=of ch2] (ch3) {
    \hspace{1.2cm}\textbf{\textcolor{bleuFonce}{\large Chapitre 3 : Conception}}\\[0.1cm]
    \hspace{1.2cm}{\small\color{grisTexte}Architecture microservices détaillée, diagrammes de classes par service, diagrammes de séquence, modèle de données et conception des interfaces utilisateur.}
  };
  \node[num, left=0.2cm of ch3.west, anchor=east] {3};

  % Chapitre 4
  \node[chapter, below=of ch3] (ch4) {
    \hspace{1.2cm}\textbf{\textcolor{bleuFonce}{\large Chapitre 4 : Réalisation}}\\[0.1cm]
    \hspace{1.2cm}{\small\color{grisTexte}Environnement de développement, implémentation des microservices Python et Java, interface React, tests, CI/CD GitHub Actions et résultats JMeter.}
  };
  \node[num, left=0.2cm of ch4.west, anchor=east] {4};

  % Flèches de connexion
  \draw[->, bleuClair, line width=2pt] (ch1.south) -- (ch2.north);
  \draw[->, bleuClair, line width=2pt] (ch2.south) -- (ch3.north);
  \draw[->, bleuClair, line width=2pt] (ch3.south) -- (ch4.north);

\end{tikzpicture}

\vspace{0.5cm}

\begin{center}
  \begin{tikzpicture}
    \node[fill=bleuTresClair, draw=bleuPrincipal, line width=1pt,
    rounded corners=5pt, inner sep=12pt, text width=14cm, align=center] {
      {\color{grisTexte}Le rapport se conclut par une \textbf{conclusion générale} synthétisant les apports du projet, les difficultés rencontrées, les compétences acquises et les perspectives d'évolution future du système DocQA-MS.}
    };
  \end{tikzpicture}
\end{center}

\newpage
\pagenumbering{arabic}
\setcounter{page}{1}

%==============================================================================
% CHAPITRE 1 : CADRE GÉNÉRAL DU PROJET
%==============================================================================

\chapter{Cadre Général du Projet}

\begin{tikzpicture}[remember picture, overlay]
    % Décoration de page
    \fill[bleuClair, opacity=0.1] 
        ([xshift=-3cm, yshift=-2cm]current page.north east) circle (4cm);
    \fill[bleuPrincipal, opacity=0.05] 
        ([xshift=2cm, yshift=3cm]current page.south west) circle (5cm);
\end{tikzpicture}

\vspace{-0.5cm}

\begin{tcolorbox}[
    enhanced,
    colback=bleuTresClair,
    colframe=bleuPrincipal,
    boxrule=0pt,
    borderline west={4pt}{0pt}{bleuPrincipal},
    arc=0mm,
    left=12pt, right=12pt, top=10pt, bottom=10pt
]
{\itshape\color{grisTexte}
Ce premier chapitre pose les fondations du projet DocQA-MS en présentant le cadre dans lequel il s'inscrit. Nous commencerons par une présentation de l'organisme d'accueil, avant d'analyser les solutions existantes sur le marché des systèmes de question-réponse documentaires. Nous décrirons ensuite l'architecture microservices proposée, la méthodologie de travail adoptée, ainsi que les outils et technologies utilisés pour le développement.
}
\end{tcolorbox}

\vspace{0.5cm}

% --- SECTION 1.1 ---
\section{Présentation de l'Organisme d'Accueil}

Cette section présente l'environnement institutionnel dans lequel s'est déroulé notre projet de fin d'études.

\subsection{Historique}

\begin{tcolorbox}[
    enhanced,
    colback=white,
    colframe=bleuPrincipal,
    boxrule=1.5pt,
    arc=3mm,
    left=10pt, right=10pt, top=10pt, bottom=10pt,
    shadow={2mm}{-2mm}{0mm}{black!15},
    title={\textcolor{white}{\faUniversity\hspace{0.2cm}EMSI Marrakech}},
    fonttitle=\bfseries\large,
    coltitle=white,
    attach boxed title to top left={yshift=-3mm, xshift=5mm},
    boxed title style={colback=bleuPrincipal, arc=2mm}
]

\textbf{EMSI} (École Marocaine des Sciences de l'Ingénieur) est un établissement d'enseignement supérieur fondé en \textbf{1986}. Située à \textbf{Marrakech, Maroc}, cette institution a su s'imposer comme un acteur majeur de la formation en ingénierie au Maroc et en Afrique.

\vspace{0.3cm}

\begin{center}
\begin{tikzpicture}[scale=0.85, transform shape]
    % Timeline horizontale
    \draw[bleuPrincipal, line width=2pt] (0,0) -- (13,0);
    
    % Points de la timeline
    \foreach \x/\year/\event in {
        0/1986/Fondation,
        4/1996/Dpt. Info,
        8.5/2010/Accréditation,
        13/2025/Leader
    } {
        \fill[bleuPrincipal] (\x,0) circle (0.15cm);
        \node[above, font=\small\bfseries, text=bleuFonce] at (\x,0.3) {\year};
        \node[below, font=\scriptsize, text=grisTexte, text width=2.5cm, align=center] at (\x,-0.3) {\event};
    }
\end{tikzpicture}
\end{center}

\vspace{0.3cm}

Le département d'informatique forme aujourd'hui des ingénieurs spécialisés dans les domaines du génie logiciel, de l'intelligence artificielle, du cloud computing et des systèmes distribués. L'accent mis sur les projets pratiques et l'innovation technologique prépare les étudiants aux défis de l'industrie numérique moderne.

\end{tcolorbox}

\subsection{Organigramme}

L'organisation hiérarchique de l'établissement reflète une structure claire orientée vers l'excellence académique.

\vspace{0.4cm}

\begin{figure}[H]
\centering
\begin{tikzpicture}[
    node distance=0.8cm and 0.5cm,
    box/.style={
        rectangle, rounded corners=3pt, draw=bleuPrincipal, fill=white,
        line width=1pt, minimum width=3cm, minimum height=0.9cm,
        text=grisTexte, font=\small, align=center
    },
    boxhead/.style={
        rectangle, rounded corners=3pt, draw=bleuFonce, fill=bleuPrincipal,
        line width=1.5pt, minimum width=4cm, minimum height=1cm,
        text=white, font=\small\bfseries, align=center
    },
    arrow/.style={->, >=stealth, bleuPrincipal, line width=1pt}
]

% Niveau 1
\node[boxhead] (dir) {Direction Générale};

% Niveau 2
\node[box, below left=1cm and 1cm of dir] (vadm) {Vice-Direction\\Admin.};
\node[box, below right=1cm and 1cm of dir] (vped) {Vice-Direction\\Pédagogique};

% Niveau 3
\node[box, below=1cm of vped] (dinfo) {\faLaptopCode\ Dpt. Informatique};

% Flèches
\draw[arrow] (dir) -- (vadm);
\draw[arrow] (dir) -- (vped);
\draw[arrow] (vped) -- (dinfo);

\end{tikzpicture}
\caption{Organigramme simplifié de l'EMSI}
\end{figure}

% --- SECTION 1.2 ---
\section{Étude de l'Existant}

Avant de concevoir notre solution, nous avons analysé les principales solutions existantes dans le domaine des systèmes de question-réponse sur documents et de l'IA appliquée au secteur médical.

\subsection{Analyse Comparative}

\begin{table}[H]
\centering
\caption{Comparatif des solutions existantes}
\renewcommand{\arraystretch}{1.3}
\begin{tabular}{|p{3cm}|c|c|c|c|}
\hline
\rowcolor{bleuPrincipal} \textcolor{white}{\textbf{Fonctionnalité}} & \textcolor{white}{\textbf{ChatGPT}} & \textcolor{white}{\textbf{Azure AI}} & \textcolor{white}{\textbf{AWS Kendra}} & \textcolor{white}{\textbf{DocQA-MS}} \\
\hline
Q\&A sur documents & \textcolor{bleuPrincipal}{\faCheck} & \textcolor{bleuPrincipal}{\faCheck} & \textcolor{bleuPrincipal}{\faCheck} & \textcolor{bleuPrincipal}{\faCheck\faCheck} \\
\hline
Exécution locale LLM & \textcolor{rougeAlert}{\faTimes} & \textcolor{rougeAlert}{\faTimes} & \textcolor{rougeAlert}{\faTimes} & \textcolor{bleuPrincipal}{\faCheck} (Ollama) \\
\hline
Anonymisation intégrée & \textcolor{rougeAlert}{\faTimes} & \textcolor{orangeWarning}{\faCircle} & \textcolor{rougeAlert}{\faTimes} & \textcolor{bleuPrincipal}{\faCheck} (NER) \\
\hline
Architecture microservices & \textcolor{rougeAlert}{\faTimes} & \textcolor{bleuPrincipal}{\faCheck} & \textcolor{bleuPrincipal}{\faCheck} & \textcolor{bleuPrincipal}{\faCheck} \\
\hline
Open Source & \textcolor{rougeAlert}{\faTimes} & \textcolor{rougeAlert}{\faTimes} & \textcolor{rougeAlert}{\faTimes} & \textcolor{bleuPrincipal}{\faCheck} \\
\hline
Audit complet & \textcolor{orangeWarning}{\faCircle} & \textcolor{bleuPrincipal}{\faCheck} & \textcolor{bleuPrincipal}{\faCheck} & \textcolor{bleuPrincipal}{\faCheck} \\
\hline
Données on-premise & \textcolor{rougeAlert}{\faTimes} & \textcolor{orangeWarning}{\faCircle} & \textcolor{rougeAlert}{\faTimes} & \textcolor{bleuPrincipal}{\faCheck} \\
\hline
\end{tabular}
\end{table}

\begin{tcolorbox}[
    enhanced,
    colback=bleuTresClair,
    colframe=bleuClair,
    boxrule=1pt,
    arc=2mm,
    left=8pt, right=8pt, top=6pt, bottom=6pt
]
\textbf{\textcolor{bleuFonce}{Analyse des lacunes identifiées :}}

\begin{itemize}[leftmargin=*, itemsep=3pt, label=\textcolor{bleuPrincipal}{\faAngleRight}]
    \item \textbf{ChatGPT/Claude} : Puissants mais nécessitent l'envoi de données vers des serveurs externes, incompatible avec les exigences de confidentialité médicale.
    \item \textbf{Azure AI / AWS Kendra} : Solutions cloud coûteuses, avec des problématiques de souveraineté des données.
    \item \textbf{Solutions open source} : Fragmentées, nécessitant une intégration complexe de multiples composants.
\end{itemize}

\textbf{DocQA-MS} se positionne comme une alternative \textbf{open source}, \textbf{on-premise}, avec \textbf{anonymisation intégrée} et exécution \textbf{locale du LLM}.
\end{tcolorbox}

% --- SECTION 1.3 ---
\section{Solution Proposée : DocQA-MS}

\begin{tcolorbox}[
    enhanced,
    colback=white,
    colframe=bleuPrincipal,
    boxrule=2pt,
    arc=5mm,
    left=15pt, right=15pt, top=15pt, bottom=15pt,
    shadow={3mm}{-3mm}{0mm}{black!20}
]
\begin{center}
{\fontsize{28}{34}\selectfont\textcolor{bleuFonce}{\textbf{Doc}}\textcolor{bleuPrincipal}{\textbf{QA-MS}}}

\vspace{0.2cm}
{\large\itshape\textcolor{grisTexte}{Système de Question-Réponse sur Documents Médicaux}}\\[0.1cm]
{\normalsize\textcolor{bleuMarine}{Architecture Microservices | LLM Local | Anonymisation RGPD}}
\end{center}

\vspace{0.3cm}
DocQA-MS est une plateforme complète permettant aux professionnels de santé d'interroger des corpus de documents médicaux en langage naturel, tout en garantissant la confidentialité des données patients grâce à une anonymisation automatique et une exécution locale des modèles d'IA.
\end{tcolorbox}

\subsection{Architecture Microservices}

\begin{figure}[H]
\centering
\begin{tikzpicture}[
    scale=0.7,
    transform shape,
    service/.style={
        rectangle, rounded corners=5pt, draw=#1, fill=#1!10,
        line width=1.5pt, minimum width=2.8cm, minimum height=1.2cm,
        font=\small\bfseries, align=center
    },
    infra/.style={
        rectangle, rounded corners=3pt, draw=gray!60, fill=gray!10,
        line width=1pt, minimum width=2.5cm, minimum height=1cm,
        font=\small, align=center
    },
    myarrow/.style={->, >=stealth, line width=1pt, #1}
]

% API Gateway (centre haut)
\node[service=bleuPrincipal] (gateway) at (0,4) {\faServer\\API Gateway};

% Services Python (gauche)
\node[service=bleuTurquoise] (ingestor) at (-5,1) {\faFileAlt\\Doc Ingestor};
\node[service=bleuTurquoise] (llm) at (-5,-2) {\faRobot\\LLM Q\&A};

% Services Java (droite)
\node[service=bleuMarine] (deid) at (5,2) {\faUserSecret\\DeID Service};
\node[service=bleuMarine] (indexer) at (5,0) {\faSearch\\Indexeur};
\node[service=bleuMarine] (synthese) at (5,-2) {\faLayerGroup\\Synthèse};
\node[service=bleuMarine] (audit) at (0,-2) {\faClipboardList\\Audit Logger};

% Infrastructure
\node[infra] (postgres) at (-3,-4.5) {\faDatabase~PostgreSQL};
\node[infra] (rabbitmq) at (0,-4.5) {\faEnvelope~RabbitMQ};
\node[infra] (ollama) at (3,-4.5) {\faBrain~Ollama};

% Frontend
\node[service=vertSucces] (react) at (0,6.5) {\faReact\\React Frontend};

% Flèches
\draw[myarrow=bleuPrincipal] (react) -- (gateway);
\draw[myarrow=bleuPrincipal] (gateway) -- (ingestor);
\draw[myarrow=bleuPrincipal] (gateway) -- (deid);
\draw[myarrow=bleuPrincipal] (gateway) -- (indexer);
\draw[myarrow=bleuPrincipal] (gateway) -- (llm);
\draw[myarrow=bleuPrincipal] (gateway) -- (synthese);
\draw[myarrow=bleuPrincipal] (gateway) -- (audit);

% Connexions infrastructure
\draw[myarrow=gray!60, dashed] (audit) -- (postgres);
\draw[myarrow=gray!60, dashed] (ingestor) -- (rabbitmq);
\draw[myarrow=gray!60, dashed] (llm) -- (ollama);

\end{tikzpicture}
\caption{Architecture microservices de DocQA-MS}
\label{fig:archi-microservices}
\end{figure}

\subsection{Description des Microservices}

\begin{table}[H]
\centering
\caption{Description des microservices DocQA-MS}
\renewcommand{\arraystretch}{1.4}
\begin{tabular}{|>{\columncolor{bleuTresClair}}p{3.5cm}|p{2cm}|p{7cm}|}
\hline
\rowcolor{bleuPrincipal}
\textcolor{white}{\textbf{Service}} & \textcolor{white}{\textbf{Techno}} & \textcolor{white}{\textbf{Responsabilité}} \\
\hline
\textbf{API Gateway} & FastAPI & Point d'entrée unique, routage, authentification \\
\hline
\textbf{Doc Ingestor} & FastAPI & Ingestion de documents, extraction de texte, chunking \\
\hline
\textbf{DeID Service} & Spring Boot & Anonymisation NER, masquage des données personnelles \\
\hline
\textbf{Indexeur Sémantique} & Spring Boot & Génération d'embeddings, indexation vectorielle \\
\hline
\textbf{LLM Q\&A Module} & FastAPI & Orchestration RAG, appel Ollama, génération réponses \\
\hline
\textbf{Synthèse Comparative} & Spring Boot & Analyse multi-documents, génération de synthèses \\
\hline
\textbf{Audit Logger} & Spring Boot & Journalisation, traçabilité, statistiques d'utilisation \\
\hline
\end{tabular}
\end{table}

% --- SECTION 1.4 ---
\section{Méthodologie de Travail : Scrum}

Pour mener à bien ce projet, nous avons adopté la méthodologie agile \textbf{Scrum}, particulièrement adaptée au développement itératif de systèmes complexes comme une architecture microservices.

\vspace{0.4cm}

\begin{tcolorbox}[
    enhanced,
    colback=white,
    colframe=bleuPrincipal,
    boxrule=1.5pt,
    arc=3mm,
    left=10pt, right=10pt, top=10pt, bottom=10pt,
    title={\textcolor{white}{\faSync\hspace{0.2cm}Cycle Scrum Adopté}},
    fonttitle=\bfseries\large,
    coltitle=white,
    attach boxed title to top left={yshift=-3mm, xshift=5mm},
    boxed title style={colback=bleuPrincipal, arc=2mm}
]

\begin{center}
\begin{tikzpicture}[scale=0.8, transform shape]
    % Cercle central
    \fill[bleuTresClair] (0,0) circle (2.5cm);
    \node[font=\bfseries, text=bleuFonce] at (0,0) {Sprint\\1 semaine};
    
    % Éléments autour
    \node[fill=bleuPrincipal, text=white, rounded corners=3pt, 
          inner sep=6pt, font=\small] at (-4,2) {Product Backlog};
    \node[fill=bleuTurquoise, text=white, rounded corners=3pt, 
          inner sep=6pt, font=\small] at (4,2) {Sprint Backlog};
    \node[fill=bleuMarine, text=white, rounded corners=3pt, 
          inner sep=6pt, font=\small] at (-4,-2) {Daily Scrum};
    \node[fill=bleuFonce, text=white, rounded corners=3pt, 
          inner sep=6pt, font=\small] at (4,-2) {Incrément};
    
    % Flèches circulaires
    \draw[->, bleuPrincipal, line width=1.5pt] (-3.5,1.5) arc (135:45:4);
    \draw[->, bleuPrincipal, line width=1.5pt] (3.5,-1.5) arc (-45:-135:4);
\end{tikzpicture}
\end{center}

\vspace{0.3cm}

\renewcommand{\arraystretch}{1.3}
\begin{center}
\begin{tabular}{|l|l|}
\hline
\rowcolor{bleuPrincipal}
\textcolor{white}{\textbf{Paramètre}} & \textcolor{white}{\textbf{Valeur}} \\
\hline
Durée des sprints & 1 semaine \\
\hline
Nombre de sprints & 5 sprints \\
\hline
Réunions Daily & 15 minutes/jour \\
\hline
Revue de sprint & Fin de chaque sprint \\
\hline
Outil de gestion & GitHub Projects \\
\hline
\end{tabular}
\end{center}

\end{tcolorbox}

% --- SECTION 1.5 ---
\section{Stack Technologique}

\begin{tcolorbox}[
    enhanced,
    colback=white,
    colframe=bleuPrincipal,
    boxrule=1.5pt,
    arc=3mm,
    left=10pt, right=10pt, top=10pt, bottom=10pt,
    title={\textcolor{white}{\faLayerGroup\hspace{0.2cm}Technologies Utilisées}},
    fonttitle=\bfseries\large,
    coltitle=white,
    attach boxed title to top left={yshift=-3mm, xshift=5mm},
    boxed title style={colback=bleuPrincipal, arc=2mm}
]

\begin{minipage}[t]{0.48\textwidth}
    \textbf{\textcolor{bleuFonce}{Backend Python}}
    \begin{tabular}{|l|l|}
    \hline
    \rowcolor{bleuTresClair}
    \textbf{Outil} & \textbf{Version} \\
    \hline
    Python & 3.11 \\
    \hline
    FastAPI & 0.104+ \\
    \hline
    LangChain & 0.1+ \\
    \hline
    Ollama (Llama 3.1) & Latest \\
    \hline
    \end{tabular}
    
    \vspace{0.3cm}
    
    \textbf{\textcolor{bleuFonce}{Backend Java}}
    \begin{tabular}{|l|l|}
    \hline
    \rowcolor{bleuTresClair}
    \textbf{Outil} & \textbf{Version} \\
    \hline
    Java JDK & 17 LTS \\
    \hline
    Spring Boot & 3.2+ \\
    \hline
    Maven & 3.9 \\
    \hline
    \end{tabular}
\end{minipage}
\hfill
\begin{minipage}[t]{0.48\textwidth}
    \textbf{\textcolor{bleuFonce}{Frontend}}
    \begin{tabular}{|l|l|}
    \hline
    \rowcolor{bleuTresClair}
    \textbf{Outil} & \textbf{Version} \\
    \hline
    Node.js & 18 LTS \\
    \hline
    React & 18 \\
    \hline
    Axios & Latest \\
    \hline
    \end{tabular}
    
    \vspace{0.3cm}
    
    \textbf{\textcolor{bleuFonce}{Infrastructure}}
    \begin{tabular}{|l|l|}
    \hline
    \rowcolor{bleuTresClair}
    \textbf{Outil} & \textbf{Version} \\
    \hline
    Docker & 24+ \\
    \hline
    Docker Compose & 2+ \\
    \hline
    PostgreSQL & 16 \\
    \hline
    RabbitMQ & 3.12 \\
    \hline
    \end{tabular}
    
    \vspace{0.3cm}
    
    \textbf{\textcolor{bleuFonce}{CI/CD}}
    \begin{tabular}{|l|l|}
    \hline
    \rowcolor{bleuTresClair}
    \textbf{Outil} & \textbf{Version} \\
    \hline
    GitHub Actions & Latest \\
    \hline
    JMeter & 5.6 \\
    \hline
    \end{tabular}
\end{minipage}

\end{tcolorbox}

\vspace{0.5cm}

%--- Conclusion du chapitre ---
\begin{tcolorbox}[
    enhanced,
    colback=white,
    colframe=bleuPrincipal,
    boxrule=0pt,
    borderline south={3pt}{0pt}{bleuPrincipal},
    arc=0mm,
    left=10pt, right=10pt, top=10pt, bottom=10pt
]
\textbf{\textcolor{bleuFonce}{Conclusion du Chapitre}}

\vspace{0.2cm}

Ce chapitre a présenté le cadre général du projet DocQA-MS. Nous avons introduit l'organisme d'accueil, analysé les solutions existantes et leurs limitations, puis décrit l'architecture microservices proposée. La méthodologie Scrum adoptée et le stack technologique choisi constituent les fondations solides sur lesquelles repose le développement du système. Le chapitre suivant détaillera l'analyse et la spécification des besoins fonctionnels et non fonctionnels.
\end{tcolorbox}

%==============================================================================
%                           CHAPITRE 2
%                   ANALYSE ET SPÉCIFICATION DES BESOINS
%==============================================================================

\chapter{Analyse et Spécification des Besoins}

\begin{tikzpicture}[remember picture, overlay]
  % Décoration de page
  \fill[bleuClair, opacity=0.1]
  ([xshift=3cm, yshift=-2cm]current page.north west) circle (4cm);
  \fill[bleuPrincipal, opacity=0.05]
  ([xshift=-2cm, yshift=3cm]current page.south east) circle (5cm);
\end{tikzpicture}

\vspace{-0.5cm}

\begin{tcolorbox}[
    enhanced,
    colback=bleuTresClair,
    colframe=bleuPrincipal,
    boxrule=0pt,
    borderline west={4pt}{0pt}{bleuPrincipal},
    arc=0mm,
    left=12pt, right=12pt, top=10pt, bottom=10pt
  ]
  {\itshape\color{grisTexte}
    Ce chapitre présente l'analyse détaillée des besoins du projet DocQA-MS. Nous commencerons par identifier les différents acteurs du système, puis nous spécifierons les besoins fonctionnels et non fonctionnels pour chaque microservice. Enfin, nous modéliserons ces besoins à travers des diagrammes de cas d'utilisation UML et présenterons le backlog produit selon la méthodologie Scrum.
  }
\end{tcolorbox}

\vspace{0.5cm}

%==============================================================================
% SECTION 2.1 : IDENTIFICATION DES ACTEURS
%==============================================================================

\section{Identification des Acteurs}

L'identification des acteurs constitue une étape fondamentale dans l'analyse des besoins. Un acteur représente une entité externe qui interagit avec le système. Dans le cadre de DocQA-MS, nous distinguons un acteur principal et plusieurs acteurs secondaires.

%------------------------------------------------------------------------------
% 2.1.1 Acteur Principal
%------------------------------------------------------------------------------

\subsection{Acteur Principal : Clinicien}

Le clinicien (médecin, infirmier, personnel de santé) représente l'utilisateur principal du système DocQA-MS. Cet acteur est au cœur de notre système et bénéficie de l'ensemble des fonctionnalités proposées.

\vspace{0.4cm}

\begin{tcolorbox}[
    enhanced,
    colback=white,
    colframe=bleuPrincipal,
    boxrule=2pt,
    arc=4mm,
    left=10pt, right=10pt, top=10pt, bottom=10pt,
    shadow={2mm}{-2mm}{0mm}{black!15}
  ]

  \begin{center}
    \begin{tikzpicture}
      % Icône utilisateur
      \fill[bleuPrincipal] (0,0) circle (1cm);
      \node[text=white, font=\Huge] at (0,0) {\faUserMd};
    \end{tikzpicture}

    \vspace{0.2cm}

    {\Large\bfseries\textcolor{bleuFonce}{Clinicien (Personnel de Santé)}}
  \end{center}

  \vspace{0.3cm}

  \begin{minipage}[t]{0.48\textwidth}
    \begin{tcolorbox}[
        enhanced,
        colback=bleuTresClair,
        colframe=bleuClair,
        boxrule=1pt,
        arc=2mm,
        left=6pt, right=6pt, top=6pt, bottom=6pt,
        title={\textcolor{bleuFonce}{\faIdCard~Profil}},
        fonttitle=\bfseries\small,
        coltitle=bleuFonce,
        attach boxed title to top left={yshift=-2mm, xshift=3mm},
        boxed title style={colback=bleuTresClair, arc=1mm, boxrule=0pt}
      ]
      \begin{itemize}[leftmargin=*, itemsep=2pt, label=\textcolor{bleuPrincipal}{\faAngleRight}]
        \item Médecin, infirmier, interne
        \item Accès aux dossiers patients
        \item Besoin d'informations rapides
        \item Contraintes de temps fortes
      \end{itemize}
    \end{tcolorbox}
  \end{minipage}
  \hfill
  \begin{minipage}[t]{0.48\textwidth}
    \begin{tcolorbox}[
        enhanced,
        colback=bleuTresClair,
        colframe=bleuClair,
        boxrule=1pt,
        arc=2mm,
        left=6pt, right=6pt, top=6pt, bottom=6pt,
        title={\textcolor{bleuFonce}{\faHeart~Besoins}},
        fonttitle=\bfseries\small,
        coltitle=bleuFonce,
        attach boxed title to top left={yshift=-2mm, xshift=3mm},
        boxed title style={colback=bleuTresClair, arc=1mm, boxrule=0pt}
      ]
      \begin{itemize}[leftmargin=*, itemsep=2pt, label=\textcolor{bleuPrincipal}{\faAngleRight}]
        \item Requêtes en langage naturel
        \item Réponses sourcées et vérifiables
        \item Synthèses multi-documents
        \item Confidentialité garantie
      \end{itemize}
    \end{tcolorbox}
  \end{minipage}

  \vspace{0.3cm}

  \begin{tcolorbox}[
      enhanced,
      colback=bleuPrincipal!10,
      colframe=bleuPrincipal!50,
      boxrule=1pt,
      arc=2mm,
      left=6pt, right=6pt, top=6pt, bottom=6pt,
      title={\textcolor{bleuFonce}{\faTasks~Actions Principales}},
      fonttitle=\bfseries\small,
      coltitle=bleuFonce,
      attach boxed title to top left={yshift=-2mm, xshift=3mm},
      boxed title style={colback=bleuPrincipal!10, arc=1mm, boxrule=0pt}
    ]
    \begin{center}
      \begin{tikzpicture}[scale=0.9]
        \foreach \x/\icon/\label in {
          0/\faUpload/Ingérer docs,
          2.8/\faUserSecret/Anonymiser,
          5.6/\faSearch/Rechercher,
          8.4/\faRobot/Q\&A IA,
          11.2/\faLayerGroup/Synthétiser
        } {
          \node[fill=bleuPrincipal!20, rounded corners=3pt, minimum width=2.3cm,
          minimum height=1.2cm, align=center, font=\scriptsize] at (\x,0)
          {\textcolor{bleuFonce}{\large\icon}\\\label};
        }
      \end{tikzpicture}
    \end{center}
  \end{tcolorbox}

\end{tcolorbox}

%------------------------------------------------------------------------------
% 2.1.2 Acteurs Secondaires
%------------------------------------------------------------------------------

\subsection{Acteurs Secondaires}

En complément du clinicien, plusieurs acteurs secondaires interagissent avec le système DocQA-MS pour assurer son bon fonctionnement.

\vspace{0.4cm}

\begin{minipage}[t]{0.32\textwidth}
  \begin{tcolorbox}[
      enhanced,
      colback=white,
      colframe=orangeWarning,
      boxrule=1.5pt,
      arc=3mm,
      left=6pt, right=6pt, top=8pt, bottom=8pt,
      height=7.5cm
    ]
    \begin{center}
      \begin{tikzpicture}
        \fill[orangeWarning] (0,0) circle (0.7cm);
        \node[text=white, font=\large] at (0,0) {\faCogs};
      \end{tikzpicture}

      \vspace{0.2cm}

      {\bfseries\textcolor{orangeWarning}{Système}}
    \end{center}

    \vspace{0.2cm}

    {\small
      \textbf{Rôle :} Gestion automatisée des tâches internes.

      \vspace{0.2cm}

      \textbf{Responsabilités :}
      \begin{itemize}[leftmargin=*, itemsep=1pt, label=\textcolor{orangeWarning}{\faAngleRight}]
        \item Routage des requêtes
        \item Messages RabbitMQ
        \item Journalisation audit
        \item Health checks
      \end{itemize}
    }
  \end{tcolorbox}
\end{minipage}
\hfill
\begin{minipage}[t]{0.32\textwidth}
  \begin{tcolorbox}[
      enhanced,
      colback=white,
      colframe=bleuMarine,
      boxrule=1.5pt,
      arc=3mm,
      left=6pt, right=6pt, top=8pt, bottom=8pt,
      height=7.5cm
    ]
    \begin{center}
      \begin{tikzpicture}
        \fill[bleuMarine] (0,0) circle (0.7cm);
        \node[text=white, font=\large] at (0,0) {\faBrain};
      \end{tikzpicture}

      \vspace{0.2cm}

      {\bfseries\textcolor{bleuMarine}{Ollama (LLM)}}
    \end{center}

    \vspace{0.2cm}

    {\small
      \textbf{Rôle :} Exécution locale du modèle Llama 3.1.

      \vspace{0.2cm}

      \textbf{Responsabilités :}
      \begin{itemize}[leftmargin=*, itemsep=1pt, label=\textcolor{bleuMarine}{\faAngleRight}]
        \item Génération de réponses
        \item Traitement RAG
        \item Embeddings textuels
        \item Inférence locale
      \end{itemize}
    }
  \end{tcolorbox}
\end{minipage}
\hfill
\begin{minipage}[t]{0.32\textwidth}
  \begin{tcolorbox}[
      enhanced,
      colback=white,
      colframe=rougeAlert,
      boxrule=1.5pt,
      arc=3mm,
      left=6pt, right=6pt, top=8pt, bottom=8pt,
      height=7.5cm
    ]
    \begin{center}
      \begin{tikzpicture}
        \fill[rougeAlert] (0,0) circle (0.7cm);
        \node[text=white, font=\large] at (0,0) {\faUserShield};
      \end{tikzpicture}

      \vspace{0.2cm}

      {\bfseries\textcolor{rougeAlert}{Administrateur}}
    \end{center}

    \vspace{0.2cm}

    {\small
      \textbf{Rôle :} Supervision et maintenance du système.

      \vspace{0.2cm}

      \textbf{Responsabilités :}
      \begin{itemize}[leftmargin=*, itemsep=1pt, label=\textcolor{rougeAlert}{\faAngleRight}]
        \item Consultation des audits
        \item Supervision services
        \item Gestion des logs
        \item Statistiques d'usage
      \end{itemize}
    }
  \end{tcolorbox}
\end{minipage}

\vspace{0.5cm}

%--- Schéma des interactions entre acteurs ---
\begin{figure}[H]
  \centering
  \begin{tikzpicture}[
      scale=0.85,
      actornode/.style={
        circle,
        draw=#1,
        fill=#1!20,
        line width=1.5pt,
        minimum size=1.5cm,
        font=\small\bfseries,
        align=center
      },
      systemnode/.style={
        rectangle,
        rounded corners=5pt,
        draw=bleuPrincipal,
        fill=bleuTresClair,
        line width=2pt,
        minimum width=5cm,
        minimum height=3cm,
        font=\large\bfseries
      },
      myarrow/.style={
        ->,
        >=stealth,
        line width=1.2pt,
        #1
      }
    ]

    % Système central
    \node[systemnode] (sys) at (0,0) {DocQA-MS};

    % Acteurs
    \node[actornode=bleuPrincipal] (user) at (-6,0) {\faUserMd\\Clinicien};
    \node[actornode=orangeWarning] (syst) at (0,-4) {\faCogs\\Système};
    \node[actornode=bleuMarine] (api) at (6,0) {\faBrain\\Ollama};
    \node[actornode=rougeAlert] (admin) at (0,4) {\faUserShield\\Admin};

    % Flèches
    \draw[myarrow=bleuPrincipal] (user) -- (sys) node[midway, above, font=\scriptsize] {Requêtes Q\&A};
    \draw[myarrow=bleuMarine] (sys) -- (api) node[midway, above, font=\scriptsize] {Inférence LLM};
    \draw[myarrow=orangeWarning] (syst) -- (sys) node[midway, right, font=\scriptsize] {Orchestration};
    \draw[myarrow=rougeAlert] (admin) -- (sys) node[midway, right, font=\scriptsize] {Supervision};

    % Flèche retour
    \draw[myarrow=bleuMarine, dashed] (api) to[bend right=20] node[midway, below, font=\scriptsize] {Réponses} (sys);

  \end{tikzpicture}
  \caption{Interactions entre les acteurs et le système DocQA-MS}
  \label{fig:acteurs}
\end{figure}

%==============================================================================
% SECTION 2.2 : SPÉCIFICATION DES BESOINS FONCTIONNELS
%==============================================================================

\newpage

\section{Spécification des Besoins Fonctionnels}

Les besoins fonctionnels décrivent les fonctionnalités que le système doit offrir. Ils sont organisés par microservice pour une meilleure traçabilité.

%------------------------------------------------------------------------------
% 2.2.1 Module Doc Ingestor
%------------------------------------------------------------------------------

\subsection{Module Doc Ingestor}

Ce module gère l'ingestion et le prétraitement des documents médicaux.

\vspace{0.3cm}

\begin{tcolorbox}[
    enhanced,
    colback=white,
    colframe=bleuTurquoise,
    boxrule=1.5pt,
    arc=3mm,
    left=8pt, right=8pt, top=8pt, bottom=8pt,
    title={\textcolor{white}{\faFileAlt~Module Doc Ingestor}},
    fonttitle=\bfseries\large,
    coltitle=white,
    attach boxed title to top left={yshift=-3mm, xshift=5mm},
    boxed title style={colback=bleuTurquoise, arc=2mm}
  ]

  \renewcommand{\arraystretch}{1.4}
  \begin{tabular}{|>{\columncolor{bleuTurquoise!10}\bfseries}c|p{7.5cm}|c|c|}
    \hline
    \rowcolor{bleuTurquoise}
    \textcolor{white}{\textbf{ID}} & \textcolor{white}{\textbf{Besoin Fonctionnel}} & \textcolor{white}{\textbf{Priorité}} & \textcolor{white}{\textbf{Sprint}} \\
    \hline
    BF-01 & Uploader des documents PDF, TXT, DOCX & \cellcolor{rougeAlert!20}Haute & 1 \\
    \hline
    BF-02 & Extraire le texte brut des documents & \cellcolor{rougeAlert!20}Haute & 1 \\
    \hline
    BF-03 & Découper le texte en chunks sémantiques & \cellcolor{rougeAlert!20}Haute & 1 \\
    \hline
    BF-04 & Stocker les métadonnées des documents & \cellcolor{orangeWarning!30}Moyenne & 2 \\
    \hline
  \end{tabular}

\end{tcolorbox}

%------------------------------------------------------------------------------
% 2.2.2 Module DeID Service
%------------------------------------------------------------------------------

\subsection{Module DeID Service (Anonymisation)}

Ce module assure l'anonymisation automatique des données personnelles de santé.

\vspace{0.3cm}

\begin{tcolorbox}[
    enhanced,
    colback=white,
    colframe=bleuMarine,
    boxrule=1.5pt,
    arc=3mm,
    left=8pt, right=8pt, top=8pt, bottom=8pt,
    title={\textcolor{white}{\faUserSecret~Module DeID Service}},
    fonttitle=\bfseries\large,
    coltitle=white,
    attach boxed title to top left={yshift=-3mm, xshift=5mm},
    boxed title style={colback=bleuMarine, arc=2mm}
  ]

  \renewcommand{\arraystretch}{1.4}
  \begin{tabular}{|>{\columncolor{bleuMarine!10}\bfseries}c|p{7.5cm}|c|c|}
    \hline
    \rowcolor{bleuMarine}
    \textcolor{white}{\textbf{ID}} & \textcolor{white}{\textbf{Besoin Fonctionnel}} & \textcolor{white}{\textbf{Priorité}} & \textcolor{white}{\textbf{Sprint}} \\
    \hline
    BF-05 & Détecter les entités nommées (NER médical) & \cellcolor{rougeAlert!20}Haute & 2 \\
    \hline
    BF-06 & Masquer les noms de patients & \cellcolor{rougeAlert!20}Haute & 2 \\
    \hline
    BF-07 & Anonymiser dates, adresses, numéros & \cellcolor{rougeAlert!20}Haute & 2 \\
    \hline
    BF-08 & Conserver la structure du texte original & \cellcolor{orangeWarning!30}Moyenne & 2 \\
    \hline
  \end{tabular}

\end{tcolorbox}

%------------------------------------------------------------------------------
% 2.2.3 Module Indexeur Sémantique
%------------------------------------------------------------------------------

\subsection{Module Indexeur Sémantique}

Ce module génère les embeddings et gère l'indexation vectorielle pour la recherche sémantique.

\vspace{0.3cm}

\begin{tcolorbox}[
    enhanced,
    colback=white,
    colframe=bleuPrincipal,
    boxrule=1.5pt,
    arc=3mm,
    left=8pt, right=8pt, top=8pt, bottom=8pt,
    title={\textcolor{white}{\faSearch~Module Indexeur Sémantique}},
    fonttitle=\bfseries\large,
    coltitle=white,
    attach boxed title to top left={yshift=-3mm, xshift=5mm},
    boxed title style={colback=bleuPrincipal, arc=2mm}
  ]

  \renewcommand{\arraystretch}{1.4}
  \begin{tabular}{|>{\columncolor{bleuPrincipal!10}\bfseries}c|p{7.5cm}|c|c|}
    \hline
    \rowcolor{bleuPrincipal}
    \textcolor{white}{\textbf{ID}} & \textcolor{white}{\textbf{Besoin Fonctionnel}} & \textcolor{white}{\textbf{Priorité}} & \textcolor{white}{\textbf{Sprint}} \\
    \hline
    BF-09 & Générer des embeddings vectoriels par chunk & \cellcolor{rougeAlert!20}Haute & 3 \\
    \hline
    BF-10 & Indexer les vecteurs dans une base vectorielle & \cellcolor{rougeAlert!20}Haute & 3 \\
    \hline
    BF-11 & Rechercher par similarité sémantique & \cellcolor{rougeAlert!20}Haute & 3 \\
    \hline
    BF-12 & Retourner les chunks les plus pertinents & \cellcolor{rougeAlert!20}Haute & 3 \\
    \hline
  \end{tabular}

\end{tcolorbox}

%------------------------------------------------------------------------------
% 2.2.4 Module LLM Q&A
%------------------------------------------------------------------------------

\subsection{Module LLM Q\&A (RAG)}

Ce module constitue le cœur du système, orchestrant le pipeline RAG pour générer des réponses.

\vspace{0.3cm}

\begin{tcolorbox}[
    enhanced,
    colback=white,
    colframe=bleuFonce,
    boxrule=1.5pt,
    arc=3mm,
    left=8pt, right=8pt, top=8pt, bottom=8pt,
    title={\textcolor{white}{\faRobot~Module LLM Q\&A (RAG)}},
    fonttitle=\bfseries\large,
    coltitle=white,
    attach boxed title to top left={yshift=-3mm, xshift=5mm},
    boxed title style={colback=bleuFonce, arc=2mm}
  ]

  \renewcommand{\arraystretch}{1.4}
  \begin{tabular}{|>{\columncolor{bleuFonce!10}\bfseries}c|p{7.5cm}|c|c|}
    \hline
    \rowcolor{bleuFonce}
    \textcolor{white}{\textbf{ID}} & \textcolor{white}{\textbf{Besoin Fonctionnel}} & \textcolor{white}{\textbf{Priorité}} & \textcolor{white}{\textbf{Sprint}} \\
    \hline
    BF-13 & Recevoir une question en langage naturel & \cellcolor{rougeAlert!20}Haute & 4 \\
    \hline
    BF-14 & Orchestrer la recherche sémantique (RAG) & \cellcolor{rougeAlert!20}Haute & 4 \\
    \hline
    BF-15 & Appeler Ollama (Llama 3.1) pour génération & \cellcolor{rougeAlert!20}Haute & 4 \\
    \hline
    BF-16 & Retourner réponse avec sources citées & \cellcolor{rougeAlert!20}Haute & 4 \\
    \hline
  \end{tabular}

\end{tcolorbox}

%------------------------------------------------------------------------------
% 2.2.5 Module Synthèse Comparative
%------------------------------------------------------------------------------

\subsection{Module Synthèse Comparative}

Ce module génère des synthèses à partir de plusieurs documents.

\vspace{0.3cm}

\begin{tcolorbox}[
    enhanced,
    colback=white,
    colframe=vertSucces,
    boxrule=1.5pt,
    arc=3mm,
    left=8pt, right=8pt, top=8pt, bottom=8pt,
    title={\textcolor{white}{\faLayerGroup~Module Synthèse Comparative}},
    fonttitle=\bfseries\large,
    coltitle=white,
    attach boxed title to top left={yshift=-3mm, xshift=5mm},
    boxed title style={colback=vertSucces, arc=2mm}
  ]

  \renewcommand{\arraystretch}{1.4}
  \begin{tabular}{|>{\columncolor{vertSucces!10}\bfseries}c|p{7.5cm}|c|c|}
    \hline
    \rowcolor{vertSucces}
    \textcolor{white}{\textbf{ID}} & \textcolor{white}{\textbf{Besoin Fonctionnel}} & \textcolor{white}{\textbf{Priorité}} & \textcolor{white}{\textbf{Sprint}} \\
    \hline
    BF-17 & Sélectionner plusieurs documents à comparer & \cellcolor{orangeWarning!30}Moyenne & 5 \\
    \hline
    BF-18 & Générer une synthèse comparative & \cellcolor{orangeWarning!30}Moyenne & 5 \\
    \hline
    BF-19 & Identifier les points communs et divergences & \cellcolor{orangeWarning!30}Moyenne & 5 \\
    \hline
  \end{tabular}

\end{tcolorbox}

%------------------------------------------------------------------------------
% 2.2.6 Module Audit Logger
%------------------------------------------------------------------------------

\subsection{Module Audit Logger}

Ce module assure la traçabilité complète des actions effectuées sur le système.

\vspace{0.3cm}

\begin{tcolorbox}[
    enhanced,
    colback=white,
    colframe=rougeAlert,
    boxrule=1.5pt,
    arc=3mm,
    left=8pt, right=8pt, top=8pt, bottom=8pt,
    title={\textcolor{white}{\faClipboardList~Module Audit Logger}},
    fonttitle=\bfseries\large,
    coltitle=white,
    attach boxed title to top left={yshift=-3mm, xshift=5mm},
    boxed title style={colback=rougeAlert, arc=2mm}
  ]

  \renewcommand{\arraystretch}{1.4}
  \begin{tabular}{|>{\columncolor{rougeAlert!10}\bfseries}c|p{7.5cm}|c|c|}
    \hline
    \rowcolor{rougeAlert}
    \textcolor{white}{\textbf{ID}} & \textcolor{white}{\textbf{Besoin Fonctionnel}} & \textcolor{white}{\textbf{Priorité}} & \textcolor{white}{\textbf{Sprint}} \\
    \hline
    BF-20 & Journaliser toutes les requêtes utilisateur & \cellcolor{rougeAlert!20}Haute & 1 \\
    \hline
    BF-21 & Enregistrer les accès aux documents & \cellcolor{rougeAlert!20}Haute & 1 \\
    \hline
    BF-22 & Fournir des statistiques d'utilisation & \cellcolor{orangeWarning!30}Moyenne & 5 \\
    \hline
    BF-23 & Permettre l'export des logs d'audit & \cellcolor{vertSucces!30}Basse & 5 \\
    \hline
  \end{tabular}

\end{tcolorbox}

%==============================================================================
% SECTION 2.3 : SPÉCIFICATION DES BESOINS NON FONCTIONNELS
%==============================================================================

\newpage

\section{Spécification des Besoins Non Fonctionnels}

Les besoins non fonctionnels définissent les critères de qualité du système.

%------------------------------------------------------------------------------
% 2.3.1 Performance
%------------------------------------------------------------------------------

\subsection{Performance}

\begin{tcolorbox}[
    enhanced,
    colback=white,
    colframe=bleuPrincipal,
    boxrule=1.5pt,
    arc=3mm,
    left=8pt, right=8pt, top=8pt, bottom=8pt,
    title={\textcolor{white}{\faTachometerAlt~Exigences de Performance}},
    fonttitle=\bfseries\large,
    coltitle=white,
    attach boxed title to top left={yshift=-3mm, xshift=5mm},
    boxed title style={colback=bleuPrincipal, arc=2mm}
  ]

  \renewcommand{\arraystretch}{1.4}
  \begin{tabular}{|>{\columncolor{bleuTresClair}\bfseries}c|p{5.5cm}|p{4.5cm}|c|}
    \hline
    \rowcolor{bleuPrincipal}
    \textcolor{white}{\textbf{ID}} & \textcolor{white}{\textbf{Besoin}} & \textcolor{white}{\textbf{Mesure/Critère}} & \textcolor{white}{\textbf{Priorité}} \\
    \hline
    BNF-01 & Temps de réponse API Gateway & < 200 ms & \cellcolor{rougeAlert!20}Haute \\
    \hline
    BNF-02 & Temps de réponse Q\&A LLM & < 10 secondes & \cellcolor{rougeAlert!20}Haute \\
    \hline
    BNF-03 & Ingestion d'un document PDF & < 5 secondes & \cellcolor{orangeWarning!30}Moyenne \\
    \hline
    BNF-04 & Anonymisation par document & < 3 secondes & \cellcolor{orangeWarning!30}Moyenne \\
    \hline
  \end{tabular}

\end{tcolorbox}

%------------------------------------------------------------------------------
% 2.3.2 Sécurité
%------------------------------------------------------------------------------

\subsection{Sécurité}

\begin{tcolorbox}[
    enhanced,
    colback=white,
    colframe=rougeAlert,
    boxrule=1.5pt,
    arc=3mm,
    left=8pt, right=8pt, top=8pt, bottom=8pt,
    title={\textcolor{white}{\faShieldAlt~Exigences de Sécurité}},
    fonttitle=\bfseries\large,
    coltitle=white,
    attach boxed title to top left={yshift=-3mm, xshift=5mm},
    boxed title style={colback=rougeAlert, arc=2mm}
  ]

  \renewcommand{\arraystretch}{1.4}
  \begin{tabular}{|>{\columncolor{rougeAlert!10}\bfseries}c|p{4.5cm}|p{5.5cm}|c|}
    \hline
    \rowcolor{rougeAlert}
    \textcolor{white}{\textbf{ID}} & \textcolor{white}{\textbf{Besoin}} & \textcolor{white}{\textbf{Description}} & \textcolor{white}{\textbf{Priorité}} \\
    \hline
    BNF-05 & Exécution locale du LLM & Aucune donnée envoyée vers cloud & \cellcolor{rougeAlert!20}Haute \\
    \hline
    BNF-06 & Anonymisation automatique & Conformité RGPD avant stockage & \cellcolor{rougeAlert!20}Haute \\
    \hline
    BNF-07 & Traçabilité complète & Audit de toutes les opérations & \cellcolor{rougeAlert!20}Haute \\
    \hline
    BNF-08 & Isolation des services & Chaque microservice isolé (Docker) & \cellcolor{orangeWarning!30}Moyenne \\
    \hline
  \end{tabular}

\end{tcolorbox}

%------------------------------------------------------------------------------
% 2.3.3 Maintenabilité
%------------------------------------------------------------------------------

\subsection{Maintenabilité}

\begin{tcolorbox}[
    enhanced,
    colback=white,
    colframe=orangeWarning,
    boxrule=1.5pt,
    arc=3mm,
    left=8pt, right=8pt, top=8pt, bottom=8pt,
    title={\textcolor{white}{\faWrench~Exigences de Maintenabilité}},
    fonttitle=\bfseries\large,
    coltitle=white,
    attach boxed title to top left={yshift=-3mm, xshift=5mm},
    boxed title style={colback=orangeWarning, arc=2mm}
  ]

  \renewcommand{\arraystretch}{1.4}
  \begin{tabular}{|>{\columncolor{orangeWarning!10}\bfseries}c|p{4.5cm}|p{5.5cm}|c|}
    \hline
    \rowcolor{orangeWarning}
    \textcolor{white}{\textbf{ID}} & \textcolor{white}{\textbf{Besoin}} & \textcolor{white}{\textbf{Description}} & \textcolor{white}{\textbf{Priorité}} \\
    \hline
    BNF-09 & Architecture microservices & Services indépendants et découplés & \cellcolor{rougeAlert!20}Haute \\
    \hline
    BNF-10 & Conteneurisation Docker & Déploiement reproductible & \cellcolor{rougeAlert!20}Haute \\
    \hline
    BNF-11 & CI/CD GitHub Actions & Intégration et déploiement continus & \cellcolor{orangeWarning!30}Moyenne \\
    \hline
    BNF-12 & Documentation API & Swagger/OpenAPI pour chaque service & \cellcolor{orangeWarning!30}Moyenne \\
    \hline
  \end{tabular}

\end{tcolorbox}

%==============================================================================
% SECTION 2.4 : DIAGRAMMES DE CAS D'UTILISATION
%==============================================================================

\newpage

\section{Diagrammes de Cas d'Utilisation}

Les diagrammes de cas d'utilisation UML permettent de visualiser les interactions entre les acteurs et le système.

%------------------------------------------------------------------------------
% 2.4.1 Diagramme Global
%------------------------------------------------------------------------------

\subsection{Diagramme de Cas d'Utilisation Global}

Le diagramme global présente une vue d'ensemble de toutes les fonctionnalités du système DocQA-MS.

\vspace{0.4cm}

\begin{figure}[H]
  \centering
  \includegraphics[width=0.9\textwidth]{images/usecase-diagram-api-gateway.png}
  \caption{Diagramme de Cas d'Utilisation Global de DocQA-MS}
  \label{fig:usecase_global}
\end{figure}

%==============================================================================
% SECTION 2.5 : BACKLOG PRODUIT
%==============================================================================

\section{Backlog Produit}

\begin{tcolorbox}[
    enhanced,
    colback=white,
    colframe=bleuPrincipal,
    boxrule=2pt,
    arc=4mm,
    left=8pt, right=8pt, top=8pt, bottom=8pt,
    title={\textcolor{white}{\faList~Product Backlog -- DocQA-MS}},
    fonttitle=\bfseries\large,
    coltitle=white,
    attach boxed title to top center={yshift=-3mm},
    boxed title style={colback=bleuPrincipal, arc=3mm}
  ]

  \renewcommand{\arraystretch}{1.3}
  \begin{longtable}{|>{\columncolor{bleuTresClair}\bfseries\small}c|p{7.5cm}|c|c|}
    \hline
    \rowcolor{bleuPrincipal}
    \textcolor{white}{\textbf{ID}} & \textcolor{white}{\textbf{User Story}} & \textcolor{white}{\textbf{Points}} & \textcolor{white}{\textbf{Sprint}} \\
    \hline
    \endfirsthead

    \multicolumn{4}{|c|}{\cellcolor{bleuTurquoise!20}\bfseries\textcolor{bleuTurquoise}{Sprint 1 -- Infrastructure \& Ingestion}} \\
    \hline
    US-01 & En tant que clinicien, je veux \textbf{uploader un document} pour l'analyser & 5 & 1 \\
    \hline
    US-02 & En tant que système, je veux \textbf{extraire le texte} d'un PDF & 3 & 1 \\
    \hline
    US-03 & En tant que système, je veux \textbf{journaliser les accès} pour l'audit & 5 & 1 \\
    \hline

    \multicolumn{4}{|c|}{\cellcolor{bleuMarine!20}\bfseries\textcolor{bleuMarine}{Sprint 2 -- Anonymisation}} \\
    \hline
    US-04 & En tant que clinicien, je veux \textbf{anonymiser un document} avant analyse & 8 & 2 \\
    \hline
    US-05 & En tant que système, je veux \textbf{détecter les entités médicales} (NER) & 8 & 2 \\
    \hline

    \multicolumn{4}{|c|}{\cellcolor{bleuPrincipal!20}\bfseries\textcolor{bleuPrincipal}{Sprint 3 -- Indexation}} \\
    \hline
    US-06 & En tant que système, je veux \textbf{générer des embeddings} pour chaque chunk & 8 & 3 \\
    \hline
    US-07 & En tant que clinicien, je veux \textbf{rechercher sémantiquement} dans mes documents & 8 & 3 \\
    \hline

    \multicolumn{4}{|c|}{\cellcolor{bleuFonce!20}\bfseries\textcolor{bleuFonce}{Sprint 4 -- Q\&A LLM}} \\
    \hline
    US-08 & En tant que clinicien, je veux \textbf{poser une question} en langage naturel & 13 & 4 \\
    \hline
    US-09 & En tant que clinicien, je veux \textbf{voir les sources} citées dans la réponse & 5 & 4 \\
    \hline

    \multicolumn{4}{|c|}{\cellcolor{vertSucces!20}\bfseries\textcolor{vertSucces}{Sprint 5 -- Synthèse \& Finitions}} \\
    \hline
    US-10 & En tant que clinicien, je veux \textbf{générer une synthèse} de plusieurs documents & 8 & 5 \\
    \hline
    US-11 & En tant qu'admin, je veux \textbf{consulter les statistiques} d'utilisation & 5 & 5 \\
    \hline

  \end{longtable}

\end{tcolorbox}

\vspace{0.5cm}

%--- Synthèse du Backlog ---
\begin{tcolorbox}[
    enhanced,
    colback=bleuTresClair,
    colframe=bleuClair,
    boxrule=1pt,
    arc=3mm,
    left=10pt, right=10pt, top=8pt, bottom=8pt
  ]
  \begin{center}
    {\large\bfseries\textcolor{bleuFonce}{Synthèse du Product Backlog}}
  \end{center}

  \vspace{0.3cm}

  \begin{minipage}[t]{0.48\textwidth}
    \begin{itemize}[leftmargin=*, itemsep=2pt, label=\textcolor{bleuPrincipal}{\faCheck}]
      \item \textbf{Total User Stories :} 11
      \item \textbf{Total Points :} 76 points
      \item \textbf{Nombre de Sprints :} 5
    \end{itemize}
  \end{minipage}
  \hfill
  \begin{minipage}[t]{0.48\textwidth}
    \begin{itemize}[leftmargin=*, itemsep=2pt, label=\textcolor{bleuPrincipal}{\faCheck}]
      \item \textbf{Vélocité moyenne :} 15,2 pts/sprint
      \item \textbf{Durée d'un sprint :} 1 semaine
      \item \textbf{Durée totale :} 5 semaines
    \end{itemize}
  \end{minipage}

\end{tcolorbox}

\vspace{0.5cm}

%--- Conclusion du chapitre ---
\begin{tcolorbox}[
    enhanced,
    colback=white,
    colframe=bleuPrincipal,
    boxrule=0pt,
    borderline south={3pt}{0pt}{bleuPrincipal},
    arc=0mm,
    left=10pt, right=10pt, top=10pt, bottom=10pt
  ]
  \textbf{\textcolor{bleuFonce}{Conclusion du Chapitre}}

  \vspace{0.2cm}

  Ce chapitre a permis de définir précisément les besoins du projet DocQA-MS. Nous avons identifié les acteurs du système, spécifié les besoins fonctionnels et non fonctionnels pour chaque microservice, et modélisé ces besoins à travers des diagrammes de cas d'utilisation. Le backlog produit ainsi constitué servira de feuille de route pour les phases de conception et de réalisation présentées dans les chapitres suivants.
\end{tcolorbox}

%==============================================================================
%                           CHAPITRE 3
%                           CONCEPTION
%==============================================================================

\chapter{Conception}

\begin{tikzpicture}[remember picture, overlay]
  % Décoration de page
  \fill[bleuClair, opacity=0.1]
  ([xshift=-3cm, yshift=-2cm]current page.north east) circle (4cm);
  \fill[bleuPrincipal, opacity=0.05]
  ([xshift=2cm, yshift=3cm]current page.south west) circle (5cm);
\end{tikzpicture}

\vspace{-0.5cm}

\begin{tcolorbox}[
    enhanced,
    colback=bleuTresClair,
    colframe=bleuPrincipal,
    boxrule=0pt,
    borderline west={4pt}{0pt}{bleuPrincipal},
    arc=0mm,
    left=12pt, right=12pt, top=10pt, bottom=10pt
  ]
  {\itshape\color{grisTexte}
    Ce chapitre présente la conception détaillée du système DocQA-MS. Nous aborderons l'architecture globale microservices, les diagrammes de classes pour chaque service, le modèle physique de données, les diagrammes de séquence illustrant les principaux flux, ainsi que la conception des interfaces utilisateur.
  }
\end{tcolorbox}

\vspace{0.5cm}

%==============================================================================
% SECTION 3.1 : ARCHITECTURE GLOBALE DU SYSTÈME
%==============================================================================

\section{Architecture Globale du Système}

L'architecture de DocQA-MS repose sur une conception microservices moderne, garantissant la scalabilité, la maintenabilité et l'isolation des composants critiques liés à la sécurité des données médicales.

%------------------------------------------------------------------------------
% 3.1.1 Architecture Logique
%------------------------------------------------------------------------------

\subsection{Architecture Logique (Microservices)}

L'application DocQA-MS adopte une architecture microservices distribuée, où chaque service encapsule une responsabilité métier spécifique.

\vspace{0.4cm}

\begin{figure}[H]
  \centering
  \begin{tikzpicture}[
      scale=0.75,
      transform shape,
      layer/.style={
        rectangle,
        rounded corners=5pt,
        draw=#1,
        fill=#1!10,
        line width=1.5pt,
        minimum width=15cm,
        minimum height=2cm,
        font=\bfseries
      },
      service/.style={
        rectangle,
        rounded corners=3pt,
        draw=#1,
        fill=white,
        line width=1pt,
        minimum width=2.5cm,
        minimum height=1cm,
        font=\small,
        align=center
      },
      myarrow/.style={
        <->,
        >=stealth,
        line width=1.2pt,
        #1
      }
    ]

    % Couche Frontend
    \node[layer=vertSucces] (pres) at (0,7) {};
    \node[above, font=\large\bfseries, text=vertSucces] at (0,8.2) {Couche Présentation};
    \node[service=vertSucces] at (0,7) {\faReact~React\\Frontend};

    % Flèche
    \draw[myarrow=bleuPrincipal] (0,5.8) -- (0,4.8) node[midway, right, font=\small] {HTTP/REST};

    % Couche Gateway
    \node[layer=bleuPrincipal] (gateway) at (0,3.5) {};
    \node[above, font=\large\bfseries, text=bleuFonce] at (0,4.7) {Couche Gateway};
    \node[service=bleuPrincipal] at (0,3.5) {\faServer~API Gateway\\(FastAPI)};

    % Flèche
    \draw[myarrow=bleuMarine] (0,2.2) -- (0,1.2) node[midway, right, font=\small] {Routage interne};

    % Couche Services
    \node[layer=bleuMarine] (services) at (0,-0.5) {};
    \node[above, font=\large\bfseries, text=bleuMarine] at (0,0.7) {Couche Microservices};

    \node[service=bleuTurquoise] at (-5.5,-0.5) {\faFileAlt\\Doc Ingestor};
    \node[service=bleuMarine] at (-2.2,-0.5) {\faUserSecret\\DeID Service};
    \node[service=bleuPrincipal] at (1.1,-0.5) {\faSearch\\Indexeur};
    \node[service=bleuFonce] at (4.4,-0.5) {\faRobot\\LLM Q\&A};

    % Services secondaires (ligne du bas)
    \node[service=vertSucces] at (-3.3,-2.5) {\faLayerGroup\\Synthèse};
    \node[service=rougeAlert] at (0,-2.5) {\faClipboardList\\Audit Logger};
    
    % Flèche
    \draw[myarrow=gray] (0,-3.5) -- (0,-4.3) node[midway, right, font=\small] {Persistance};

    % Couche Infrastructure
    \node[layer=gray] (infra) at (0,-5.5) {};
    \node[above, font=\large\bfseries, text=gray] at (0,-4.3) {Couche Infrastructure};

    \node[service=gray] at (-4,-5.5) {\faDatabase\\PostgreSQL};
    \node[service=gray] at (0,-5.5) {\faEnvelope\\RabbitMQ};
    \node[service=gray] at (4,-5.5) {\faBrain\\Ollama};

  \end{tikzpicture}
  \caption{Architecture logique microservices de DocQA-MS}
  \label{fig:archi-logique}
\end{figure}

\vspace{0.3cm}

\begin{tcolorbox}[
    enhanced,
    colback=bleuTresClair,
    colframe=bleuClair,
    boxrule=1pt,
    arc=2mm,
    left=8pt, right=8pt, top=6pt, bottom=6pt
  ]
  \textbf{\textcolor{bleuFonce}{Description des couches :}}

  \begin{itemize}[leftmargin=*, itemsep=3pt, label=\textcolor{bleuPrincipal}{\faAngleRight}]
    \item \textbf{Couche Présentation :} Interface React moderne communiquant via API REST avec le backend.
    \item \textbf{Couche Gateway :} Point d'entrée unique (API Gateway FastAPI) gérant le routage, l'authentification et la validation.
    \item \textbf{Couche Microservices :} 6 services spécialisés (Python/FastAPI et Java/Spring Boot) encapsulant la logique métier.
    \item \textbf{Couche Infrastructure :} PostgreSQL (persistance), RabbitMQ (messaging asynchrone), Ollama (LLM local).
  \end{itemize}
\end{tcolorbox}

%------------------------------------------------------------------------------
% 3.1.2 Architecture Physique (Déploiement)
%------------------------------------------------------------------------------

\newpage

\subsection{Architecture Physique (Docker Compose)}

Le diagramme de déploiement illustre la distribution physique des composants conteneurisés et leurs interconnexions réseau.

\vspace{0.4cm}

\begin{figure}[H]
  \centering
  \begin{tikzpicture}[
      scale=0.7,
      transform shape,
      node distance=1cm,
      container/.style={
        rectangle,
        rounded corners=5pt,
        draw=#1,
        fill=#1!5,
        line width=1.5pt,
        minimum width=3cm,
        minimum height=1.3cm,
        font=\small,
        align=center
      },
      dockernet/.style={
        rectangle,
        rounded corners=8pt,
        draw=bleuPrincipal,
        fill=bleuTresClair,
        line width=2pt,
        minimum width=16cm,
        minimum height=8cm
      },
      myarrow/.style={
        <->,
        >=stealth,
        line width=1pt,
        #1
      }
    ]

    % Docker Network
    \node[dockernet] (network) at (0,0) {};
    \node[above, font=\bfseries, text=bleuFonce] at (0,4.5) {Docker Network: docqa-network};

    % Containers - Ligne 1
    \node[container=bleuPrincipal] (gateway) at (-5,2.5) {\faServer~api-gateway\\:8000};
    \node[container=bleuTurquoise] (ingestor) at (-1.5,2.5) {\faFileAlt~doc-ingestor\\:8001};
    \node[container=bleuMarine] (deid) at (2,2.5) {\faUserSecret~deid-service\\:8002};
    \node[container=bleuPrincipal] (indexer) at (5.5,2.5) {\faSearch~indexeur\\:8003};

    % Containers - Ligne 2
    \node[container=bleuFonce] (llm) at (-3.5,0) {\faRobot~llm-qa-module\\:8004};
    \node[container=vertSucces] (synthese) at (0.5,0) {\faLayerGroup~synthese\\:8005};
    \node[container=rougeAlert] (audit) at (4.5,0) {\faClipboardList~audit-logger\\:8006};

    % Containers - Ligne 3 (Infrastructure)
    \node[container=gray] (postgres) at (-3.5,-2.5) {\faDatabase~postgres\\:5432};
    \node[container=gray] (rabbitmq) at (0.5,-2.5) {\faEnvelope~rabbitmq\\:5672};
    \node[container=gray] (ollama) at (4.5,-2.5) {\faBrain~ollama\\:11434};

    % Frontend externe
    \node[container=vertSucces] (react) at (-5,5.5) {\faReact~interface-clinique\\:3000};

    % Connexions
    \draw[myarrow=bleuPrincipal] (react) -- (gateway);

  \end{tikzpicture}
  \caption{Architecture de déploiement Docker Compose}
  \label{fig:archi-physique}
\end{figure}

\vspace{0.3cm}

\begin{tcolorbox}[
    enhanced,
    colback=white,
    colframe=bleuPrincipal,
    boxrule=1pt,
    arc=2mm,
    left=8pt, right=8pt, top=6pt, bottom=6pt
  ]
  \textbf{\textcolor{bleuFonce}{Mapping des ports :}}

  \renewcommand{\arraystretch}{1.2}
  \begin{center}
  \begin{tabular}{|l|c|l|}
    \hline
    \rowcolor{bleuPrincipal}
    \textcolor{white}{\textbf{Service}} & \textcolor{white}{\textbf{Port}} & \textcolor{white}{\textbf{Technologie}} \\
    \hline
    API Gateway & 8000 & Python / FastAPI \\
    \hline
    Doc Ingestor & 8001 & Python / FastAPI \\
    \hline
    DeID Service & 8002 & Java / Spring Boot \\
    \hline
    Indexeur Sémantique & 8003 & Java / Spring Boot \\
    \hline
    LLM Q\&A Module & 8004 & Python / FastAPI \\
    \hline
    Synthèse Comparative & 8005 & Java / Spring Boot \\
    \hline
    Audit Logger & 8006 & Java / Spring Boot \\
    \hline
    Interface Clinique & 3000 & React / Node.js \\
    \hline
  \end{tabular}
  \end{center}
\end{tcolorbox}

%------------------------------------------------------------------------------
% 3.1.3 Flux de Communication
%------------------------------------------------------------------------------

\subsection{Flux de Communication entre Services}

Les microservices communiquent selon deux modes : synchrone (HTTP/REST) et asynchrone (RabbitMQ).

\vspace{0.4cm}

\begin{tcolorbox}[
    enhanced,
    colback=white,
    colframe=bleuPrincipal,
    boxrule=1.5pt,
    arc=3mm,
    left=10pt, right=10pt, top=10pt, bottom=10pt,
    title={\textcolor{white}{\faExchangeAlt~Modes de Communication}},
    fonttitle=\bfseries,
    coltitle=white,
    attach boxed title to top left={yshift=-3mm, xshift=5mm},
    boxed title style={colback=bleuPrincipal, arc=2mm}
  ]

  \begin{minipage}[t]{0.48\textwidth}
    \textbf{\textcolor{bleuFonce}{Synchrone (HTTP/REST)}}
    \begin{itemize}[leftmargin=*, itemsep=2pt, label=\textcolor{bleuPrincipal}{\faAngleRight}]
      \item Frontend $\rightarrow$ API Gateway
      \item Gateway $\rightarrow$ Services métier
      \item Services $\rightarrow$ Ollama (LLM)
      \item Services $\rightarrow$ PostgreSQL
    \end{itemize}
  \end{minipage}
  \hfill
  \begin{minipage}[t]{0.48\textwidth}
    \textbf{\textcolor{bleuFonce}{Asynchrone (RabbitMQ)}}
    \begin{itemize}[leftmargin=*, itemsep=2pt, label=\textcolor{orangeWarning}{\faAngleRight}]
      \item Doc Ingestor $\rightarrow$ DeID Service
      \item DeID Service $\rightarrow$ Indexeur
      \item Tous services $\rightarrow$ Audit Logger
    \end{itemize}
  \end{minipage}

\end{tcolorbox}

%==============================================================================
% SECTION 3.2 : CONCEPTION DÉTAILLÉE
%==============================================================================

\newpage

\section{Conception Détaillée}

Cette section présente les diagrammes de classes détaillés pour les principaux modules du système.

%------------------------------------------------------------------------------
% 3.2.1 Diagramme de Classes : API Gateway
%------------------------------------------------------------------------------

\subsection{Diagramme de Classes : API Gateway}

\begin{figure}[H]
  \centering
  \begin{tikzpicture}[
      scale=0.72,
      transform shape,
      classname/.style={
        rectangle,
        draw=bleuPrincipal,
        fill=bleuPrincipal!20,
        line width=1.5pt,
        minimum width=5.5cm,
        minimum height=0.7cm,
        font=\small\bfseries,
        text=bleuFonce
      },
      attr/.style={
        rectangle,
        draw=bleuPrincipal,
        fill=white,
        line width=1pt,
        minimum width=5.5cm,
        font=\scriptsize,
        align=left,
        text=black
      },
      myarrow/.style={
        ->,
        >=stealth,
        line width=1pt,
        bleuPrincipal
      }
    ]

    % GatewayRouter
    \node[classname] (router) at (0,6) {GatewayRouter};
    \node[attr, below=0pt of router, text width=5.2cm] (routerattr) {
      - routes: Dict[str, str]\\
      - services: ServiceRegistry
    };
    \node[attr, below=0pt of routerattr, text width=5.2cm] (routermeth) {
      + route\_request(path, method)\\
      + health\_check(): Dict
    };

    % ServiceRegistry
    \node[classname] (registry) at (7,6) {ServiceRegistry};
    \node[attr, below=0pt of registry, text width=5.2cm] (regattr) {
      - services: Dict[str, ServiceInfo]
    };
    \node[attr, below=0pt of regattr, text width=5.2cm] (regmeth) {
      + register(name, url)\\
      + discover(name): str\\
      + check\_health(): List[bool]
    };

    % DocumentController
    \node[classname] (docctrl) at (0,2) {DocumentController};
    \node[attr, below=0pt of docctrl, text width=5.2cm] (docctrlmeth) {
      + upload\_document(file): Response\\
      + get\_document(id): Document
    };

    % QAController
    \node[classname] (qactrl) at (7,2) {QAController};
    \node[attr, below=0pt of qactrl, text width=5.2cm] (qactrlmeth) {
      + ask\_question(query): Answer\\
      + get\_history(): List[QA]
    };

    % Relations
    \draw[myarrow] (router) -- (registry);
    \draw[myarrow] (router) -- (docctrl);
    \draw[myarrow] (router) -- (qactrl);

  \end{tikzpicture}
  \caption{Diagramme de classes -- API Gateway}
  \label{fig:class-gateway}
\end{figure}

%------------------------------------------------------------------------------
% 3.2.2 Diagramme de Classes : LLM Q&A Module
%------------------------------------------------------------------------------

\subsection{Diagramme de Classes : Module LLM Q\&A (RAG)}

\begin{figure}[H]
  \centering
  \begin{tikzpicture}[
      scale=0.72,
      transform shape,
      classname/.style={
        rectangle,
        draw=bleuFonce,
        fill=bleuFonce!20,
        line width=1.5pt,
        minimum width=5.5cm,
        minimum height=0.7cm,
        font=\small\bfseries,
        text=bleuFonce
      },
      attr/.style={
        rectangle,
        draw=bleuFonce,
        fill=white,
        line width=1pt,
        minimum width=5.5cm,
        font=\scriptsize,
        align=left,
        text=black
      },
      myarrow/.style={
        ->,
        >=stealth,
        line width=1pt,
        bleuFonce
      }
    ]

    % RAGService
    \node[classname] (ragsvc) at (0,6) {RAGService};
    \node[attr, below=0pt of ragsvc, text width=5.2cm] (ragsvcattr) {
      - vectorStore: VectorStore\\
      - llmClient: OllamaClient\\
      - embedder: EmbeddingModel
    };
    \node[attr, below=0pt of ragsvcattr, text width=5.2cm] (ragsvcmeth) {
      + process\_question(query): Answer\\
      - retrieve\_context(query): List[Chunk]\\
      - generate\_answer(ctx, query): str
    };

    % OllamaClient
    \node[classname] (ollama) at (7,6) {OllamaClient};
    \node[attr, below=0pt of ollama, text width=5.2cm] (ollamaattr) {
      - base\_url: str\\
      - model: str = "llama3.1"
    };
    \node[attr, below=0pt of ollamaattr, text width=5.2cm] (ollamameth) {
      + generate(prompt): str\\
      + embed(text): List[float]
    };

    % VectorStore
    \node[classname] (vector) at (-3.5,1.5) {VectorStore};
    \node[attr, below=0pt of vector, text width=5.2cm] (vectormeth) {
      + add(embedding, metadata)\\
      + search(query, k): List[Chunk]
    };

    % Chunk
    \node[classname] (chunk) at (3.5,1.5) {Chunk};
    \node[attr, below=0pt of chunk, text width=5.2cm] (chunkattr) {
      - id: str\\
      - content: str\\
      - document\_id: str\\
      - score: float
    };

    % Relations
    \draw[myarrow] (ragsvc) -- (ollama);
    \draw[myarrow] (ragsvc) -- (vector);
    \draw[myarrow] (vector) -- (chunk);

  \end{tikzpicture}
  \caption{Diagramme de classes -- Module LLM Q\&A (RAG)}
  \label{fig:class-rag}
\end{figure}

%------------------------------------------------------------------------------
% 3.2.3 Diagramme de Classes : DeID Service
%------------------------------------------------------------------------------

\subsection{Diagramme de Classes : DeID Service (Anonymisation)}

\begin{figure}[H]
  \centering
  \begin{tikzpicture}[
      scale=0.72,
      transform shape,
      classname/.style={
        rectangle,
        draw=bleuMarine,
        fill=bleuMarine!20,
        line width=1.5pt,
        minimum width=5.5cm,
        minimum height=0.7cm,
        font=\small\bfseries,
        text=bleuMarine
      },
      attr/.style={
        rectangle,
        draw=bleuMarine,
        fill=white,
        line width=1pt,
        minimum width=5.5cm,
        font=\scriptsize,
        align=left,
        text=black
      },
      myarrow/.style={
        ->,
        >=stealth,
        line width=1pt,
        bleuMarine
      }
    ]

    % DeIdService
    \node[classname] (deidsvc) at (0,6) {DeIdService};
    \node[attr, below=0pt of deidsvc, text width=5.2cm] (deidsvcattr) {
      - nerModel: NERModel\\
      - maskingRules: List<Rule>
    };
    \node[attr, below=0pt of deidsvcattr, text width=5.2cm] (deidsvcmeth) {
      + anonymize(text): AnonymizedText\\
      - detectEntities(text): List<Entity>\\
      - applyMasking(entities): String
    };

    % NERModel
    \node[classname] (ner) at (7,6) {NERModel};
    \node[attr, below=0pt of ner, text width=5.2cm] (nerattr) {
      - modelPath: String
    };
    \node[attr, below=0pt of nerattr, text width=5.2cm] (nermeth) {
      + predict(text): List<Entity>
    };

    % Entity
    \node[classname] (entity) at (0,1.5) {Entity};
    \node[attr, below=0pt of entity, text width=5.2cm] (entityattr) {
      - type: EntityType\\
      - value: String\\
      - start: int\\
      - end: int
    };

    % EntityType (enum)
    \node[classname] (entitytype) at (7,1.5) {<<enum>> EntityType};
    \node[attr, below=0pt of entitytype, text width=5.2cm] (entitytypeattr) {
      PERSON, DATE, ADDRESS,\\
      PHONE, SSN, MEDICAL\_ID
    };

    % Relations
    \draw[myarrow] (deidsvc) -- (ner);
    \draw[myarrow] (deidsvc) -- (entity);
    \draw[myarrow] (entity) -- (entitytype);

  \end{tikzpicture}
  \caption{Diagramme de classes -- DeID Service (Anonymisation)}
  \label{fig:class-deid}
\end{figure}

%==============================================================================
% SECTION 3.3 : DIAGRAMMES DE SÉQUENCE
%==============================================================================

\newpage

\section{Diagrammes de Séquence}

Les diagrammes de séquence illustrent les interactions temporelles entre les composants pour les principaux scénarios.

%------------------------------------------------------------------------------
% 3.3.1 Séquence : Pipeline RAG complet
%------------------------------------------------------------------------------

\subsection{Séquence : Question-Réponse (Pipeline RAG)}

\begin{figure}[H]
  \centering
  \resizebox{\textwidth}{!}{%
    \begin{tikzpicture}[
        transform shape,
        actor/.style={font=\small\bfseries, align=center},
        lifeline/.style={dashed, gray!50, line width=0.8pt},
        message/.style={->, >=stealth, line width=1pt, #1},
        return/.style={->, >=stealth, dashed, line width=0.8pt, #1}
      ]

      % Acteurs et composants
      \node[actor] (user) at (0,0) {\faUserMd\\Clinicien};
      \node[actor] (react) at (3,0) {\faReact\\Frontend};
      \node[actor] (gateway) at (6,0) {\faServer\\Gateway};
      \node[actor] (llm) at (9,0) {\faRobot\\LLM Q\&A};
      \node[actor] (indexer) at (12,0) {\faSearch\\Indexeur};
      \node[actor] (ollama) at (15,0) {\faBrain\\Ollama};

      % Lignes de vie
      \foreach \x in {0,3,6,9,12,15} {
        \draw[lifeline] (\x,-0.8) -- (\x,-14);
      }

      % Messages
      \draw[message=bleuPrincipal] (0,-1.5) -- (3,-1.5) node[midway, above, font=\scriptsize] {1: Saisir question};
      \draw[message=bleuPrincipal] (3,-2.5) -- (6,-2.5) node[midway, above, font=\scriptsize] {2: POST /api/qa/ask};
      \draw[message=bleuPrincipal] (6,-3.5) -- (9,-3.5) node[midway, above, font=\scriptsize] {3: forward(query)};

      \draw[message=bleuMarine] (9,-4.5) -- (12,-4.5) node[midway, above, font=\scriptsize] {4: search(embedding)};
      \draw[return=gray] (12,-5.5) -- (9,-5.5) node[midway, above, font=\scriptsize] {5: relevant\_chunks};

      \draw[message=bleuFonce] (9,-6.5) -- (9,-7) node[right, font=\scriptsize] {6: build\_prompt()};

      \draw[message=bleuPrincipal] (9,-8) -- (15,-8) node[midway, above, font=\scriptsize] {7: generate(prompt)};
      \draw[return=gray] (15,-9.5) -- (9,-9.5) node[midway, above, font=\scriptsize] {8: llm\_response};

      \draw[return=gray] (9,-10.5) -- (6,-10.5) node[midway, above, font=\scriptsize] {9: Answer + sources};
      \draw[return=gray] (6,-11.5) -- (3,-11.5) node[midway, above, font=\scriptsize] {10: 200 OK + JSON};
      \draw[message=vertSucces] (3,-12.5) -- (0,-12.5) node[midway, above, font=\scriptsize] {11: Afficher réponse};

    \end{tikzpicture}%
  }
  \caption{Diagramme de séquence -- Pipeline RAG (Question-Réponse)}
  \label{fig:seq-rag}
\end{figure}

%------------------------------------------------------------------------------
% 3.3.2 Séquence : Ingestion et Anonymisation
%------------------------------------------------------------------------------

\subsection{Séquence : Ingestion et Anonymisation de Document}

\begin{figure}[H]
  \centering
  \begin{tikzpicture}[
      scale=0.65,
      transform shape,
      actor/.style={font=\small\bfseries, align=center},
      lifeline/.style={dashed, gray!50, line width=0.8pt},
      message/.style={->, >=stealth, line width=1pt, #1},
      return/.style={->, >=stealth, dashed, line width=0.8pt, #1}
    ]

    % Acteurs et composants
    \node[actor] (user) at (0,0) {\faUserMd\\Clinicien};
    \node[actor] (gateway) at (3,0) {\faServer\\Gateway};
    \node[actor] (ingestor) at (6,0) {\faFileAlt\\Ingestor};
    \node[actor] (deid) at (9,0) {\faUserSecret\\DeID};
    \node[actor] (indexer) at (12,0) {\faSearch\\Indexeur};
    \node[actor] (audit) at (15,0) {\faClipboardList\\Audit};

    % Lignes de vie
    \foreach \x in {0,3,6,9,12,15} {
      \draw[lifeline] (\x,-0.5) -- (\x,-13);
    }

    % Messages
    \draw[message=bleuPrincipal] (0,-1) -- (3,-1) node[midway, above, font=\scriptsize] {1: upload(file)};
    \draw[message=bleuPrincipal] (3,-2) -- (6,-2) node[midway, above, font=\scriptsize] {2: ingest(file)};
    \draw[message=bleuTurquoise] (6,-3) -- (6,-3.5) node[right, font=\scriptsize] {3: extract\_text()};
    \draw[message=bleuTurquoise] (6,-4) -- (6,-4.5) node[right, font=\scriptsize] {4: chunk\_text()};

    \draw[message=bleuMarine] (6,-5.5) -- (9,-5.5) node[midway, above, font=\scriptsize] {5: anonymize(chunks)};
    \draw[return=gray] (9,-6.5) -- (6,-6.5) node[midway, above, font=\scriptsize] {6: anonymized};

    \draw[message=bleuPrincipal] (6,-7.5) -- (12,-7.5) node[midway, above, font=\scriptsize] {7: index(chunks)};
    \draw[return=gray] (12,-8.5) -- (6,-8.5) node[midway, above, font=\scriptsize] {8: indexed};

    \draw[message=rougeAlert] (6,-9.5) -- (15,-9.5) node[midway, above, font=\scriptsize] {9: log(action)};

    \draw[return=gray] (6,-10.5) -- (3,-10.5) node[midway, above, font=\scriptsize] {10: success};
    \draw[message=vertSucces] (3,-11.5) -- (0,-11.5) node[midway, above, font=\scriptsize] {11: confirmation};

  \end{tikzpicture}
  \caption{Diagramme de séquence -- Ingestion et Anonymisation}
  \label{fig:seq-ingest}
\end{figure}

%==============================================================================
% SECTION 3.4 : CONCEPTION DES INTERFACES
%==============================================================================

\newpage

\section{Conception des Interfaces Utilisateur}

Cette section présente la conception des interfaces utilisateur de l'application DocQA-MS.

%------------------------------------------------------------------------------
% 3.4.1 Wireframes
%------------------------------------------------------------------------------

\subsection{Wireframes / Maquettes}

\begin{tcolorbox}[
    enhanced,
    colback=white,
    colframe=bleuPrincipal,
    boxrule=1.5pt,
    arc=3mm,
    left=10pt, right=10pt, top=10pt, bottom=10pt,
    title={\textcolor{white}{\faDesktop~Écrans Principaux}},
    fonttitle=\bfseries\large,
    coltitle=white,
    attach boxed title to top left={yshift=-3mm, xshift=5mm},
    boxed title style={colback=bleuPrincipal, arc=2mm}
  ]

  \begin{minipage}[t]{0.48\textwidth}
    \textbf{\textcolor{bleuFonce}{Dashboard}}
    \begin{itemize}[leftmargin=*, itemsep=2pt, label=\textcolor{bleuPrincipal}{\faAngleRight}]
      \item Statistiques d'utilisation
      \item Documents récents
      \item Questions récentes
      \item Accès rapide aux fonctions
    \end{itemize}
  \end{minipage}
  \hfill
  \begin{minipage}[t]{0.48\textwidth}
    \textbf{\textcolor{bleuFonce}{Interface Q\&A}}
    \begin{itemize}[leftmargin=*, itemsep=2pt, label=\textcolor{bleuPrincipal}{\faAngleRight}]
      \item Zone de saisie de question
      \item Affichage de la réponse
      \item Sources citées avec liens
      \item Historique des conversations
    \end{itemize}
  \end{minipage}

  \vspace{0.4cm}

  \begin{minipage}[t]{0.48\textwidth}
    \textbf{\textcolor{bleuFonce}{Upload Documents}}
    \begin{itemize}[leftmargin=*, itemsep=2pt, label=\textcolor{bleuPrincipal}{\faAngleRight}]
      \item Drag \& drop de fichiers
      \item Barre de progression
      \item Option d'anonymisation
      \item Prévisualisation
    \end{itemize}
  \end{minipage}
  \hfill
  \begin{minipage}[t]{0.48\textwidth}
    \textbf{\textcolor{bleuFonce}{Journal d'Audit}}
    \begin{itemize}[leftmargin=*, itemsep=2pt, label=\textcolor{bleuPrincipal}{\faAngleRight}]
      \item Liste des événements
      \item Filtres par date/type
      \item Détails des actions
      \item Export des logs
    \end{itemize}
  \end{minipage}

\end{tcolorbox}

%------------------------------------------------------------------------------
% 3.4.2 Charte Graphique
%------------------------------------------------------------------------------

\subsection{Charte Graphique}

\begin{tcolorbox}[
    enhanced,
    colback=white,
    colframe=bleuPrincipal,
    boxrule=1.5pt,
    arc=3mm,
    left=10pt, right=10pt, top=10pt, bottom=10pt,
    title={\textcolor{white}{\faPalette~Palette de Couleurs}},
    fonttitle=\bfseries\large,
    coltitle=white,
    attach boxed title to top left={yshift=-3mm, xshift=5mm},
    boxed title style={colback=bleuPrincipal, arc=2mm}
  ]

  \begin{center}
    \begin{tikzpicture}[scale=0.9]
      % Couleurs
      \foreach \x/\color/\hex/\name in {
        0/{rgb,255:red,41;green,128;blue,185}/#2980B9/Bleu Principal,
        2.8/{rgb,255:red,23;green,74;blue,117}/#174A75/Bleu Foncé,
        5.6/{rgb,255:red,72;green,201;blue,176}/#48C9B0/Turquoise,
        8.4/{rgb,255:red,39;green,174;blue,96}/#27AE60/Succès,
        11.2/{rgb,255:red,231;green,76;blue,60}/#E74C3C/Alerte,
        14/{rgb,255:red,51;green,51;blue,51}/#333333/Texte
      } {
        \fill[\color] (\x,0) rectangle (\x+2.2,1.5);
        \draw[gray!50] (\x,0) rectangle (\x+2.2,1.5);
        \node[below, font=\scriptsize\bfseries] at (\x+1.1,0) {\hex};
        \node[below, font=\tiny] at (\x+1.1,-0.4) {\name};
      }
    \end{tikzpicture}
  \end{center}

\end{tcolorbox}

\vspace{0.5cm}

%==============================================================================
% SECTION 3.5 : DIAGRAMMES UML DÉTAILLÉS
%==============================================================================

\newpage

\section{Diagrammes UML Détaillés}

Cette section présente les diagrammes UML générés à partir de la modélisation du système.

\subsection{Diagramme de Classes Global}

\begin{figure}[H]
  \centering
  \includegraphics[width=0.95\textwidth]{images/class-diagram.png}
  \caption{Diagramme de classes global du système DocQA-MS}
  \label{fig:class-diagram-full}
\end{figure}

\subsection{Diagramme de Classes -- Service DeID}

\begin{figure}[H]
  \centering
  \includegraphics[width=0.9\textwidth]{images/class-diagram-deid.png}
  \caption{Diagramme de classes du service de dé-identification (DeID)}
  \label{fig:class-diagram-deid}
\end{figure}

\subsection{Diagrammes de Cas d'Utilisation}

\begin{figure}[H]
  \centering
  \includegraphics[width=0.9\textwidth]{images/usecase-diagram-indexeur.png}
  \caption{Diagramme de cas d'utilisation -- Indexeur Sémantique}
  \label{fig:usecase-indexeur}
\end{figure}

\begin{figure}[H]
  \centering
  \includegraphics[width=0.85\textwidth]{images/usecase-diagram-LLM.png}
  \caption{Diagramme de cas d'utilisation -- Module LLM Q\&A}
  \label{fig:usecase-llm}
\end{figure}

\subsection{Diagramme d'Architecture Globale}

\begin{figure}[H]
  \centering
  \includegraphics[width=0.9\textwidth]{images/architecture-global-diagram.png}
  \caption{Architecture globale du système DocQA-MS en microservices}
  \label{fig:archi-diagram-full}
\end{figure}

\vspace{0.5cm}

%--- Conclusion du chapitre ---
\begin{tcolorbox}[
    enhanced,
    colback=white,
    colframe=bleuPrincipal,
    boxrule=0pt,
    borderline south={3pt}{0pt}{bleuPrincipal},
    arc=0mm,
    left=10pt, right=10pt, top=10pt, bottom=10pt
  ]
  \textbf{\textcolor{bleuFonce}{Conclusion du Chapitre}}

  \vspace{0.2cm}

  Ce chapitre a présenté la conception complète du système DocQA-MS. Nous avons détaillé l'architecture microservices avec ses 7 services distribués, les diagrammes de classes pour les modules clés (Gateway, RAG, DeID), les diagrammes de séquence illustrant les flux principaux (Q\&A et Ingestion), ainsi que la conception des interfaces utilisateur. Cette phase de conception prépare la réalisation technique présentée dans le chapitre suivant.
\end{tcolorbox}

%==============================================================================
%                           CHAPITRE 4
%                          RÉALISATION
%==============================================================================

\chapter{Réalisation}

\begin{tikzpicture}[remember picture, overlay]
  % Décoration de page
  \fill[bleuClair, opacity=0.1]
  ([xshift=3cm, yshift=-2cm]current page.north west) circle (4cm);
  \fill[bleuPrincipal, opacity=0.05]
  ([xshift=-2cm, yshift=3cm]current page.south east) circle (5cm);
\end{tikzpicture}

\vspace{-0.5cm}

\begin{tcolorbox}[
    enhanced,
    colback=bleuTresClair,
    colframe=bleuPrincipal,
    boxrule=0pt,
    borderline west={4pt}{0pt}{bleuPrincipal},
    arc=0mm,
    left=12pt, right=12pt, top=10pt, bottom=10pt
  ]
  {\itshape\color{grisTexte}
    Ce chapitre présente la phase de réalisation du projet DocQA-MS. Nous détaillerons l'environnement de développement, l'implémentation des microservices Python et Java, le développement de l'interface React, ainsi que les tests et l'assurance qualité incluant les pipelines CI/CD et les tests de performance JMeter.
  }
\end{tcolorbox}

\vspace{0.5cm}

%==============================================================================
% SECTION 4.1 : ENVIRONNEMENT DE DÉVELOPPEMENT
%==============================================================================

\section{Environnement de Développement}

Cette section présente l'environnement technique mis en place pour le développement du projet DocQA-MS.

%------------------------------------------------------------------------------
% 4.1.1 Configuration Matérielle
%------------------------------------------------------------------------------

\subsection{Configuration Matérielle}

Le développement de DocQA-MS a été réalisé sur des postes de travail répondant aux spécifications suivantes :

\vspace{0.4cm}

\begin{tcolorbox}[
    enhanced,
    colback=white,
    colframe=bleuPrincipal,
    boxrule=1.5pt,
    arc=3mm,
    left=10pt, right=10pt, top=10pt, bottom=10pt,
    title={\textcolor{white}{\faDesktop~Configuration Matérielle}},
    fonttitle=\bfseries\large,
    coltitle=white,
    attach boxed title to top left={yshift=-3mm, xshift=5mm},
    boxed title style={colback=bleuPrincipal, arc=2mm}
  ]

  \renewcommand{\arraystretch}{1.4}
  \begin{center}
    \begin{tabular}{|>{\columncolor{bleuTresClair}}p{4cm}|p{8cm}|}
      \hline
      \rowcolor{bleuPrincipal}
      \textcolor{white}{\textbf{Composant}} & \textcolor{white}{\textbf{Spécification}} \\
      \hline
      \textbf{Processeur} & Intel Core i5/i7 (10ème génération ou supérieur) \\
      \hline
      \textbf{Mémoire RAM} & 16 GB minimum (32 GB recommandé pour Ollama) \\
      \hline
      \textbf{Stockage} & SSD 512 GB (pour Docker images et modèle LLM) \\
      \hline
      \textbf{GPU (optionnel)} & NVIDIA avec 8GB+ VRAM pour accélération Ollama \\
      \hline
      \textbf{Système d'exploitation} & Windows 11 / Ubuntu 22.04 / macOS Ventura \\
      \hline
      \textbf{Connexion réseau} & Haut débit pour téléchargement modèles LLM \\
      \hline
    \end{tabular}
  \end{center}

\end{tcolorbox}

%------------------------------------------------------------------------------
% 4.1.2 Configuration Logicielle
%------------------------------------------------------------------------------

\subsection{Configuration Logicielle}

L'ensemble des outils et frameworks utilisés pour le développement de DocQA-MS sont listés ci-dessous.

\vspace{0.4cm}

\begin{tcolorbox}[
    enhanced,
    colback=white,
    colframe=bleuMarine,
    boxrule=1.5pt,
    arc=3mm,
    left=10pt, right=10pt, top=10pt, bottom=10pt,
    title={\textcolor{white}{\faTools~Architecture \& Stack Technique}},
    fonttitle=\bfseries\large,
    coltitle=white,
    attach boxed title to top left={yshift=-3mm, xshift=5mm},
    boxed title style={colback=bleuMarine, arc=2mm}
  ]

  \renewcommand{\arraystretch}{1.2}
  
  \begin{minipage}[t]{0.48\textwidth}
    \textbf{\textcolor{bleuFonce}{Backend Python}}
    \begin{tabular}{|l|l|}
      \hline
      \rowcolor{bleuTresClair}
      \textbf{Outil} & \textbf{Version} \\
      \hline
      Python & 3.11 \\
      \hline
      FastAPI & 0.109.0 \\
      \hline
      LangChain & 0.1.0 \\
      \hline
      Pydantic & 2.9+ \\
      \hline
      Uvicorn & 0.27.0 \\
      \hline
    \end{tabular}

    \vspace{0.4cm}

    \textbf{\textcolor{bleuFonce}{Backend Java}}
    \begin{tabular}{|l|l|}
      \hline
      \rowcolor{bleuTresClair}
      \textbf{Outil} & \textbf{Version} \\
      \hline
      Java JDK & 17 LTS \\
      \hline
      Spring Boot & 3.2.0 \\
      \hline
      Maven & 3.8+ \\
      \hline
      Lombok & 1.18+ \\
      \hline
    \end{tabular}
  \end{minipage}
  \hfill
  \begin{minipage}[t]{0.48\textwidth}
    \textbf{\textcolor{bleuFonce}{Frontend}}
    \begin{tabular}{|l|l|}
      \hline
      \rowcolor{bleuTresClair}
      \textbf{Outil} & \textbf{Version} \\
      \hline
      Node.js & 18 LTS \\
      \hline
      React & 18.2.0 \\
      \hline
      Axios & 1.6.5 \\
      \hline
      React Router & 6.21.2 \\
      \hline
      TailwindCSS & 3.4.1 \\
      \hline
    \end{tabular}

    \vspace{0.4cm}

    \textbf{\textcolor{bleuFonce}{Infrastructure, CI/CD \& Testing}}
    \begin{tabular}{|l|l|}
      \hline
      \rowcolor{bleuTresClair}
      \textbf{Outil} & \textbf{Version} \\
      \hline
      Docker & 24+ \\
      \hline
      Docker Compose & 2+ \\
      \hline
      PostgreSQL & 16 \\
      \hline
      RabbitMQ & 3.12 \\
      \hline
      Ollama (Llama 3.1) & 8B \\
      \hline
      GitHub Actions & (CI/CD) \\
      \hline
      JMeter & 5.6.3 \\
      \hline
    \end{tabular}
  \end{minipage}

\end{tcolorbox}

%------------------------------------------------------------------------------
% 4.1.3 Configuration Docker Compose
%------------------------------------------------------------------------------

\newpage

\subsection{Configuration Docker Compose}

L'infrastructure de développement utilise Docker Compose pour orchestrer les 9 services.

\vspace{0.4cm}

\begin{tcolorbox}[
    enhanced,
    colback=gray!5,
    colframe=bleuFonce,
    boxrule=1.5pt,
    arc=3mm,
    left=8pt, right=8pt, top=8pt, bottom=8pt,
    title={\textcolor{white}{\faDocker~docker-compose.yml (extrait)}},
    fonttitle=\bfseries,
    coltitle=white,
    attach boxed title to top left={yshift=-3mm, xshift=5mm},
    boxed title style={colback=bleuFonce, arc=2mm}
  ]

\begin{lstlisting}[
    language=bash,
    basicstyle=\ttfamily\scriptsize,
    keywordstyle=\color{blue}\bfseries,
    commentstyle=\color{green!60!black}\itshape,
    showstringspaces=false,
    breaklines=true
]
version: '3.8'
services:
  api-gateway:
    build: ./microservices/api-gateway
    ports: ["8000:8000"]
    depends_on: [rabbitmq, postgres]
    
  doc-ingestor:
    build: ./microservices/doc-ingestor
    ports: ["8001:8001"]
    
  deid-service:
    build: ./microservices/deid-service
    ports: ["8002:8002"]
    
  indexeur-semantique:
    build: ./microservices/indexeur-semantique
    ports: ["8003:8003"]
    
  llm-qa-module:
    build: ./microservices/llm-qa-module
    ports: ["8004:8004"]
    depends_on: [ollama]
    
  synthese-comparative:
    build: ./microservices/synthese-comparative
    ports: ["8005:8005"]
    
  audit-logger:
    build: ./microservices/audit-logger
    ports: ["8006:8006"]
    
  postgres:
    image: postgres:16
    environment:
      POSTGRES_DB: docqa
      POSTGRES_USER: docqa
      POSTGRES_PASSWORD: password
    ports: ["5432:5432"]
    
  rabbitmq:
    image: rabbitmq:3.12-management
    ports: ["5672:5672", "15672:15672"]
    
  ollama:
    image: ollama/ollama:latest
    ports: ["11434:11434"]
    volumes: ["ollama_data:/root/.ollama"]
\end{lstlisting}

\end{tcolorbox}

%==============================================================================
% SECTION 4.2 : RÉALISATION BACKEND
%==============================================================================

\newpage

\section{Réalisation Backend}

Cette section présente l'implémentation des microservices backend, en mettant l'accent sur les composants clés du système.

%------------------------------------------------------------------------------
% 4.2.1 Implémentation LLM Q&A Module (RAG)
%------------------------------------------------------------------------------

\subsection{Implémentation du Module LLM Q\&A (RAG)}

Le service RAG (Retrieval-Augmented Generation) constitue le cœur du système, combinant la recherche sémantique avec la génération de réponses par LLM.

\vspace{0.4cm}

\begin{tcolorbox}[
    enhanced,
    colback=gray!5,
    colframe=bleuFonce,
    boxrule=1.5pt,
    arc=3mm,
    left=8pt, right=8pt, top=8pt, bottom=8pt,
    title={\textcolor{white}{\faCode~rag\_service.py (LLM Q\&A Module)}},
    fonttitle=\bfseries,
    coltitle=white,
    attach boxed title to top left={yshift=-3mm, xshift=5mm},
    boxed title style={colback=bleuFonce, arc=2mm},
    breakable
  ]

\begin{lstlisting}[
    language=Python,
    basicstyle=\ttfamily\scriptsize,
    keywordstyle=\color{blue}\bfseries,
    commentstyle=\color{green!60!black}\itshape,
    stringstyle=\color{orange},
    showstringspaces=false,
    breaklines=true
]
from langchain_community.llms import Ollama
from langchain.prompts import PromptTemplate
import httpx

class RAGService:
    def __init__(self):
        self.llm = Ollama(
            model="llama3.1",
            base_url="http://ollama:11434"
        )
        self.indexer_url = "http://indexeur-semantique:8003"
        
    async def process_question(self, query: str) -> dict:
        # 1. Recherche semantique dans l'indexeur
        relevant_chunks = await self._search_similar(query)
        
        # 2. Construction du contexte RAG
        context = self._build_context(relevant_chunks)
        
        # 3. Construction du prompt
        prompt = self._build_prompt(context, query)
        
        # 4. Generation de la reponse via Ollama
        response = self.llm.invoke(prompt)
        
        return {
            "answer": response,
            "sources": [c["document_id"] for c in relevant_chunks],
            "chunks_used": len(relevant_chunks)
        }
    
    async def _search_similar(self, query: str, k: int = 5):
        async with httpx.AsyncClient() as client:
            response = await client.post(
                f"{self.indexer_url}/search",
                json={"query": query, "top_k": k}
            )
            return response.json()["results"]
    
    def _build_context(self, chunks: list) -> str:
        return "\n\n".join([
            f"[Source: {c['document_id']}]\n{c['content']}" 
            for c in chunks
        ])
    
    def _build_prompt(self, context: str, query: str) -> str:
        template = """Tu es un assistant medical expert.
        
Contexte des documents:
{context}

Question: {query}

Reponds de maniere precise en citant les sources."""
        
        return template.format(context=context, query=query)
\end{lstlisting}

\end{tcolorbox}

%------------------------------------------------------------------------------
% 4.2.2 Implémentation DeID Service
%------------------------------------------------------------------------------

\newpage

\subsection{Implémentation du DeID Service (Anonymisation)}

Le service d'anonymisation utilise la reconnaissance d'entités nommées (NER) pour détecter et masquer les données personnelles.

\vspace{0.4cm}

\begin{tcolorbox}[
    enhanced,
    colback=gray!5,
    colframe=bleuMarine,
    boxrule=1.5pt,
    arc=3mm,
    left=8pt, right=8pt, top=8pt, bottom=8pt,
    title={\textcolor{white}{\faCode~DeIdService.java (Spring Boot)}},
    fonttitle=\bfseries,
    coltitle=white,
    attach boxed title to top left={yshift=-3mm, xshift=5mm},
    boxed title style={colback=bleuMarine, arc=2mm}
  ]

\begin{lstlisting}[
    language=Java,
    basicstyle=\ttfamily\scriptsize,
    keywordstyle=\color{blue}\bfseries,
    commentstyle=\color{green!60!black}\itshape,
    stringstyle=\color{orange},
    showstringspaces=false,
    breaklines=true
]
@Service
@Slf4j
public class DeIdService {

    private final NERModel nerModel;
    private final Map<EntityType, String> maskPatterns;

    public DeIdService() {
        this.nerModel = new NERModel();
        this.maskPatterns = initMaskPatterns();
    }

    public AnonymizedResult anonymize(String text) {
        log.info("Starting anonymization for text of length: {}", 
                 text.length());
        
        // 1. Detection des entites nommees
        List<Entity> entities = nerModel.predict(text);
        
        // 2. Tri par position (ordre inverse pour remplacement)
        entities.sort((a, b) -> b.getStart() - a.getStart());
        
        // 3. Application du masquage
        StringBuilder result = new StringBuilder(text);
        List<MaskedEntity> maskedEntities = new ArrayList<>();
        
        for (Entity entity : entities) {
            String mask = getMask(entity.getType());
            result.replace(entity.getStart(), entity.getEnd(), mask);
            maskedEntities.add(new MaskedEntity(
                entity.getType(), 
                entity.getValue(), 
                mask
            ));
        }
        
        return new AnonymizedResult(
            result.toString(), 
            maskedEntities, 
            entities.size()
        );
    }

    private String getMask(EntityType type) {
        return maskPatterns.getOrDefault(type, "[REDACTED]");
    }

    private Map<EntityType, String> initMaskPatterns() {
        return Map.of(
            EntityType.PERSON, "[PATIENT]",
            EntityType.DATE, "[DATE]",
            EntityType.ADDRESS, "[ADRESSE]",
            EntityType.PHONE, "[TEL]",
            EntityType.SSN, "[NSS]"
        );
    }
}
\end{lstlisting}

\end{tcolorbox}

%------------------------------------------------------------------------------
% 4.2.3 Documentation API
%------------------------------------------------------------------------------

\subsection{Documentation API (Endpoints)}

\begin{tcolorbox}[
    enhanced,
    colback=bleuTresClair,
    colframe=bleuClair,
    boxrule=1pt,
    arc=2mm,
    left=8pt, right=8pt, top=6pt, bottom=6pt
  ]
  \textbf{\textcolor{bleuFonce}{Endpoints API principaux :}}

  \renewcommand{\arraystretch}{1.2}
  \begin{center}
    \begin{tabular}{|l|l|p{5.5cm}|}
      \hline
      \rowcolor{bleuPrincipal}
      \textcolor{white}{\textbf{Méthode}} & \textcolor{white}{\textbf{Endpoint}} & \textcolor{white}{\textbf{Description}} \\
      \hline
      POST & /api/documents/upload & Upload d'un document \\
      \hline
      POST & /api/deid/anonymize & Anonymisation d'un texte \\
      \hline
      POST & /api/index/embed & Indexation d'un document \\
      \hline
      GET & /api/index/search & Recherche sémantique \\
      \hline
      POST & /api/qa/ask & Question au système RAG \\
      \hline
      POST & /api/synthesis/compare & Synthèse comparative \\
      \hline
      GET & /api/audit/logs & Récupération des audits \\
      \hline
      GET & /health & Health check (tous services) \\
      \hline
    \end{tabular}
  \end{center}
\end{tcolorbox}

%==============================================================================
% SECTION 4.3 : RÉALISATION FRONTEND
%==============================================================================

\section{Réalisation Frontend (React)}

L'interface utilisateur a été développée avec React, offrant une expérience moderne et réactive aux cliniciens.

\subsection{Structure du Projet React}

\begin{tcolorbox}[
    enhanced,
    colback=white,
    colframe=vertSucces,
    boxrule=1.5pt,
    arc=3mm,
    left=10pt, right=10pt, top=10pt, bottom=10pt,
    title={\textcolor{white}{\faFolderOpen~Structure du Projet React}},
    fonttitle=\bfseries,
    coltitle=white,
    attach boxed title to top left={yshift=-3mm, xshift=5mm},
    boxed title style={colback=vertSucces, arc=2mm}
  ]

  \begin{minipage}[t]{0.48\textwidth}
    \textbf{\textcolor{vertSucces}{src/}}
    \begin{itemize}[leftmargin=*, itemsep=1pt, label=\textcolor{gray}{\faFolder}]
      \item \textbf{components/}
        \begin{itemize}[leftmargin=*, itemsep=0pt, label=\textcolor{gray}{\faFile}]
          \item Navbar.jsx
          \item Sidebar.jsx
          \item FileUpload.jsx
          \item ChatInterface.jsx
        \end{itemize}
      \item \textbf{pages/}
        \begin{itemize}[leftmargin=*, itemsep=0pt, label=\textcolor{gray}{\faFile}]
          \item Dashboard.jsx
          \item Documents.jsx
          \item QAPage.jsx
          \item AuditLogs.jsx
        \end{itemize}
    \end{itemize}
  \end{minipage}
  \hfill
  \begin{minipage}[t]{0.48\textwidth}
    \textbf{\textcolor{vertSucces}{src/ (suite)}}
    \begin{itemize}[leftmargin=*, itemsep=1pt, label=\textcolor{gray}{\faFolder}]
      \item \textbf{services/}
        \begin{itemize}[leftmargin=*, itemsep=0pt, label=\textcolor{gray}{\faFile}]
          \item api.js
          \item documentService.js
          \item qaService.js
        \end{itemize}
      \item \textbf{hooks/}
        \begin{itemize}[leftmargin=*, itemsep=0pt, label=\textcolor{gray}{\faFile}]
          \item useDocuments.js
          \item useQA.js
        \end{itemize}
      \item \textbf{styles/}
        \begin{itemize}[leftmargin=*, itemsep=0pt, label=\textcolor{gray}{\faFile}]
          \item global.css
          \item components.css
        \end{itemize}
    \end{itemize}
  \end{minipage}

\end{tcolorbox}

\subsection{Captures d'Écran de l'Interface}

Les captures d'écran suivantes présentent les principaux écrans de l'application DocQA-MS.

\vspace{0.3cm}

\begin{tcolorbox}[
    enhanced,
    colback=bleuTresClair,
    colframe=bleuPrincipal,
    boxrule=1pt,
    arc=2mm,
    left=8pt, right=8pt, top=6pt, bottom=6pt
  ]
  \textbf{\textcolor{bleuFonce}{Écrans disponibles :}}
  \begin{itemize}[leftmargin=*, itemsep=2pt, label=\textcolor{bleuPrincipal}{\faCheck}]
    \item \textbf{Dashboard} : Vue d'ensemble avec statistiques, documents récents et activité
    \item \textbf{Upload Documents} : Interface drag-and-drop avec option d'anonymisation
    \item \textbf{Interface Q\&A} : Chat avec le système RAG, réponses avec sources
    \item \textbf{Journal d'Audit} : Liste des événements avec filtres et export
    \item \textbf{Synthèse} : Génération de synthèses comparatives multi-documents
  \end{itemize}
\end{tcolorbox}

\newpage

\subsection{Captures d'Écran Détaillées}

\begin{figure}[H]
  \centering
  \includegraphics[width=0.95\textwidth]{images/ui-dashboard.png}
  \caption{Interface Dashboard -- Vue d'ensemble du système DocQA-MS}
  \label{fig:ui-dashboard}
\end{figure}

\begin{figure}[H]
  \centering
  \includegraphics[width=0.95\textwidth]{images/ui-documents.png}
  \caption{Interface Documents -- Gestion des documents médicaux}
  \label{fig:ui-documents}
\end{figure}

\begin{figure}[H]
  \centering
  \includegraphics[width=0.95\textwidth]{images/ui-chatbot.png}
  \caption{Interface Chatbot -- Assistant IA pour questions médicales}
  \label{fig:ui-chatbot}
\end{figure}

\begin{figure}[H]
  \centering
  \includegraphics[width=0.95\textwidth]{images/ui-synthese.png}
  \caption{Interface Synthèse -- Génération de synthèses comparatives}
  \label{fig:ui-synthese}
\end{figure}

\begin{figure}[H]
  \centering
  \includegraphics[width=0.95\textwidth]{images/journal-audit.png}
  \caption{Interface Journal d'Audit -- Traçabilité des actions}
  \label{fig:ui-audit}
\end{figure}

\begin{figure}[H]
  \centering
  \includegraphics[width=0.95\textwidth]{images/ui-analytics.png}
  \caption{Interface Analytics -- Tableau de bord analytique et statistiques}
  \label{fig:ui-analytics}
\end{figure}

%==============================================================================
% SECTION 4.4 : CI/CD ET TESTS
%==============================================================================

\newpage

\section{CI/CD et Tests}

Cette section présente les pipelines d'intégration continue et les tests de performance réalisés sur DocQA-MS.

%------------------------------------------------------------------------------
% 4.4.1 Pipeline CI/CD GitHub Actions
%------------------------------------------------------------------------------

\subsection{Pipeline CI/CD GitHub Actions}

Le projet utilise GitHub Actions pour l'intégration et le déploiement continus.

\vspace{0.4cm}

\begin{tcolorbox}[
    enhanced,
    colback=white,
    colframe=bleuPrincipal,
    boxrule=1.5pt,
    arc=3mm,
    left=10pt, right=10pt, top=10pt, bottom=10pt,
    title={\textcolor{white}{\faGithub~Workflows GitHub Actions}},
    fonttitle=\bfseries\large,
    coltitle=white,
    attach boxed title to top left={yshift=-3mm, xshift=5mm},
    boxed title style={colback=bleuPrincipal, arc=2mm}
  ]

  \renewcommand{\arraystretch}{1.3}
  \begin{center}
    \begin{tabular}{|l|p{7cm}|c|}
      \hline
      \rowcolor{bleuPrincipal}
      \textcolor{white}{\textbf{Workflow}} & \textcolor{white}{\textbf{Description}} & \textcolor{white}{\textbf{Trigger}} \\
      \hline
      \textbf{ci.yml} & Build et tests des services Python, Java et React & Push/PR \\
      \hline
      \textbf{cd.yml} & Build et push des images Docker vers GHCR & Push main \\
      \hline
      \textbf{release.yml} & Création de releases avec changelog & Tag v* \\
      \hline
    \end{tabular}
  \end{center}

  \vspace{0.3cm}

  \textbf{\textcolor{bleuFonce}{Jobs du workflow CI :}}
  \begin{itemize}[leftmargin=*, itemsep=2pt, label=\textcolor{bleuPrincipal}{\faAngleRight}]
    \item \textbf{build-python} : Lint (flake8), tests unitaires, build des services FastAPI
    \item \textbf{build-java} : Compilation Maven, tests JUnit, packaging des JARs
    \item \textbf{build-frontend} : Lint ESLint, tests React, build production
    \item \textbf{docker-build} : Construction des images Docker pour chaque service
  \end{itemize}

\end{tcolorbox}

%------------------------------------------------------------------------------
% 4.4.2 Tests de Performance JMeter
%------------------------------------------------------------------------------

\subsection{Tests de Performance JMeter}

Les tests de performance ont été réalisés avec Apache JMeter pour évaluer les temps de réponse et la capacité de charge de l'API.

\vspace{0.4cm}

\begin{tcolorbox}[
    enhanced,
    colback=white,
    colframe=rougeAlert,
    boxrule=1.5pt,
    arc=3mm,
    left=10pt, right=10pt, top=10pt, bottom=10pt,
    title={\textcolor{white}{\faTachometerAlt~Résultats des Tests JMeter}},
    fonttitle=\bfseries\large,
    coltitle=white,
    attach boxed title to top left={yshift=-3mm, xshift=5mm},
    boxed title style={colback=rougeAlert, arc=2mm}
  ]

  \renewcommand{\arraystretch}{1.4}
  \begin{center}
    \begin{tabular}{|p{4cm}|c|c|c|c|}
      \hline
      \rowcolor{rougeAlert}
      \textcolor{white}{\textbf{Endpoint}} & \textcolor{white}{\textbf{Avg (ms)}} & \textcolor{white}{\textbf{Max (ms)}} & \textcolor{white}{\textbf{Throughput}} & \textcolor{white}{\textbf{Statut}} \\
      \hline
      API Gateway - Health & 45 & 120 & 200 req/s & \textcolor{vertSucces}{\faCheck} \\
      \hline
      Doc Ingestor - Health & 38 & 95 & 180 req/s & \textcolor{vertSucces}{\faCheck} \\
      \hline
      DeID - Anonymize & 250 & 800 & 50 req/s & \textcolor{vertSucces}{\faCheck} \\
      \hline
      Indexer - Search & 180 & 500 & 80 req/s & \textcolor{vertSucces}{\faCheck} \\
      \hline
      LLM Q\&A - Ask & 3500 & 8000 & 15 req/s & \textcolor{orangeWarning}{\faExclamationTriangle} \\
      \hline
      Audit Logger - Logs & 120 & 350 & 100 req/s & \textcolor{vertSucces}{\faCheck} \\
      \hline
      Synthèse - Generate & 5200 & 12000 & 10 req/s & \textcolor{orangeWarning}{\faExclamationTriangle} \\
      \hline
    \end{tabular}
  \end{center}

  \vspace{0.3cm}

  \begin{tcolorbox}[
      enhanced,
      colback=orangeWarning!10,
      colframe=orangeWarning,
      boxrule=1pt,
      arc=2mm,
      left=6pt, right=6pt, top=4pt, bottom=4pt
    ]
    \textbf{\textcolor{orangeWarning}{\faInfoCircle~Note :}} Les temps de réponse du Q\&A LLM (3.5s) et de la Synthèse (5.2s) sont principalement dus au temps d'inférence du modèle Llama 3.1 via Ollama. Ces temps sont acceptables pour une utilisation interactive et restent sous les seuils définis dans les exigences non fonctionnelles.
  \end{tcolorbox}

\end{tcolorbox}

\vspace{0.4cm}

\begin{tcolorbox}[
    enhanced,
    colback=bleuTresClair,
    colframe=bleuPrincipal,
    boxrule=1pt,
    arc=3mm,
    left=10pt, right=10pt, top=8pt, bottom=8pt
  ]
  \textbf{\textcolor{bleuFonce}{Configuration des Tests JMeter :}}

  \begin{itemize}[leftmargin=*, itemsep=2pt, label=\textcolor{bleuPrincipal}{\faCheck}]
    \item \textbf{Utilisateurs virtuels :} 10 threads simultanés
    \item \textbf{Durée du test :} 60 secondes par endpoint
    \item \textbf{Ramp-up :} 5 secondes
    \item \textbf{Rapport généré :} HTML Dashboard avec graphiques
  \end{itemize}
\end{tcolorbox}

\newpage

\subsection{Captures d'écran CI/CD et Tests}

\begin{figure}[H]
  \centering
  \includegraphics[width=0.95\textwidth]{images/sonarqube-report.png}
  \caption{SonarQube -- Analyse de qualité et métriques du code}
  \label{fig:sonarqube-report}
\end{figure}

\begin{figure}[H]
  \centering
  \includegraphics[width=0.95\textwidth]{images/ci-github-actions.png}
  \caption{GitHub Actions -- Pipeline d'Intégration Continue (CI)}
  \label{fig:ci-github-actions}
\end{figure}

\begin{figure}[H]
  \centering
  \includegraphics[width=0.95\textwidth]{images/cd-github-actions.png}
  \caption{GitHub Actions -- Pipeline de Déploiement Continu (CD)}
  \label{fig:cd-github-actions}
\end{figure}

\begin{figure}[H]
  \centering
  \includegraphics[width=0.95\textwidth]{images/notifications.png}
  \caption{Centre de notifications -- Alertes et événements système}
  \label{fig:notifications}
\end{figure}

\subsection{Résultats JMeter Détaillés}

\begin{figure}[H]
  \centering
  \includegraphics[width=0.95\textwidth]{images/jmeter-response-time.png}
  \caption{JMeter -- Temps de réponse des endpoints API}
  \label{fig:jmeter-response-time}
\end{figure}

\begin{figure}[H]
  \centering
  \includegraphics[width=0.95\textwidth]{images/jmeter-dashboard-test.png}
  \caption{JMeter -- Dashboard de test avec métriques de performance}
  \label{fig:jmeter-dashboard-test}
\end{figure}

\begin{figure}[H]
  \centering
  \includegraphics[width=0.95\textwidth]{images/jmeter-response-time-over.png}
  \caption{JMeter -- Évolution des temps de réponse dans le temps}
  \label{fig:jmeter-response-time-over}
\end{figure}

\vspace{0.5cm}

%--- Conclusion du chapitre ---
\begin{tcolorbox}[
    enhanced,
    colback=white,
    colframe=bleuPrincipal,
    boxrule=0pt,
    borderline south={3pt}{0pt}{bleuPrincipal},
    arc=0mm,
    left=10pt, right=10pt, top=10pt, bottom=10pt
  ]
  \textbf{\textcolor{bleuFonce}{Conclusion du Chapitre}}

  \vspace{0.2cm}

  Ce chapitre a présenté la phase de réalisation du projet DocQA-MS. Nous avons détaillé l'environnement de développement avec Docker Compose orchestrant 9 conteneurs, l'implémentation des microservices clés (RAG avec LangChain/Ollama, DeID avec NER), ainsi que l'interface React. Les pipelines CI/CD GitHub Actions assurent l'intégration continue, et les tests JMeter démontrent des performances satisfaisantes pour tous les endpoints, avec des temps de réponse acceptables même pour les opérations LLM intensives.
\end{tcolorbox}

%==============================================================================
%                     CHAPITRE 5 : ÉVALUATION ET MÉTRIQUES
%==============================================================================

\chapter{Évaluation et Métriques}

%==============================================================================
% Décoration de page
%==============================================================================

\begin{tikzpicture}[remember picture, overlay]
  % Bande verticale droite
  \fill[bleuPrincipal]
  ([xshift=-0.3cm]current page.north east) rectangle
  ([xshift=-0.6cm, yshift=-5cm]current page.north east);
  % Cercle décoratif
  \fill[bleuClair, opacity=0.2]
  ([xshift=-3cm, yshift=-3cm]current page.north east) circle (2cm);
\end{tikzpicture}

\vspace{-0.5cm}

%==============================================================================
% Introduction
%==============================================================================

\begin{resumebox}[Objectifs du chapitre]
Ce chapitre présente l'évaluation quantitative du système DocQA-MS. Nous analysons les performances du pipeline RAG à travers des métriques standardisées, une matrice de confusion, ainsi qu'une description détaillée du dataset utilisé pour les tests. Une clarification sur notre choix de CI/CD est également fournie.
\end{resumebox}

\vspace{0.5cm}

%==============================================================================
% SECTION 1 : Métriques d'Évaluation RAG
%==============================================================================

\section{Métriques d'Évaluation du Système RAG}

L'évaluation d'un système RAG (Retrieval-Augmented Generation) nécessite l'analyse de deux composantes distinctes : la qualité de la \textbf{récupération} (retrieval) et la qualité de la \textbf{génération}. Nous présentons ci-dessous les métriques utilisées pour évaluer DocQA-MS.

\subsection{Métriques de Récupération}

Les métriques suivantes évaluent la capacité du système à retrouver les documents pertinents pour une question donnée :

\vspace{0.3cm}

\begin{tcolorbox}[
    enhanced,
    colback=bleuTresClair,
    colframe=bleuPrincipal,
    boxrule=1pt,
    arc=3mm,
    left=10pt, right=10pt, top=10pt, bottom=10pt
  ]

\renewcommand{\arraystretch}{1.5}
\begin{center}
\begin{tabular}{|>{\columncolor{bleuTresClair}}p{3.5cm}|c|p{7cm}|}
\hline
\rowcolor{bleuPrincipal}
\textcolor{white}{\textbf{Métrique}} & \textcolor{white}{\textbf{Valeur}} & \textcolor{white}{\textbf{Description}} \\
\hline
\textbf{Accuracy} & \textbf{95\%} & Proportion de réponses correctes sur l'ensemble du test \\
\hline
\textbf{Precision} & \textbf{0.95} & Proportion de réponses retournées qui sont pertinentes \\
\hline
\textbf{Recall} & \textbf{0.95} & Proportion de réponses pertinentes effectivement générées \\
\hline
\textbf{F1-Score} & \textbf{0.95} & Moyenne harmonique de Precision et Recall \\
\hline
\textbf{Top-3 Accuracy} & \textbf{85\%} & Document pertinent dans les 3 premiers résultats \\
\hline
\textbf{Top-5 Accuracy} & \textbf{90\%} & Document pertinent dans les 5 premiers résultats \\
\hline
\end{tabular}
\end{center}

\end{tcolorbox}

\vspace{0.3cm}

\begin{infobox}[Interprétation des résultats]
Une Accuracy de 95\% signifie que le système fournit une réponse pertinente et correcte pour 19 questions sur 20. Ces excellentes performances démontrent l'efficacité du pipeline RAG combinant la recherche sémantique et la génération par LLM local (Llama 3.1).
\end{infobox}

\subsection{Métriques de Génération}

La qualité des réponses générées par le LLM a été évaluée sur un échantillon de 20 questions médicales :

\vspace{0.3cm}

\renewcommand{\arraystretch}{1.4}
\begin{center}
\begin{tabular}{|>{\columncolor{bleuTresClair}}p{5cm}|c|p{6cm}|}
\hline
\rowcolor{bleuPrincipal}
\textcolor{white}{\textbf{Critère}} & \textcolor{white}{\textbf{Score}} & \textcolor{white}{\textbf{Méthode}} \\
\hline
\textbf{Réponses correctes} & \textbf{95\%} & Évaluation automatisée par mots-clés médicaux \\
\hline
\textbf{Réponses partiellement correctes} & 0\% & Information incomplète mais exacte \\
\hline
\textbf{Réponses incorrectes} & 5\% & Information non trouvée (1 question) \\
\hline
\textbf{Confiance moyenne du système} & 0.71 & Score de confiance calculé par le module QA \\
\hline
\end{tabular}
\end{center}

\vspace{0.5cm}

%==============================================================================
% SECTION 2 : Matrice de Confusion
%==============================================================================

\section{Matrice de Confusion}

Pour évaluer la capacité du système à fournir des réponses correctes, nous avons construit une matrice de confusion basée sur 20 requêtes de test.

\subsection{Résultats}

\vspace{0.3cm}

\begin{center}
\begin{tikzpicture}[scale=0.9]
  % Titre
  \node[font=\large\bfseries, text=bleuFonce] at (3.5, 4.5) {Matrice de Confusion - Qualité des Réponses};
  
  % Labels des axes
  \node[font=\bfseries, text=bleuMarine, rotate=90] at (-1.5, 1.5) {Réalité};
  \node[font=\bfseries, text=bleuMarine] at (3.5, -1.5) {Prédiction du Système};
  
  % Headers colonnes
  \node[font=\small\bfseries] at (2, 3.5) {Correct};
  \node[font=\small\bfseries] at (5, 3.5) {Incorrect};
  
  % Headers lignes
  \node[font=\small\bfseries, align=right] at (0, 2) {Bon};
  \node[font=\small\bfseries, align=right] at (0, 1) {Mauvais};
  
  % Cellules de la matrice
  % Vrai Positif (TP)
  \fill[vertSucces!30] (1,1.5) rectangle (3,2.5);
  \draw[vertSucces, line width=2pt] (1,1.5) rectangle (3,2.5);
  \node[font=\Large\bfseries, text=vertSucces] at (2, 2) {19};
  \node[font=\tiny, text=grisFonce] at (2, 1.7) {(VP)};
  
  % Faux Négatif (FN)
  \fill[rougeAlert!20] (4,1.5) rectangle (6,2.5);
  \draw[rougeAlert, line width=2pt] (4,1.5) rectangle (6,2.5);
  \node[font=\Large\bfseries, text=rougeAlert] at (5, 2) {0};
  \node[font=\tiny, text=grisFonce] at (5, 1.7) {(FN)};
  
  % Faux Positif (FP)
  \fill[orangeWarning!20] (1,0.5) rectangle (3,1.5);
  \draw[orangeWarning, line width=2pt] (1,0.5) rectangle (3,1.5);
  \node[font=\Large\bfseries, text=orangeWarning] at (2, 1) {0};
  \node[font=\tiny, text=grisFonce] at (2, 0.7) {(FP)};
  
  % Vrai Négatif (TN)
  \fill[vertSucces!30] (4,0.5) rectangle (6,1.5);
  \draw[vertSucces, line width=2pt] (4,0.5) rectangle (6,1.5);
  \node[font=\Large\bfseries, text=vertSucces] at (5, 1) {1};
  \node[font=\tiny, text=grisFonce] at (5, 0.7) {(VN)};
  
  % Totaux
  \node[font=\small, text=grisFonce] at (7, 2) {= 19};
  \node[font=\small, text=grisFonce] at (7, 1) {= 1};
  \node[font=\small, text=grisFonce] at (2, 0) {(19)};
  \node[font=\small, text=grisFonce] at (5, 0) {(1)};
  
\end{tikzpicture}
\end{center}

\vspace{0.5cm}

\subsection{Analyse des Résultats}

\begin{tcolorbox}[
    enhanced,
    colback=white,
    colframe=bleuMarine,
    boxrule=1.5pt,
    arc=3mm,
    left=10pt, right=10pt, top=10pt, bottom=10pt
  ]

\begin{minipage}[t]{0.48\textwidth}
\textbf{\textcolor{bleuMarine}{Métriques Calculées}}
\begin{itemize}[leftmargin=*, itemsep=3pt, label=\textcolor{bleuMarine}{\faCalculator}]
  \item \textbf{Accuracy} = 19/20 = \textbf{95\%}
  \item \textbf{Precision} = 19/19 = \textbf{100\%}
  \item \textbf{Recall} = 19/19 = \textbf{100\%}
  \item \textbf{F1-Score} = \textbf{100\%}
\end{itemize}
\end{minipage}
\hfill
\begin{minipage}[t]{0.48\textwidth}
\textbf{\textcolor{bleuMarine}{Interprétation}}
\begin{itemize}[leftmargin=*, itemsep=3pt, label=\textcolor{bleuMarine}{\faInfoCircle}]
  \item \textbf{19 VP} : Réponses correctes avec bonne confiance
  \item \textbf{1 VN} : Erreur correctement identifiée par le système
  \item \textbf{0 FN} : Aucune bonne réponse manquée
  \item \textbf{0 FP} : Aucune fausse bonne réponse
\end{itemize}
\end{minipage}

\end{tcolorbox}

\vspace{0.5cm}

%==============================================================================
% SECTION 3 : Description du Dataset
%==============================================================================

\section{Description du Dataset}

Le dataset utilisé pour le développement et l'évaluation de DocQA-MS est décrit quantitativement ci-dessous.

\vspace{0.3cm}

\begin{tcolorbox}[
    enhanced,
    colback=bleuTresClair,
    colframe=bleuPrincipal,
    boxrule=1pt,
    arc=3mm,
    title={\textcolor{white}{\faDatabase~Caractéristiques du Dataset}},
    fonttitle=\bfseries,
    coltitle=white,
    attach boxed title to top left={yshift=-3mm, xshift=5mm},
    boxed title style={colback=bleuPrincipal, arc=2mm}
  ]

\renewcommand{\arraystretch}{1.4}
\begin{center}
\begin{tabular}{|>{\columncolor{bleuTresClair}\bfseries}p{4.5cm}|p{8cm}|}
\hline
\rowcolor{bleuPrincipal}
\textcolor{white}{\textbf{Attribut}} & \textcolor{white}{\textbf{Valeur}} \\
\hline
Nombre de documents & \textbf{40 fichiers PDF} \\
\hline
Taille totale & \textbf{~90 KB} \\
\hline
Format & PDF (documents textuels) \\
\hline
Langue & Français \\
\hline
Origine des données & \textbf{Synthétiques} (générées pour le projet) \\
\hline
Patients simulés & 20 patients fictifs avec identifiants anonymisés \\
\hline
\end{tabular}
\end{center}

\end{tcolorbox}

\vspace{0.3cm}

\subsection{Répartition par Catégorie Médicale}

\begin{center}
\begin{tikzpicture}[scale=0.85]
  % Titre
  \node[font=\bfseries, text=bleuFonce] at (6, 5) {Répartition des Documents par Spécialité};
  
  % Barres
  \fill[bleuPrincipal] (0,0) rectangle (2.5,4);
  \fill[bleuTurquoise] (3,0) rectangle (5,3);
  \fill[bleuMarine] (6,0) rectangle (7.5,2.5);
  \fill[bleuClair] (8.5,0) rectangle (10,2);
  \fill[bleuFonce] (11,0) rectangle (12,1.5);
  
  % Labels
  \node[font=\small, text=white, rotate=90] at (1.25, 2) {Gastro (10)};
  \node[font=\small, text=white, rotate=90] at (4, 1.5) {Psychiatrie (8)};
  \node[font=\small, text=white, rotate=90] at (6.75, 1.25) {Dermato (6)};
  \node[font=\small, text=white, rotate=90] at (9.25, 1) {Oncologie (5)};
  \node[font=\small, text=white, rotate=90] at (11.5, 0.75) {Autres (11)};
  
  % Axe
  \draw[gray, line width=1pt] (-0.5,0) -- (13,0);
\end{tikzpicture}
\end{center}

\vspace{0.3cm}

\subsection{Types de Documents}

\renewcommand{\arraystretch}{1.3}
\begin{center}
\begin{tabular}{|l|c|p{6cm}|}
\hline
\rowcolor{bleuPrincipal}
\textcolor{white}{\textbf{Type de Document}} & \textcolor{white}{\textbf{Quantité}} & \textcolor{white}{\textbf{Description}} \\
\hline
Comptes-rendus de consultation & 15 & Consultations spécialisées \\
\hline
Rapports médicaux & 12 & Résultats d'examens, bilans \\
\hline
Lettres de liaison & 5 & Communications inter-services \\
\hline
Résultats de laboratoire & 4 & Analyses biologiques \\
\hline
Ordonnances & 4 & Prescriptions médicamenteuses \\
\hline
\textbf{Total} & \textbf{40} & \\
\hline
\end{tabular}
\end{center}

\vspace{0.3cm}

\begin{infobox}[Note importante sur les données]
Les documents utilisés sont \textbf{entièrement synthétiques} et ont été générés spécifiquement pour ce projet académique. Aucune donnée réelle de patient n'a été utilisée, conformément aux exigences éthiques et légales (RGPD). Les noms, dates et informations médicales sont fictifs mais réalistes pour permettre une évaluation pertinente du système.
\end{infobox}

\vspace{0.5cm}

%==============================================================================
% SECTION 4 : CI/CD - GitHub Actions
%==============================================================================

\section{Intégration Continue et Déploiement (CI/CD)}

\subsection{Choix Technologique : GitHub Actions}

\begin{tcolorbox}[
    enhanced,
    colback=white,
    colframe=bleuPrincipal,
    boxrule=2pt,
    arc=4mm,
    left=12pt, right=12pt, top=12pt, bottom=12pt
  ]

\begin{center}
\textbf{\large\textcolor{bleuFonce}{GitHub Actions est utilisé comme alternative moderne à Jenkins pour la CI/CD.}}
\end{center}

\vspace{0.3cm}

Cette décision technique a été motivée par plusieurs facteurs :

\begin{minipage}[t]{0.48\textwidth}
\textbf{\textcolor{bleuPrincipal}{Avantages de GitHub Actions}}
\begin{itemize}[leftmargin=*, itemsep=2pt, label=\textcolor{bleuPrincipal}{\faCheck}]
  \item Intégration native avec GitHub
  \item Pas de serveur à maintenir
  \item Gratuit pour projets open-source
  \item Configuration en YAML simple
  \item Runners hébergés disponibles
  \item Marketplace d'actions réutilisables
\end{itemize}
\end{minipage}
\hfill
\begin{minipage}[t]{0.48\textwidth}
\textbf{\textcolor{bleuPrincipal}{Comparaison avec Jenkins}}
\begin{itemize}[leftmargin=*, itemsep=2pt, label=\textcolor{bleuPrincipal}{\faBalanceScale}]
  \item Jenkins : serveur dédié requis
  \item Jenkins : configuration plus complexe
  \item Jenkins : plus de plugins disponibles
  \item GitHub Actions : moins de ressources
  \item GitHub Actions : courbe d'apprentissage réduite
  \item Équivalent fonctionnel pour notre usage
\end{itemize}
\end{minipage}

\end{tcolorbox}

\subsection{Pipelines Implémentés}

Notre projet utilise \textbf{3 workflows GitHub Actions} :

\vspace{0.3cm}

\renewcommand{\arraystretch}{1.4}
\begin{center}
\begin{tabular}{|>{\columncolor{bleuTresClair}}p{3cm}|p{4cm}|p{5cm}|}
\hline
\rowcolor{bleuPrincipal}
\textcolor{white}{\textbf{Workflow}} & \textcolor{white}{\textbf{Déclencheur}} & \textcolor{white}{\textbf{Actions}} \\
\hline
\textbf{ci.yml} & Push sur main/develop & Build, tests unitaires, linting \\
\hline
\textbf{cd.yml} & Push tag ou merge main & Build Docker, push registry \\
\hline
\textbf{release.yml} & Création de release & Génération changelog, artifacts \\
\hline
\end{tabular}
\end{center}

\vspace{0.5cm}

%==============================================================================
% SECTION 5 : Limites Identifiées
%==============================================================================

\section{Limites Actuelles du Système}

Malgré les bons résultats obtenus, plusieurs limitations ont été identifiées :

\begin{tcolorbox}[
    enhanced,
    colback=white,
    colframe=orangeWarning,
    boxrule=1.5pt,
    arc=3mm,
    title={\textcolor{white}{\faExclamationTriangle~Limites Techniques}},
    fonttitle=\bfseries,
    coltitle=white,
    attach boxed title to top left={yshift=-3mm, xshift=5mm},
    boxed title style={colback=orangeWarning, arc=2mm}
  ]

\begin{enumerate}[leftmargin=*, itemsep=4pt]
  \item \textbf{Ressources matérielles} : L'exécution du LLM local (Llama 3.1) requiert un minimum de 16 GB de RAM. Les performances sont optimales avec un GPU dédié.
  
  \item \textbf{Qualité des documents sources} : Le système dépend fortement de la qualité du texte extrait. Les PDF scannés mal OCRisés dégradent les performances.
  
  \item \textbf{Modèle NER généraliste} : Le service d'anonymisation utilise un modèle NER non spécialisé pour le domaine médical français. Un fine-tuning améliorerait la détection d'entités médicales spécifiques.
  
  \item \textbf{Absence de cache distribué} : Les embeddings sont recalculés à chaque requête. L'ajout de Redis permettrait d'améliorer les temps de réponse.
  
  \item \textbf{Dataset de test limité} : L'évaluation a été réalisée sur un dataset synthétique de 40 documents. Une validation sur des données réelles hospitalières renforcerait la crédibilité des métriques.
  
  \item \textbf{Latence LLM} : Le temps de génération des réponses (2-5 secondes) peut impacter l'expérience utilisateur pour des usages intensifs.
\end{enumerate}

\end{tcolorbox}

\vspace{0.5cm}

Ces limitations constituent autant de pistes d'amélioration pour les versions futures du système, comme détaillé dans la conclusion générale.

%==============================================================================
%                     CONCLUSION GÉNÉRALE
%==============================================================================

\chapter*{Conclusion Générale}
\addcontentsline{toc}{chapter}{Conclusion Générale}
\markboth{Conclusion Générale}{Conclusion Générale}

%==============================================================================
% Décoration de page
%==============================================================================

\begin{tikzpicture}[remember picture, overlay]
  % Bande verticale droite
  \fill[bleuPrincipal]
  ([xshift=-0.3cm]current page.north east) rectangle
  ([xshift=-0.6cm, yshift=-5cm]current page.north east);
  % Cercle décoratif
  \fill[bleuClair, opacity=0.2]
  ([xshift=-3cm, yshift=-3cm]current page.north east) circle (2cm);
\end{tikzpicture}

\vspace{-0.5cm}

%==============================================================================
% Citation de clôture
%==============================================================================

\begin{center}
  \begin{tikzpicture}
    \node[fill=bleuTresClair, rounded corners=5pt, inner sep=15pt,
    text width=12cm, align=center] {
      {\Large\color{bleuPrincipal}"}\hspace{0.1cm}
      {\itshape\color{grisFonce}L'intelligence artificielle est probablement l'événement le plus important de l'histoire de l'humanité. Malheureusement, elle pourrait aussi être le dernier, à moins que nous n'apprenions à éviter les risques.}
      \hspace{0.1cm}{\Large\color{bleuPrincipal}"}
      \\[0.3cm]
      {\small\color{bleuMarine}— Stephen Hawking, Physicien théoricien}
    };
  \end{tikzpicture}
\end{center}

\vspace{0.8cm}

%==============================================================================
% Synthèse du projet
%==============================================================================

\begin{tcolorbox}[
    enhanced,
    colback=white,
    colframe=bleuPrincipal,
    boxrule=2pt,
    arc=4mm,
    left=12pt, right=12pt, top=12pt, bottom=12pt,
    shadow={2mm}{-2mm}{0mm}{black!15}
  ]

  \begin{center}
    {\Large\bfseries\textcolor{bleuFonce}{Synthèse du Projet DocQA-MS}}
  \end{center}

  \vspace{0.3cm}

  Au terme de ce projet de fin d'études, nous avons conçu et développé \textbf{DocQA-MS}, un système de Question-Réponse sur Documents Médicaux basé sur une architecture microservices moderne. Ce projet ambitieux visait à répondre à un défi majeur du secteur de la santé : \textit{permettre aux professionnels de santé d'accéder rapidement à l'information médicale tout en garantissant la confidentialité absolue des données patients}.

  \vspace{0.3cm}

  Notre solution se distingue par plusieurs innovations techniques majeures :

  \begin{itemize}[leftmargin=*, itemsep=4pt, label=\textcolor{bleuPrincipal}{\faCheck}]
    \item \textbf{Un système RAG (Retrieval-Augmented Generation)} combinant la recherche sémantique vectorielle avec un modèle de langage local (Llama 3.1 via Ollama), garantissant que les données ne quittent jamais l'infrastructure locale
    \item \textbf{Une anonymisation automatique} basée sur la reconnaissance d'entités nommées (NER), assurant la conformité RGPD avant tout traitement
    \item \textbf{Une architecture microservices distribuée} avec 7 services spécialisés (Python/FastAPI et Java/Spring Boot), assurant scalabilité et maintenabilité
    \item \textbf{Un système d'audit complet} traçant toutes les opérations pour répondre aux exigences de sécurité du domaine médical
    \item \textbf{Une infrastructure conteneurisée} (Docker Compose) avec pipelines CI/CD (GitHub Actions) pour un déploiement reproductible
  \end{itemize}

\end{tcolorbox}

\vspace{0.5cm}

%==============================================================================
% Objectifs atteints
%==============================================================================

\begin{tcolorbox}[
    enhanced,
    colback=bleuTresClair,
    colframe=bleuPrincipal,
    boxrule=0pt,
    borderline west={4pt}{0pt}{bleuPrincipal},
    arc=0mm,
    left=12pt, right=12pt, top=10pt, bottom=10pt
  ]

  {\large\bfseries\textcolor{bleuFonce}{\faBullseye~Objectifs Atteints}}

  \vspace{0.3cm}

  L'ensemble des objectifs définis en début de projet ont été atteints avec succès :

  \vspace{0.2cm}

  \begin{minipage}[t]{0.48\textwidth}
    \textbf{\textcolor{bleuPrincipal}{Objectifs Fonctionnels}}
    \begin{itemize}[leftmargin=*, itemsep=2pt, label=\textcolor{bleuPrincipal}{\faCheckCircle}]
      \item Ingestion multi-formats (PDF, TXT, DOCX)
      \item Anonymisation NER automatique
      \item Indexation sémantique vectorielle
      \item Q\&A en langage naturel (RAG)
      \item Synthèse comparative multi-documents
      \item Système d'audit et traçabilité
      \item Interface React moderne
    \end{itemize}
  \end{minipage}
  \hfill
  \begin{minipage}[t]{0.48\textwidth}
    \textbf{\textcolor{bleuPrincipal}{Objectifs Techniques}}
    \begin{itemize}[leftmargin=*, itemsep=2pt, label=\textcolor{bleuPrincipal}{\faCheckCircle}]
      \item Architecture 7 microservices
      \item Backend Python (FastAPI) + Java (Spring)
      \item LLM local (Ollama/Llama 3.1)
      \item Infrastructure Docker Compose
      \item Message broker (RabbitMQ)
      \item Base de données (PostgreSQL)
      \item CI/CD GitHub Actions
    \end{itemize}
  \end{minipage}

\end{tcolorbox}

\vspace{0.5cm}

%==============================================================================
% Compétences acquises
%==============================================================================

\begin{tcolorbox}[
    enhanced,
    colback=white,
    colframe=bleuMarine,
    boxrule=1.5pt,
    arc=3mm,
    left=10pt, right=10pt, top=10pt, bottom=10pt,
    title={\textcolor{white}{\faGraduationCap~Compétences Acquises}},
    fonttitle=\bfseries\large,
    coltitle=white,
    attach boxed title to top left={yshift=-3mm, xshift=5mm},
    boxed title style={colback=bleuMarine, arc=2mm}
  ]

  Ce projet nous a permis de développer et consolider de nombreuses compétences techniques et transversales :

  \vspace{0.3cm}

  \begin{minipage}[t]{0.48\textwidth}
    \textbf{\textcolor{bleuMarine}{Compétences Techniques}}
    \begin{itemize}[leftmargin=*, itemsep=2pt, label=\textcolor{bleuMarine}{\faCode}]
      \item Architecture microservices distribuée
      \item Développement Python (FastAPI, LangChain)
      \item Développement Java (Spring Boot)
      \item Intégration de LLM locaux (Ollama)
      \item Systèmes RAG et embeddings vectoriels
      \item Conteneurisation Docker
      \item CI/CD avec GitHub Actions
      \item Tests de performance (JMeter)
    \end{itemize}
  \end{minipage}
  \hfill
  \begin{minipage}[t]{0.48\textwidth}
    \textbf{\textcolor{bleuMarine}{Compétences Transversales}}
    \begin{itemize}[leftmargin=*, itemsep=2pt, label=\textcolor{bleuMarine}{\faUsers}]
      \item Gestion de projet agile (Scrum)
      \item Travail collaboratif en équipe
      \item Compréhension du domaine médical
      \item Sensibilisation RGPD et sécurité
      \item Résolution de problèmes complexes
      \item Veille technologique IA
      \item Rédaction technique
      \item Communication et présentation
    \end{itemize}
  \end{minipage}

\end{tcolorbox}

\newpage

%==============================================================================
% Difficultés rencontrées
%==============================================================================

\begin{tcolorbox}[
    enhanced,
    colback=white,
    colframe=orangeWarning,
    boxrule=1.5pt,
    arc=3mm,
    left=10pt, right=10pt, top=10pt, bottom=10pt,
    title={\textcolor{white}{\faExclamationTriangle~Difficultés Rencontrées et Solutions}},
    fonttitle=\bfseries\large,
    coltitle=white,
    attach boxed title to top left={yshift=-3mm, xshift=5mm},
    boxed title style={colback=orangeWarning, arc=2mm}
  ]

  Le développement de DocQA-MS n'a pas été sans défis. Voici les principales difficultés rencontrées et les solutions apportées :

  \vspace{0.3cm}

  \renewcommand{\arraystretch}{1.4}
  \begin{tabular}{|>{\columncolor{orangeWarning!10}}p{5cm}|p{7.5cm}|}
    \hline
    \rowcolor{orangeWarning}
    \textcolor{white}{\textbf{Difficulté}} & \textcolor{white}{\textbf{Solution Apportée}} \\
    \hline
    \textbf{Intégration LLM local} : Configuration d'Ollama et gestion des ressources GPU/CPU & Optimisation des paramètres de contexte, utilisation de modèles quantifiés plus légers \\
    \hline
    \textbf{Communication inter-services} : Synchronisation et gestion des erreurs entre microservices & Architecture hybride sync (REST) et async (RabbitMQ), patterns de retry \\
    \hline
    \textbf{Temps de réponse LLM} : Latence de génération impactant l'UX & Affichage progressif (streaming), caching des embeddings, chunking optimisé \\
    \hline
    \textbf{Anonymisation précise} : Détection fiable des entités médicales spécifiques & Utilisation de modèles NER spécialisés, règles métier complémentaires \\
    \hline
    \textbf{CI/CD complexe} : Orchestration du build de 7 services différents & Workflows parallélisés, continue-on-error pour isolation des échecs \\
    \hline
  \end{tabular}

\end{tcolorbox}

\vspace{0.5cm}

%==============================================================================
% Perspectives d'évolution
%==============================================================================

\begin{tcolorbox}[
    enhanced,
    colback=white,
    colframe=bleuFonce,
    boxrule=1.5pt,
    arc=3mm,
    left=10pt, right=10pt, top=10pt, bottom=10pt,
    title={\textcolor{white}{\faRocket~Perspectives d'Évolution}},
    fonttitle=\bfseries\large,
    coltitle=white,
    attach boxed title to top left={yshift=-3mm, xshift=5mm},
    boxed title style={colback=bleuFonce, arc=2mm}
  ]

  DocQA-MS constitue une base solide pour de nombreuses évolutions futures :

  \vspace{0.3cm}

  \begin{minipage}[t]{0.48\textwidth}
    \textbf{\textcolor{bleuFonce}{Court terme (3-6 mois)}}
    \begin{itemize}[leftmargin=*, itemsep=2pt, label=\textcolor{bleuFonce}{\faAngleRight}]
      \item Support d'images médicales (OCR)
      \item Amélioration du modèle NER médical
      \item Interface de configuration admin
      \item Export des synthèses en PDF
      \item Authentification SSO
    \end{itemize}
  \end{minipage}
  \hfill
  \begin{minipage}[t]{0.48\textwidth}
    \textbf{\textcolor{bleuFonce}{Moyen terme (6-12 mois)}}
    \begin{itemize}[leftmargin=*, itemsep=2pt, label=\textcolor{bleuFonce}{\faAngleRight}]
      \item Orchestration Kubernetes
      \item Haute disponibilité (HA)
      \item Intégration avec DPI hospitaliers
      \item Support multilingue
      \item Fine-tuning du LLM sur corpus médical
    \end{itemize}
  \end{minipage}

  \vspace{0.4cm}

  \begin{minipage}[t]{0.48\textwidth}
    \textbf{\textcolor{bleuFonce}{Long terme (1-2 ans)}}
    \begin{itemize}[leftmargin=*, itemsep=2pt, label=\textcolor{bleuFonce}{\faAngleRight}]
      \item Certification dispositif médical
      \item Études cliniques de validation
      \item Partenariats avec établissements
      \item Interface vocale
      \item Alertes et recommandations proactives
    \end{itemize}
  \end{minipage}
  \hfill
  \begin{minipage}[t]{0.48\textwidth}
    \textbf{\textcolor{bleuFonce}{Vision à long terme}}
    \begin{itemize}[leftmargin=*, itemsep=2pt, label=\textcolor{bleuFonce}{\faAngleRight}]
      \item Référence en IA médicale locale
      \item Contribution à la souveraineté des données
      \item Amélioration de la qualité des soins
      \item Réduction du temps administratif
      \item Open source pour la communauté
    \end{itemize}
  \end{minipage}

\end{tcolorbox}

\vspace{0.5cm}

%==============================================================================
% Mot de la fin
%==============================================================================

\begin{tcolorbox}[
    enhanced,
    colback=bleuPrincipal!10,
    colframe=bleuPrincipal,
    boxrule=2pt,
    arc=5mm,
    left=15pt, right=15pt, top=15pt, bottom=15pt,
    shadow={3mm}{-3mm}{0mm}{black!20}
  ]

  \begin{center}
    {\Large\bfseries\textcolor{bleuFonce}{Mot de la Fin}}
  \end{center}

  \vspace{0.3cm}

  Ce projet de fin d'études a été une expérience enrichissante à tous points de vue. Au-delà des compétences techniques acquises en intelligence artificielle et en architecture distribuée, il nous a permis de travailler sur un sujet porteur de sens : \textbf{l'amélioration de l'accès à l'information médicale}.

  \vspace{0.3cm}

  Dans un contexte où l'IA révolutionne tous les secteurs, le domaine de la santé requiert une attention particulière à la \textbf{confidentialité} et à la \textbf{fiabilité}. DocQA-MS démontre qu'il est possible de tirer parti des avancées en traitement du langage naturel tout en respectant les contraintes strictes de protection des données patients.

  \vspace{0.3cm}

  Nous espérons que ce projet pourra inspirer d'autres initiatives alliant innovation technologique et responsabilité éthique. L'intelligence artificielle au service de la santé représente un formidable potentiel, à condition qu'elle soit développée avec rigueur et dans le respect des individus.

  \vspace{0.2cm}

  \begin{center}
    {\large\itshape\textcolor{bleuFonce}{"La technologie au service de l'humain, jamais l'inverse."}}
  \end{center}

  \vspace{0.3cm}

  \begin{flushright}
    {\itshape L'équipe DocQA-MS}\\
    {\small\textcolor{grisTexte}{Achraf ELHOUFI, Saad KARZOUZ, Yassir LAMBRASS, Anas ELMALYARI}}
  \end{flushright}

\end{tcolorbox}

%==============================================================================
%                     PAGE FINALE DU RAPPORT
%==============================================================================

\newpage
\thispagestyle{empty}

\begin{tikzpicture}[remember picture, overlay]

  %==========================================================================
  % FOND AVEC DÉGRADÉ
  %==========================================================================

  % Fond blanc de base
  \fill[white] (current page.south west) rectangle (current page.north east);

  % Grande forme géométrique bleue (côté gauche)
  \fill[bleuPrincipal]
  ([xshift=6cm]current page.north west) --
  (current page.north west) --
  (current page.south west) --
  ([xshift=3cm]current page.south west) --
  ([xshift=8cm, yshift=8cm]current page.south west) -- cycle;

  % Forme superposée plus claire
  \fill[bleuTurquoise, opacity=0.6]
  ([xshift=4cm]current page.north west) --
  (current page.north west) --
  ([yshift=5cm]current page.south west) --
  ([xshift=6cm, yshift=10cm]current page.south west) -- cycle;

  % Cercles décoratifs (réseau)
  \foreach \x/\y/\r in {
    2/12/0.8, 1/10/0.5, 0/8/0.6, 1.5/6/0.4, 0.5/4/0.7,
    -0.5/11/0.3, -1/9/0.5, -0.5/7/0.4, -1.5/5/0.6, 0/2/0.5
  } {
    \fill[white, opacity=0.15] ([xshift=\x cm, yshift=\y cm]current page.west) circle (\r cm);
  }

  %==========================================================================
  % BANDE DÉCORATIVE SUPÉRIEURE
  %==========================================================================

  \fill[bleuFonce]
  (current page.north west) --
  ([yshift=-1.2cm]current page.north west) --
  ([xshift=5cm, yshift=-1.2cm]current page.north west) --
  ([xshift=5cm, yshift=0cm]current page.north west) -- cycle;

  \fill[bleuFonce]
  (current page.north east) --
  ([yshift=-1.2cm]current page.north east) --
  ([xshift=-5cm, yshift=-1.2cm]current page.north east) --
  ([xshift=-5cm, yshift=0cm]current page.north east) -- cycle;

\end{tikzpicture}

%==============================================================================
% CONTENU DE LA PAGE FINALE
%==============================================================================

\vspace*{4cm}

\begin{center}

  % Logo stylisé
  \begin{tikzpicture}[scale=1.2]
    % Cercle extérieur
    \draw[bleuPrincipal, line width=4pt] (0,0) circle (2cm);
    \fill[white] (0,0) circle (1.6cm);
    % Document stylisé
    \fill[bleuPrincipal]
    (-0.5,0.8) -- (0.5,0.8) -- (0.5,-0.8) -- (-0.5,-0.8) -- cycle;
    % Lignes de texte
    \draw[white, line width=1.5pt] (-0.3,0.5) -- (0.3,0.5);
    \draw[white, line width=1.5pt] (-0.3,0.2) -- (0.3,0.2);
    \draw[white, line width=1.5pt] (-0.3,-0.1) -- (0.2,-0.1);
    % Point d'interrogation
    \node[text=bleuTurquoise, font=\bfseries\large] at (0.7,0) {?};
    % Points (microservices)
    \fill[bleuFonce] (-0.8,-0.5) circle (0.08);
    \fill[bleuFonce] (0.8,-0.5) circle (0.08);
    \fill[bleuFonce] (0,-1.2) circle (0.08);
    % Connexions
    \draw[bleuFonce, line width=1pt] (-0.8,-0.5) -- (0,-1.2);
    \draw[bleuFonce, line width=1pt] (0.8,-0.5) -- (0,-1.2);
  \end{tikzpicture}

  \vspace{1.5cm}

  % Nom de l'application
  {\fontsize{60}{72}\selectfont\textcolor{bleuFonce}{\textbf{Doc}}\textcolor{bleuPrincipal}{\textbf{QA-MS}}}

  \vspace{0.8cm}

  % Tagline
  {\LARGE\itshape\textcolor{grisFonce}{Système de Question-Réponse sur Documents Médicaux}}

  \vspace{0.5cm}

  {\large\textcolor{grisTexte}{Architecture Microservices | RAG | LLM Local}}

  \vspace{2cm}

  % Séparateur
  \begin{tikzpicture}
    \draw[bleuPrincipal, line width=2pt] (-4,0) -- (4,0);
    \fill[bleuPrincipal] (0,0) circle (0.15);
  \end{tikzpicture}

  \vspace{2cm}

  % Informations du projet
  \begin{tcolorbox}[
      enhanced,
      colback=bleuTresClair,
      colframe=bleuPrincipal,
      boxrule=1pt,
      arc=5mm,
      width=12cm,
      halign=center,
      left=15pt, right=15pt, top=15pt, bottom=15pt
    ]

    {\large\bfseries\textcolor{bleuFonce}{Projet Académique}}

    \vspace{0.3cm}

    {\normalsize\textcolor{grisTexte}{École Marocaine des Sciences de l'Ingénieur}}

    \vspace{0.2cm}

    {\normalsize\textcolor{grisTexte}{Département d'Informatique}}

    \vspace{0.2cm}

    {\normalsize\textcolor{grisTexte}{Filière : Ingénierie Informatique et Réseaux}}

    \vspace{0.5cm}

    {\large\bfseries\textcolor{bleuPrincipal}{Année Universitaire 2025 — 2026}}

  \end{tcolorbox}

  \vspace{2cm}

  % Tags technologiques
  \begin{tikzpicture}
    \node[fill=bleuPrincipal, text=white, rounded corners=15pt,
    inner xsep=12pt, inner ysep=6pt, font=\small] (github) {\faGithub~GitHub};
    \node[fill=bleuTurquoise, text=white, rounded corners=15pt,
    inner xsep=12pt, inner ysep=6pt, font=\small, right=0.3cm of github] (docker) {\faDocker~Docker};
    \node[fill=bleuMarine, text=white, rounded corners=15pt,
    inner xsep=12pt, inner ysep=6pt, font=\small, right=0.3cm of docker] (ai) {\faBrain~IA};
  \end{tikzpicture}

  \vspace{1.5cm}

  % Message final
  {\small\itshape\textcolor{grisTexte}{"L'intelligence artificielle au service de la santé, dans le respect de la vie privée."}}

\end{center}

%==============================================================================
%                     FIN DU DOCUMENT
%==============================================================================


\end{document}
