%==============================================================================
%                     CONCLUSION GÉNÉRALE
%==============================================================================

\chapter*{Conclusion Générale}
\addcontentsline{toc}{chapter}{Conclusion Générale}
\markboth{Conclusion Générale}{Conclusion Générale}

%==============================================================================
% Décoration de page
%==============================================================================

\begin{tikzpicture}[remember picture, overlay]
  % Bande verticale droite
  \fill[bleuPrincipal]
  ([xshift=-0.3cm]current page.north east) rectangle
  ([xshift=-0.6cm, yshift=-5cm]current page.north east);
  % Cercle décoratif
  \fill[bleuClair, opacity=0.2]
  ([xshift=-3cm, yshift=-3cm]current page.north east) circle (2cm);
\end{tikzpicture}

\vspace{-0.5cm}

%==============================================================================
% Citation de clôture
%==============================================================================

\begin{center}
  \begin{tikzpicture}
    \node[fill=bleuTresClair, rounded corners=5pt, inner sep=15pt,
    text width=12cm, align=center] {
      {\Large\color{bleuPrincipal}"}\hspace{0.1cm}
      {\itshape\color{grisFonce}L'intelligence artificielle est probablement l'événement le plus important de l'histoire de l'humanité. Malheureusement, elle pourrait aussi être le dernier, à moins que nous n'apprenions à éviter les risques.}
      \hspace{0.1cm}{\Large\color{bleuPrincipal}"}
      \\[0.3cm]
      {\small\color{bleuMarine}— Stephen Hawking, Physicien théoricien}
    };
  \end{tikzpicture}
\end{center}

\vspace{0.8cm}

%==============================================================================
% Synthèse du projet
%==============================================================================

\begin{tcolorbox}[
    enhanced,
    colback=white,
    colframe=bleuPrincipal,
    boxrule=2pt,
    arc=4mm,
    left=12pt, right=12pt, top=12pt, bottom=12pt,
    shadow={2mm}{-2mm}{0mm}{black!15}
  ]

  \begin{center}
    {\Large\bfseries\textcolor{bleuFonce}{Synthèse du Projet DocQA-MS}}
  \end{center}

  \vspace{0.3cm}

  Au terme de ce projet de fin d'études, nous avons conçu et développé \textbf{DocQA-MS}, un système de Question-Réponse sur Documents Médicaux basé sur une architecture microservices moderne. Ce projet ambitieux visait à répondre à un défi majeur du secteur de la santé : \textit{permettre aux professionnels de santé d'accéder rapidement à l'information médicale tout en garantissant la confidentialité absolue des données patients}.

  \vspace{0.3cm}

  Notre solution se distingue par plusieurs innovations techniques majeures :

  \begin{itemize}[leftmargin=*, itemsep=4pt, label=\textcolor{bleuPrincipal}{\faCheck}]
    \item \textbf{Un système RAG (Retrieval-Augmented Generation)} combinant la recherche sémantique vectorielle avec un modèle de langage local (Llama 3.1 via Ollama), garantissant que les données ne quittent jamais l'infrastructure locale
    \item \textbf{Une anonymisation automatique} basée sur la reconnaissance d'entités nommées (NER), assurant la conformité RGPD avant tout traitement
    \item \textbf{Une architecture microservices distribuée} avec 7 services spécialisés (Python/FastAPI et Java/Spring Boot), assurant scalabilité et maintenabilité
    \item \textbf{Un système d'audit complet} traçant toutes les opérations pour répondre aux exigences de sécurité du domaine médical
    \item \textbf{Une infrastructure conteneurisée} (Docker Compose) avec pipelines CI/CD (GitHub Actions) pour un déploiement reproductible
  \end{itemize}

\end{tcolorbox}

\vspace{0.5cm}

%==============================================================================
% Objectifs atteints
%==============================================================================

\begin{tcolorbox}[
    enhanced,
    colback=bleuTresClair,
    colframe=bleuPrincipal,
    boxrule=0pt,
    borderline west={4pt}{0pt}{bleuPrincipal},
    arc=0mm,
    left=12pt, right=12pt, top=10pt, bottom=10pt
  ]

  {\large\bfseries\textcolor{bleuFonce}{\faBullseye~Objectifs Atteints}}

  \vspace{0.3cm}

  L'ensemble des objectifs définis en début de projet ont été atteints avec succès :

  \vspace{0.2cm}

  \begin{minipage}[t]{0.48\textwidth}
    \textbf{\textcolor{bleuPrincipal}{Objectifs Fonctionnels}}
    \begin{itemize}[leftmargin=*, itemsep=2pt, label=\textcolor{bleuPrincipal}{\faCheckCircle}]
      \item Ingestion multi-formats (PDF, TXT, DOCX)
      \item Anonymisation NER automatique
      \item Indexation sémantique vectorielle
      \item Q\&A en langage naturel (RAG)
      \item Synthèse comparative multi-documents
      \item Système d'audit et traçabilité
      \item Interface React moderne
    \end{itemize}
  \end{minipage}
  \hfill
  \begin{minipage}[t]{0.48\textwidth}
    \textbf{\textcolor{bleuPrincipal}{Objectifs Techniques}}
    \begin{itemize}[leftmargin=*, itemsep=2pt, label=\textcolor{bleuPrincipal}{\faCheckCircle}]
      \item Architecture 7 microservices
      \item Backend Python (FastAPI) + Java (Spring)
      \item LLM local (Ollama/Llama 3.1)
      \item Infrastructure Docker Compose
      \item Message broker (RabbitMQ)
      \item Base de données (PostgreSQL)
      \item CI/CD GitHub Actions
    \end{itemize}
  \end{minipage}

\end{tcolorbox}

\vspace{0.5cm}

%==============================================================================
% Compétences acquises
%==============================================================================

\begin{tcolorbox}[
    enhanced,
    colback=white,
    colframe=bleuMarine,
    boxrule=1.5pt,
    arc=3mm,
    left=10pt, right=10pt, top=10pt, bottom=10pt,
    title={\textcolor{white}{\faGraduationCap~Compétences Acquises}},
    fonttitle=\bfseries\large,
    coltitle=white,
    attach boxed title to top left={yshift=-3mm, xshift=5mm},
    boxed title style={colback=bleuMarine, arc=2mm}
  ]

  Ce projet nous a permis de développer et consolider de nombreuses compétences techniques et transversales :

  \vspace{0.3cm}

  \begin{minipage}[t]{0.48\textwidth}
    \textbf{\textcolor{bleuMarine}{Compétences Techniques}}
    \begin{itemize}[leftmargin=*, itemsep=2pt, label=\textcolor{bleuMarine}{\faCode}]
      \item Architecture microservices distribuée
      \item Développement Python (FastAPI, LangChain)
      \item Développement Java (Spring Boot)
      \item Intégration de LLM locaux (Ollama)
      \item Systèmes RAG et embeddings vectoriels
      \item Conteneurisation Docker
      \item CI/CD avec GitHub Actions
      \item Tests de performance (JMeter)
    \end{itemize}
  \end{minipage}
  \hfill
  \begin{minipage}[t]{0.48\textwidth}
    \textbf{\textcolor{bleuMarine}{Compétences Transversales}}
    \begin{itemize}[leftmargin=*, itemsep=2pt, label=\textcolor{bleuMarine}{\faUsers}]
      \item Gestion de projet agile (Scrum)
      \item Travail collaboratif en équipe
      \item Compréhension du domaine médical
      \item Sensibilisation RGPD et sécurité
      \item Résolution de problèmes complexes
      \item Veille technologique IA
      \item Rédaction technique
      \item Communication et présentation
    \end{itemize}
  \end{minipage}

\end{tcolorbox}

\newpage

%==============================================================================
% Difficultés rencontrées
%==============================================================================

\begin{tcolorbox}[
    enhanced,
    colback=white,
    colframe=orangeWarning,
    boxrule=1.5pt,
    arc=3mm,
    left=10pt, right=10pt, top=10pt, bottom=10pt,
    title={\textcolor{white}{\faExclamationTriangle~Difficultés Rencontrées et Solutions}},
    fonttitle=\bfseries\large,
    coltitle=white,
    attach boxed title to top left={yshift=-3mm, xshift=5mm},
    boxed title style={colback=orangeWarning, arc=2mm}
  ]

  Le développement de DocQA-MS n'a pas été sans défis. Voici les principales difficultés rencontrées et les solutions apportées :

  \vspace{0.3cm}

  \renewcommand{\arraystretch}{1.4}
  \begin{tabular}{|>{\columncolor{orangeWarning!10}}p{5cm}|p{7.5cm}|}
    \hline
    \rowcolor{orangeWarning}
    \textcolor{white}{\textbf{Difficulté}} & \textcolor{white}{\textbf{Solution Apportée}} \\
    \hline
    \textbf{Intégration LLM local} : Configuration d'Ollama et gestion des ressources GPU/CPU & Optimisation des paramètres de contexte, utilisation de modèles quantifiés plus légers \\
    \hline
    \textbf{Communication inter-services} : Synchronisation et gestion des erreurs entre microservices & Architecture hybride sync (REST) et async (RabbitMQ), patterns de retry \\
    \hline
    \textbf{Temps de réponse LLM} : Latence de génération impactant l'UX & Affichage progressif (streaming), caching des embeddings, chunking optimisé \\
    \hline
    \textbf{Anonymisation précise} : Détection fiable des entités médicales spécifiques & Utilisation de modèles NER spécialisés, règles métier complémentaires \\
    \hline
    \textbf{CI/CD complexe} : Orchestration du build de 7 services différents & Workflows parallélisés, continue-on-error pour isolation des échecs \\
    \hline
  \end{tabular}

\end{tcolorbox}

\vspace{0.5cm}

%==============================================================================
% Perspectives d'évolution
%==============================================================================

\begin{tcolorbox}[
    enhanced,
    colback=white,
    colframe=bleuFonce,
    boxrule=1.5pt,
    arc=3mm,
    left=10pt, right=10pt, top=10pt, bottom=10pt,
    title={\textcolor{white}{\faRocket~Perspectives d'Évolution}},
    fonttitle=\bfseries\large,
    coltitle=white,
    attach boxed title to top left={yshift=-3mm, xshift=5mm},
    boxed title style={colback=bleuFonce, arc=2mm}
  ]

  DocQA-MS constitue une base solide pour de nombreuses évolutions futures :

  \vspace{0.3cm}

  \begin{minipage}[t]{0.48\textwidth}
    \textbf{\textcolor{bleuFonce}{Court terme (3-6 mois)}}
    \begin{itemize}[leftmargin=*, itemsep=2pt, label=\textcolor{bleuFonce}{\faAngleRight}]
      \item Support d'images médicales (OCR)
      \item Amélioration du modèle NER médical
      \item Interface de configuration admin
      \item Export des synthèses en PDF
      \item Authentification SSO
    \end{itemize}
  \end{minipage}
  \hfill
  \begin{minipage}[t]{0.48\textwidth}
    \textbf{\textcolor{bleuFonce}{Moyen terme (6-12 mois)}}
    \begin{itemize}[leftmargin=*, itemsep=2pt, label=\textcolor{bleuFonce}{\faAngleRight}]
      \item Orchestration Kubernetes
      \item Haute disponibilité (HA)
      \item Intégration avec DPI hospitaliers
      \item Support multilingue
      \item Fine-tuning du LLM sur corpus médical
    \end{itemize}
  \end{minipage}

  \vspace{0.4cm}

  \begin{minipage}[t]{0.48\textwidth}
    \textbf{\textcolor{bleuFonce}{Long terme (1-2 ans)}}
    \begin{itemize}[leftmargin=*, itemsep=2pt, label=\textcolor{bleuFonce}{\faAngleRight}]
      \item Certification dispositif médical
      \item Études cliniques de validation
      \item Partenariats avec établissements
      \item Interface vocale
      \item Alertes et recommandations proactives
    \end{itemize}
  \end{minipage}
  \hfill
  \begin{minipage}[t]{0.48\textwidth}
    \textbf{\textcolor{bleuFonce}{Vision à long terme}}
    \begin{itemize}[leftmargin=*, itemsep=2pt, label=\textcolor{bleuFonce}{\faAngleRight}]
      \item Référence en IA médicale locale
      \item Contribution à la souveraineté des données
      \item Amélioration de la qualité des soins
      \item Réduction du temps administratif
      \item Open source pour la communauté
    \end{itemize}
  \end{minipage}

\end{tcolorbox}

\vspace{0.5cm}

%==============================================================================
% Mot de la fin
%==============================================================================

\begin{tcolorbox}[
    enhanced,
    colback=bleuPrincipal!10,
    colframe=bleuPrincipal,
    boxrule=2pt,
    arc=5mm,
    left=15pt, right=15pt, top=15pt, bottom=15pt,
    shadow={3mm}{-3mm}{0mm}{black!20}
  ]

  \begin{center}
    {\Large\bfseries\textcolor{bleuFonce}{Mot de la Fin}}
  \end{center}

  \vspace{0.3cm}

  Ce projet de fin d'études a été une expérience enrichissante à tous points de vue. Au-delà des compétences techniques acquises en intelligence artificielle et en architecture distribuée, il nous a permis de travailler sur un sujet porteur de sens : \textbf{l'amélioration de l'accès à l'information médicale}.

  \vspace{0.3cm}

  Dans un contexte où l'IA révolutionne tous les secteurs, le domaine de la santé requiert une attention particulière à la \textbf{confidentialité} et à la \textbf{fiabilité}. DocQA-MS démontre qu'il est possible de tirer parti des avancées en traitement du langage naturel tout en respectant les contraintes strictes de protection des données patients.

  \vspace{0.3cm}

  Nous espérons que ce projet pourra inspirer d'autres initiatives alliant innovation technologique et responsabilité éthique. L'intelligence artificielle au service de la santé représente un formidable potentiel, à condition qu'elle soit développée avec rigueur et dans le respect des individus.

  \vspace{0.2cm}

  \begin{center}
    {\large\itshape\textcolor{bleuFonce}{"La technologie au service de l'humain, jamais l'inverse."}}
  \end{center}

  \vspace{0.3cm}

  \begin{flushright}
    {\itshape L'équipe DocQA-MS}\\
    {\small\textcolor{grisTexte}{Achraf ELHOUFI, Saad KARZOUZ, Yassir LAMBRASS, Anas ELMALYARI}}
  \end{flushright}

\end{tcolorbox}

%==============================================================================
%                     PAGE FINALE DU RAPPORT
%==============================================================================

\newpage
\thispagestyle{empty}

\begin{tikzpicture}[remember picture, overlay]

  %==========================================================================
  % FOND AVEC DÉGRADÉ
  %==========================================================================

  % Fond blanc de base
  \fill[white] (current page.south west) rectangle (current page.north east);

  % Grande forme géométrique bleue (côté gauche)
  \fill[bleuPrincipal]
  ([xshift=6cm]current page.north west) --
  (current page.north west) --
  (current page.south west) --
  ([xshift=3cm]current page.south west) --
  ([xshift=8cm, yshift=8cm]current page.south west) -- cycle;

  % Forme superposée plus claire
  \fill[bleuTurquoise, opacity=0.6]
  ([xshift=4cm]current page.north west) --
  (current page.north west) --
  ([yshift=5cm]current page.south west) --
  ([xshift=6cm, yshift=10cm]current page.south west) -- cycle;

  % Cercles décoratifs (réseau)
  \foreach \x/\y/\r in {
    2/12/0.8, 1/10/0.5, 0/8/0.6, 1.5/6/0.4, 0.5/4/0.7,
    -0.5/11/0.3, -1/9/0.5, -0.5/7/0.4, -1.5/5/0.6, 0/2/0.5
  } {
    \fill[white, opacity=0.15] ([xshift=\x cm, yshift=\y cm]current page.west) circle (\r cm);
  }

  %==========================================================================
  % BANDE DÉCORATIVE SUPÉRIEURE
  %==========================================================================

  \fill[bleuFonce]
  (current page.north west) --
  ([yshift=-1.2cm]current page.north west) --
  ([xshift=5cm, yshift=-1.2cm]current page.north west) --
  ([xshift=5cm, yshift=0cm]current page.north west) -- cycle;

  \fill[bleuFonce]
  (current page.north east) --
  ([yshift=-1.2cm]current page.north east) --
  ([xshift=-5cm, yshift=-1.2cm]current page.north east) --
  ([xshift=-5cm, yshift=0cm]current page.north east) -- cycle;

\end{tikzpicture}

%==============================================================================
% CONTENU DE LA PAGE FINALE
%==============================================================================

\vspace*{4cm}

\begin{center}

  % Logo stylisé
  \begin{tikzpicture}[scale=1.2]
    % Cercle extérieur
    \draw[bleuPrincipal, line width=4pt] (0,0) circle (2cm);
    \fill[white] (0,0) circle (1.6cm);
    % Document stylisé
    \fill[bleuPrincipal]
    (-0.5,0.8) -- (0.5,0.8) -- (0.5,-0.8) -- (-0.5,-0.8) -- cycle;
    % Lignes de texte
    \draw[white, line width=1.5pt] (-0.3,0.5) -- (0.3,0.5);
    \draw[white, line width=1.5pt] (-0.3,0.2) -- (0.3,0.2);
    \draw[white, line width=1.5pt] (-0.3,-0.1) -- (0.2,-0.1);
    % Point d'interrogation
    \node[text=bleuTurquoise, font=\bfseries\large] at (0.7,0) {?};
    % Points (microservices)
    \fill[bleuFonce] (-0.8,-0.5) circle (0.08);
    \fill[bleuFonce] (0.8,-0.5) circle (0.08);
    \fill[bleuFonce] (0,-1.2) circle (0.08);
    % Connexions
    \draw[bleuFonce, line width=1pt] (-0.8,-0.5) -- (0,-1.2);
    \draw[bleuFonce, line width=1pt] (0.8,-0.5) -- (0,-1.2);
  \end{tikzpicture}

  \vspace{1.5cm}

  % Nom de l'application
  {\fontsize{60}{72}\selectfont\textcolor{bleuFonce}{\textbf{Doc}}\textcolor{bleuPrincipal}{\textbf{QA-MS}}}

  \vspace{0.8cm}

  % Tagline
  {\LARGE\itshape\textcolor{grisFonce}{Système de Question-Réponse sur Documents Médicaux}}

  \vspace{0.5cm}

  {\large\textcolor{grisTexte}{Architecture Microservices | RAG | LLM Local}}

  \vspace{2cm}

  % Séparateur
  \begin{tikzpicture}
    \draw[bleuPrincipal, line width=2pt] (-4,0) -- (4,0);
    \fill[bleuPrincipal] (0,0) circle (0.15);
  \end{tikzpicture}

  \vspace{2cm}

  % Informations du projet
  \begin{tcolorbox}[
      enhanced,
      colback=bleuTresClair,
      colframe=bleuPrincipal,
      boxrule=1pt,
      arc=5mm,
      width=12cm,
      halign=center,
      left=15pt, right=15pt, top=15pt, bottom=15pt
    ]

    {\large\bfseries\textcolor{bleuFonce}{Projet Académique}}

    \vspace{0.3cm}

    {\normalsize\textcolor{grisTexte}{École Marocaine des Sciences de l'Ingénieur}}

    \vspace{0.2cm}

    {\normalsize\textcolor{grisTexte}{Département d'Informatique}}

    \vspace{0.2cm}

    {\normalsize\textcolor{grisTexte}{Filière : Ingénierie Informatique et Réseaux}}

    \vspace{0.5cm}

    {\large\bfseries\textcolor{bleuPrincipal}{Année Universitaire 2025 — 2026}}

  \end{tcolorbox}

  \vspace{2cm}

  % Tags technologiques
  \begin{tikzpicture}
    \node[fill=bleuPrincipal, text=white, rounded corners=15pt,
    inner xsep=12pt, inner ysep=6pt, font=\small] (github) {\faGithub~GitHub};
    \node[fill=bleuTurquoise, text=white, rounded corners=15pt,
    inner xsep=12pt, inner ysep=6pt, font=\small, right=0.3cm of github] (docker) {\faDocker~Docker};
    \node[fill=bleuMarine, text=white, rounded corners=15pt,
    inner xsep=12pt, inner ysep=6pt, font=\small, right=0.3cm of docker] (ai) {\faBrain~IA};
  \end{tikzpicture}

  \vspace{1.5cm}

  % Message final
  {\small\itshape\textcolor{grisTexte}{"L'intelligence artificielle au service de la santé, dans le respect de la vie privée."}}

\end{center}

%==============================================================================
%                     FIN DU DOCUMENT
%==============================================================================
